\lesson{22}{Thr Nov 20 2025 14:30}{}

\begin{remark} \leavevmode
    
    If $ T $ is invertible and $ P(T) = 0 $,
    \[
        T^m + a_{m-1}T^{m-1} + \cdots + a_1T + a_0 I = 0, \hspace{1em} a_0 \neq 0
    \]

    then 
    \[
        T^{-1}(T^m + \cdots + a_1T + a_0I) = 0
    \]

    so $ T^{m-1} + a_{m-1}T^{m-2} + \cdots + a_1I + a_0T^{-1} = 0 $
    
    Move every term except the last term to the other side and divide by $ a_0 $.
    \[
        T^{-1} = -\frac{1}{a_0}T^{m-1} - \frac{a_{m-1}}{a_0}T^{m-2} \cdots -\frac{a_1}{a_0}I
    \]
\end{remark}

\begin{theorem} \leavevmode
    
    \begin{itemize}
        \item 
            $ T $ is diagonalizable $ \iff $ the minimal polynomial of $ T $ is a product of distinct linear polynomial.

        \item 
            There is a basis with respect to which $ M(T) $ is upper-triangular $ \iff $ minimal polynomial of $ T $ completely factors.
    \end{itemize}

\end{theorem}

\vspace{2em}
\underbar{\textbf{Minimal Polynomial of upper triangular matrices:}}

\begin{eg}
    
    $ T \in L(\mathbb{C}^3) $

    $ T(x, y, z) = (x + y + 5z, y + 4z, -z) $

    $ M(T) $ with respect to standard basis:
    \[
        M(T) = \begin{bmatrix}
          1 & 1 & 5 \\
          0 & 1 & 4 \\
          0 & 0 & -1 \\
        \end{bmatrix}
    \]

    eigenvalue: $ 1, -1 $, $ p(x) $ minimal polynomial.

    \vspace{1em}
    By theorem 5.13, $ (x-1)(x+1) $ is a factor of $ p(x) $. Also, $ \text{deg }p(x)\leqslant 3 $.

    So now check 
    \[
        M((T-I)(T+I)) = 
        \begin{bmatrix}
          0 & 1 & 5 \\
          0 & 0 & 4 \\
          0 & 0 & -2 \\
        \end{bmatrix}
        \begin{bmatrix}
          2 & 1 & 5 \\
          0 & 2 & 4 \\
          0 & 0 & 0 \\
        \end{bmatrix}
        =
        \begin{bmatrix}
          0 & 2 & 4 \\
          0 & 0 & 0 \\
          0 & 0 & 0 \\
        \end{bmatrix}
    \]

    Since every zero of $ p(x) $ is an eigenvalue and since eigenvalue of $ T = {1, -1} $. So
    \[
        p(x) = (x-1)^2(x+1) \hspace{1em} \text{or } \hspace{1em} p(x) = (x-1)(x+1)^2
    \]
\end{eg}


\setcounter{chapter}{7}
\chapter{Operators on Complex Vector Spaces}
\section{Generalized Eigenvectors and Nilpotent Operators}

\begin{definition}[generalized eigenvector] \leavevmode
    
    $ \lambda $: an eigenvalue for $ T $

    $ v \neq 0 $: a general eigenvector for $ \lambda $ if there is $ j \geqslant 1 $ such that
    \[
        (T-\lambda I)^j v = 0
    \]

\end{definition}


\begin{eg}

    $ T(x, y, z) = (2z, 3y, 0) $, $ T \in L(\mathbb{C}^3) $

    \[
        M(T) = \begin{bmatrix}
          0 & 0 & 2 \\
          0 & 3 & 0 \\
          0 & 0 & 0 \\
        \end{bmatrix}
    \]

    the eigenvalues: 0, 3

    Let $ \lambda = 0 $, eigenvectors: $ E(0,T) = \text{span }\{(1, 0, 0)\} $

    \vspace{1em}
    Generalized eigenvector: $ Tv = 0 $ or $ T^2v = 0 $ or $ T^3v = 0 $.

    \vspace{1em}
    $ T^2 = (0, 9y, 0) $. For example, (0,0,1) is a generalized eigenvector corresponding to eigenvalue 0.

\end{eg}


\section{Generalized Eigenspace Decomposition}
\begin{definition}[generalized eigenspace] \leavevmode
    
    For $ \lambda $, 
    \begin{align*}
        G(\lambda, T)
        &= \{\text{generalized eigenvector }_{\text{for }\lambda} \cup \{0\}\} \\
        &=\text{null }(T - \lambda I) \cup \text{null }(T - \lambda I)^2 \cup \cdots \cup \text{null }(T-\lambda I)^m \cup \cdots 
    \end{align*}

\end{definition}

\begin{theorem}
    
    Let $ S = T - \lambda I $, then
    \begin{enumerate}
        \item $ \text{null }S^i \subseteq \text{null }S^{i+1} $
        \item If $ \text{null }S^i = \text{null }S^{i+1} $, then $ \text{null }S^i = \text{null }S^{i+1} = \text{null }S^{i+2} \cdots $
    \end{enumerate}

\end{theorem}