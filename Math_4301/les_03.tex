\lesson{3}{Tue Sep 2 2025 14:30}{}

\begin{definition} \leavevmode

    \noindent Given $ V $ a vector space over $ F $. $ U_1, \cdots U_m $ are subspaces of $ V $, then 
    \[
        U_1 + \cdots U_m: = \{u_1 + \cdots u_m | u_1 \in U_1, \cdots u_m \in U_m\}
    \]
    is a subspace of V.
\end{definition}


\vspace{1em}
\begin{eg}
    Given:
    \[
        U_1 = \{(x_1, x_2, x_{3}) | x_1=0\}
    \]
    \[
        U_2 = \{(x_1, x_2, x_3) | x_2+x_3 = 0\}
    \]
    then, $ U_1+U_2 = \mathbb{R}^3 $.

    $ (x_1, x_2, x_3) \in \mathbb{R}^3 $, and $ (x_1, x_2, x_3) = (0, x_2-1, x_3+1) + (x_1, 1, -1) $. First element in $ U_1 $, second in $ U_2 $.

\end{eg}

\begin{theorem}
    $ U_1 + \cdots + U_m $ is the smallest subspace of $ V $ which contains $ U_1, \cdots U_m $.
\end{theorem}

\vspace{1em}
\begin{proof}
    \begin{enumerate}
        \item $ u_i \in U_1 + \cdots + U_m $ for each $ i $:
        
        if $ u_i \in U_i $, then $ u_i = 0+0+ \cdots  + u_i + \cdots + 0 $ (First element in $ U_1 $, second in $ U_2 \cdots $ )

        \item $ U_1 + \cdots + U_m $ is a subspace:
        
        \begin{itemize}
            \item $ 0 = 0+ \cdots +0 $
            \item If $ u = u_1 + \cdots +u_m \in U_1 + \cdots + U_m $, then
                \begin{align*}
                    au & = a(u_1 + \cdots  + u_m) \\ 
                    & = (au_1) + \cdots + (au_m) \\ 
                    & \in U_1 + \cdots  + U_m \\
                \end{align*}
            \item if $ u = u_1 + \cdots +u_m \in U_1 + \cdots + U_m$, $ v = v_1 + \cdots  + v_m \in U_1 + \cdots + U_m$, then
            \begin{align*}
                u+v &= (u_1 + \cdots +u_m) + (v_1 + \cdots +v_m) \\
                & = (u_1+v_1) + \cdots + (u_m + v_m) \in U_1 + \cdots + U_m\\
            \end{align*}
        \end{itemize}

        \item If $ W $ is a subspace of $ V $ that contains $ U_1, \cdots ,U_m $, then $ U_1 + \cdots + U_m \subseteq W $:
        
            If $ u = u_1 + \cdots +u_m \in U_1 + \cdots +U_m $. And $ U_i \in W, i = 1, \cdots , m $, then $ u_i \in W $. So $ U_1 + \cdots + U_m \subseteq W $
    \end{enumerate}
\end{proof}

\begin{definition} Direct Sum \leavevmode

    $ U_1, \cdots U_m $ subspaces of $ V $. The sum $ U_1 + \cdots + U_m $ is called a direct sum if each vector $ v \in U_1 + \cdots + U_m $ can be written in a unique way as 
    \[
        v = u_1 + \cdots + u_m \hspace{2em} u_i \in U_i
    \]

    The notation is $ U \bigoplus \cdots \bigoplus U_m $
\end{definition}

\begin{eg}
    \begin{enumerate}
        \item $ V = \mathbb{R}^3 $, $ U_1 = \{(x_1, x_2, x_3) | x_1 = 0\} $
        
            $ U_2 =  \{(x_1, x_2, x_3) | x_2 + x_3 = 0\}$

            We know $ U_1 + U_2 = \mathbb{R}^3 $.

            is this a direct sum?

            $ (1, 0, 0) = (0, -1, 1) \in U_1 + (1, 1, -1) \in U_2 = (0, -2, 2) \in U_1 + (1, 2, -2) \in U_2$ 

            There are two ways to express the sum, so it is not a direct sum.

        \item $ V = \mathbb{R}^2, U_1 + U_2 = \mathbb{R}^2 $.

            \[
                U_1 = \{(x_1,x_2) | x_1 = 0\}
            \]
            \[
                U_{2} = \{(x_1,x_2) | x_{1} + x_2 = 0\}
            \]

            Claim: $ U_1 + U_2 $ is a direct sum.
            \begin{explanation}
                If $ (a_1,b_1) \in U_1, (a_2, b_2) \in U_2 $, then $ (a_1, b_1) + (a_2, b_2) = (a_1', b_1') +(a_2', b_2') $

                $ a_1 = a_1' = 0 $
                $ b_2 = -a_2, b_2'=-a_2' $

                \vspace{1em}
                $ \implies (0, b_1) + (a_2, -a_2) = (0, b_1') + (a'_2, -a'_2) \implies a_2 = a_2'$ and $ b_1 - a_2 = b_1' - a_2' \implies b_1 = b_1' $

                \vspace{1em}
                So $ (a_1, b_1) = (a_1', b_1') $

                $ \mathbb{R}^2 = U_{1} + U_{2} $
            \end{explanation}
    \end{enumerate}
\end{eg}


\begin{theorem}
    $ U_1 + \cdots + U_m $ is a direct sum if and only if $ 0 = u_1 + \cdots  + u_m, u_i \in U_i $ implies $ u_1 = u_2 = \cdots =u_m = 0 $
\end{theorem}

\vspace{1em}
\begin{proof} \leavevmode

    $ \implies 0 = 0 + \cdots +0 = u_1 + \cdots +u_m $

    Since we have a direct sum, we conclude $ 0 = u_{1}, 0 = u_2, \cdots 0 = u_m $

    \vspace{1em}
    $ \impliedby V = U_1 + \cdots +U_m = U_1' + \cdots  + U_m', u_i, u_i' \in U_i$

    Then,
    \begin{align*}
        0 & = (u_1' + \cdots +u_{m}') - (u_1 + \cdots +u_{m}) \\
        & = (u_1' -u_1 ) + \cdots + (u_m' -u_{m} ) | u_i'-u_i \in U_i \\
    \end{align*}

    So $ u_i - u_i' = 0  $ for all $ i $.

    So $ u_i' = u_i $
\end{proof}


\vspace{2em}
\begin{theorem}
    If $ U, W $ are subspaces of $ V_1 $, then $ U + W $ is a direct sum if and only if $ U \cap W = \{0\} $
\end{theorem}

\vspace{1em}
\begin{proof}
    $ \implies $ Suppose $ U+ W $ is a direct sum, and $ v \in U \cap W $, then $ v \in U $ and $ v \in W $, so $ -v \in W $.
    
    So,
    \[
        0 = v + (-v)
    \].

    So by Theorem 3 $ v = 0 $ and $ -v = 0 $.


    \vspace{1em}
    $ \impliedby $ Suppose $ U \cap W = \{0\}$ and $ 0 = u + w, u \in U, w \in W $.

    Then $ -w = u \implies -w \in U \implies w \in U$.

    Combine These two condition together, $ w \in U \cap W $.

    \vspace{1em}
    So $ w = 0 $, $ u = 0 $. By theorem 3 we have a direct sum.
\end{proof}

\vspace{2em}
\begin{eg} \leavevmode

    $ V = \mathbb{R}^3, U = \{(x_1, x_2, x_3) | x_1 = 0\}, W = \{(x_1, x_2, x_3) | x_2 + x_3 = 0\} $.
    
    $U + V  $ is \underbar{not} a direct sum as $ 0 \neq (0, 1, -1) \in U \cap W $

    \vspace{2em}
    What if we have $ U_1, U_2, U_3 $
    \begin{enumerate}
        \item 
            $ U_1 + U_2 + U_3 $ is a direct sum implies
            \begin{align*}
                U_1 \cap U_2 = \{0\} \\
                U_{2} \cap U_{3} = \{0\} \\
                U_1 \cap U_{3} = \{0\} \\
            \end{align*}

            \begin{explanation}
                $ U_1 \cap U_2 = \{0\} $:

                If $ v \in U_1 \cap U_2 $, then $ v \in U_1 $ and $ v \in U_2 $, so $ -v \in U_2 $

                \[
                    0 = v \in U_1 + (-v) \in U_2 + 0 \in U_3
                \]

                Using theorem 3 it shows $ v = 0 $
            \end{explanation}

        \item 
            the converse is not true.
            \begin{explanation}
                $ V = \mathbb{R}^2 $

                \begin{align*}
                    &U_1 = \{(x_1, x_2) | x_2 = 0\} \\
                    &U_{2} = \{(x_1, x_2) | x_{1} = 0\} \\
                    &U_{3} = \{(x_1, x_2) | x_2 + x_1 = 0\} \\
                \end{align*}

                It is not a direct sum as:
                \[
                    0 = (1,0) \in U_1 + (0,-1) \in U_2 + (-1, 1) \in U_3
                \]
            \end{explanation}
    \end{enumerate}
\end{eg}


\chapter{Finite-Dimensional Vector Spaces}
\section{Span of vectors}

\begin{definition} \leavevmode

    \begin{itemize}
        \item A \underbar{list} of \underbar{length} $ n $ is an ordered collection of $ n $ elements.
            \begin{itemize}
                \item list of numbers: $ (1, 2, 1, 0) $
                \item list of vectors: $ v_1, v_2, \cdots v_n $
            \end{itemize}
        \item A \underbar{linear combination} of a list of $ v_1, \cdots  v_n $ of vectors in $ V $ is any vector of the form:
        \[
            a_1v_1 + \cdots + a_nv_n, \hspace{2em} a_1, \cdots a_n \in F
        \]
    \end{itemize}
\end{definition}

\begin{eg} \leavevmode

    $ V $: vector space of continuous functions $ f: \mathbb{R} \to \mathbb{R} $

    \vspace{1em}
    $ \cos(x), \sin(x) \in V $.

    \vspace{1em}
    $ a\cos x + b\sin x, \hspace{2em} a,b \in \mathbb{R} $
\end{eg}

\begin{definition} \leavevmode

    The set of all linear combinations of $ v_1, \cdots ,v_n $ span $ (v_1, \cdots , v_n) = \{a_1v_1+ \cdots + a_nv_n | a_i \in F\}$.
    $ (v_1, \cdots , v_n) $ is the \underbar{\textbf{linear span}} of $ v_1, \cdots , v_n $
\end{definition}

 