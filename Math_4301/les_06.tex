\lesson{6}{Thr Sep 11 2025 14:30}{}

\begin{note}
    If $ V$ is a finite-dimensional vector space and $U \subsetneq V $ is a subspace, then $ \dim U < \dim V $.
\end{note}

\begin{explanation}
    
    We know $ U $ is finite-dimensional, so it has a basis $ u_1, \cdots , u_{k} $. Then $ u_1, \cdots , u_{k} $ is linearly independent in $ U $, so it is also linearly independent in $ V $.

    \vspace{1em}
    If $ \text{span}(u_1, \cdots , u_{k}) = V$, then $ U = V $. So $ u_1, \cdots , u_{k} $ does not span $ V $. We can extend it to a basis for $ V $: $ u_1, \cdots , u_{k}, u_{k+1}, \cdots , u_{n} $. Then $ \dim V = n > k = \dim U $

\end{explanation}

\vspace{1em}
\begin{eg}
    $ P_2(\mathbb{R}) = \{a_2x^2 + a_1x + a_0 | a_0, a_1, a_2 \in \mathbb{R}\} $.

    Then $ \dim P_2(\mathbb{R}) = 3 $. The list $ 1, x, x^2 $ is a basis.

    \vspace{1em}
    Let $ U = \{ P(x) \in P_2(\mathbb{R}) | P(3) = 0 \} $. It is trivial to show it is a subspace.

    One basis for $ U $ is $ x-3, (x-3)^2  \in U $. $ U \neq P_2(\mathbb{R}) \implies \dim U \leqslant 2 $
\end{eg}


\vspace{2em}
\begin{prop} Dimension of a sum (2.43) \leavevmode

    If $ V $ is finite-dimensional, and $ U_1, U_2 \subseteq V $ are subspaces of $ V $, then 
    \[
        \dim (U_1 + U_2) = \dim U_1 + \dim U_2 - \dim (U_1 \cap U_2)
    \]

\end{prop}

\begin{proof} \leavevmode

    Let $ u_1, \cdots , u_{k} $ be a basis for $ U_1 \cap U_2 $. Since $ u_1, \cdots , u_{k} $ are linearly independent in $ U_1 $, we can extend them to $ u_1, \cdots , u_{k}, w_1, \cdots , w_{l} $ a basis for $ U_1 $. 
    Similarly, they can be extended to $ u_1, \cdots , u_{k}, v_1, \cdots , v_{t} $ a basis for $ U_2 $. (so $ \dim U_1 = k+l $, $ \dim U_2 = k+t $)

    \vspace{1em}
    Claim: $ u_1, \cdots , u_{k}, w_1, \cdots , w_{l}, v_1, \cdots , v_{t} $ is a basis for $ U_1 + U_2 $. 

    Need to show they are in and span in $ U_1 + U_2 $ and are linearly independent.

    \begin{enumerate}
        \item 
            \begin{itemize}
            \item $ u_1, \cdots ,u_{k}, v_1, \cdots , v_{t} \in U_2 \subseteq U_1 + U_2 $
            \item $ u_1, \cdots ,u_{k}, w_1, \cdots , w_{l} \in U_1 \subseteq U_1 + U_2 $
            \end{itemize}

            So $ u_1, \cdots , u_{k}, w_1, \cdots , w_{l}, v_1, \cdots , v_{t}  \in U_1 + U_2 $

        \item
            $ \text{span}(u_1, \cdots , u_{k}, w_1, \cdots , w_{l}, v_1, \cdots , v_{t}) = U_1 + U_2 $
            \begin{itemize}
                \item Any vector in $ U_1 + U_2 $ is of form $ v + v' $, $ v \in U_1, v' \in U_2 $.
                \item $ v = a_1u_1 + \cdots  + a_{k}u_{k} + b_1w_1 + \cdots + b_{l}w_{l} $
                
                        $ v' = c_1u_1 + \cdots + c_{k}u_{k} + d_1v_1 + \cdots + d_{t}v_{t} $

                \item $ v + v' \in \text{span}(u_1, \cdots , u_{k}, w_1, \cdots , w_{l}, v_1, \cdots , v_{t})  $
            \end{itemize}

        \item 
            They are linearly independent

            If $ a_1u_1 + \cdots + a_{k}u_{k} + b_1v_1 + \cdots + b_{t}v_{t} + c_1u_1 + \cdots + c_{l}w_{l} = 0 $

            $ a_1u_1 + \cdots + a_{k}u_{k} + b_1v_1 + \cdots + b_{t}v_{t} = - c_1u_1 + \cdots + c_{l}w_{l} $.

            LHS $ \in U_2 $, RHS in $ U_1 $ $ \implies $ LHS $ \in U_1 \cap U_2 \implies \text{LHS} \in \text{span}(u_1, \cdots , u_{k})$

            $ \implies a_1u_1 + \cdots  + a_{k}u_{k} + b_11v_1 + \cdots + b_{t}v_{t} = r_1u_1 + \cdots + r_{k}u_{k} $

            $ \implies (a_1-r_1)u_1 + \cdots  + (a_{k}-r_{k})u_{k} + b_1v_1 + \cdots  + b_{t}v_{t} = 0 $

            $ a_1-r_1 = \cdots = a_{k}-r_{k} = b_1 = \cdots = b_{t} = 0 $.

            \vspace{1em}
            Similarly, we can show $ c_1 = \cdots = c_{l} = 0 $. So they are linearly independent.
    \end{enumerate}
    So $ u_1, \cdots , u_{k}, w_1, \cdots , w_{l}, v_1, \cdots , v_{t} $ is a basis for $ U_1 + U_2 $.
    
\end{proof}

\vspace{2em}
\begin{corollary} \leavevmode
    
    \begin{enumerate}
        \item $ \dim (U_1 + U_2) \leqslant  \dim U_1 + \dim U_2 $
        \begin{itemize}
            \item For example, if $ \dim V = 3 $, $ U_1, U_2 $ are subspaces of $ \dim=2 $, then $ U_1 \cap U_2 \neq \{0\} $
        \end{itemize}
        \item  \begin{align*}
            \dim (U_1 + U_2) = \dim U_1 + \dim U_2 
            &\iff \dim U_1 \cap U_2 = 0 \\
            &\iff U_1 \cap U_2 = \{0\} \\
            &\iff \text{ we have a direct sum} U_1 \bigoplus U_2
        \end{align*}.
    \end{enumerate}

\end{corollary}

\chapter{Linear Maps}
\section{Vector Space of Linear Maps}

\begin{definition} Linear Map (3.1) \leavevmode
    
    $ V, W $ vector spaces over $ F $. Then a \underbar{\textbf{linear map}} is a function $ T: V \to W $ such that
    \begin{itemize}
        \item $ T(v_1 + v_2) = T(v_1) + T(v_2) $
        \item $ T(av) = aT(v) $, $ a \in F, v \in V $.
    \end{itemize}

\end{definition}

\vspace{1em}
\begin{notation} (3.2)

    $ L(V,W) =  $ set of all linear map from $ V $ to $ W $.
\end{notation}

\vspace{1em}
\begin{eg} \leavevmode
    \begin{enumerate}
        \item $ I : V \to V, I(v) = v$
        \item The zero function: $ T: V \to W, T(v) = 0 $
        \item $ V = P(\mathbb{R}) $ polynomials with real coefficients, $ T: V \to V, T(p(x)) = p'(x) $
        \item $ S: V \to V $, $ S(p(x)) = xp(x) $
        \item $ R: V \to V $, $ R(p(x)) = (p(x))^2 $. This is a \underbar{Counter Example}. \underbar{Not} linear.
        \begin{itemize}
            \item $ (3p(x))^2 \neq 3(p(x))^2 $
        \end{itemize}
        \item $ T: F^\infty \to F^n $, $ T(a_1, \cdots , a_{n}, a_{n+1}, \cdots ) = (a_1, \cdots , a_{n}) $.
    \end{enumerate}

\end{eg}

\vspace{2em}
\begin{theorem}
    
    If $ T: V \to W $ is a linear map, then $ T(0) = 0 $.

\end{theorem}

\begin{explanation}
    $ T(0) = T(0 + 0) = T(0) + T(0) $. Let $ T(0) = w $. $ w = w+w \implies w = 0 $.
\end{explanation}


\vspace{2em}
\begin{prop} linear map lemma (3.4) \leavevmode

    $ V, W $: two finite-dimensional vector space. $ v_1, \cdots , v_{n} $ is a basis for $ V $, and $ w_1, \cdots , w_{n} \in W $. Then there exists a \underbar{unique} linear map $ T: V \to W $ such that $ T(v_{i}) = w_{i} $.    
\end{prop}

\begin{eg}\leavevmode

    \begin{enumerate}
        \item 
            $ V = \mathbb{R}^2, W = \mathbb{R}^3 $

            $ v_1 = (1,2) $, $ v_2 = (2,4) \in V $

            $ w_1 = (1,0,0), w_2 = (0,0,1) \in W $

            \vspace{1em}
            Is there a linear map $ T: V \to W $

            \vspace{1em}
            No, $ v_2 = 2v_1 \implies T(v_2) = T(2v_1) = 2T(v_1) = 2w_2 $. But $ w_2 \neq 2w_1 $. 

        \item $ V = \mathbb{R}^3, W = \mathbb{R}^2 $
        
        $ v_1 = (1,0,0), v_2 = (0,1,0) \in V $
        $ w_1 = (10,), w_2 = (0,2) \in W $

        \vspace{1em}
        Is there a linear map $ T: V \to W $ with $ T(v_i) = u_{i} $?

        \item $ T: \mathbb{R}^2 \to \mathbb{R}^2 $
        $ T(x,y) = (x^2, y) $ not linear mapping.

        \item $ T(x,y) = (x + 2y, 2x - y) $ is linear mapping.
    \end{enumerate}
\end{eg}
