\lesson{11}{Thur Oct 2 2025 14:30}{}

\section{Duality}

Now assume $ U_1, \cdots , U_{n} $ are subspaces of $ V $.

\begin{align*}
    &\phi: U_1 \times \cdots \times U_{n} \to U_1 + \cdots + U_{n} \subset V \\
    &\phi(u_1, \cdots , u_{n}) = u_1 + \cdots + u_{n} \\
\end{align*}

$ \phi $ is a linear map which is always surjective.

$ \phi $ injective $ \iff $ $ \phi(u_1, \cdots , u_{n}) = 0 \implies u_1 = \cdots = u_{n} = 0  \iff U_1 + \cdots + U_{n}$ is a direct sum

\vspace{1em}
$ \phi $ is isomorphism $ \iff $ $ \phi $ is injective $ \iff U_1 + \cdots + U_{n}$ direct sum.



\begin{corollary}
    
    $ U_1 + \cdots + U_{n} $ is a direct sum if and only if 
    \[
        \dim (U_1 + \cdots + U_n) = \dim U_1 + \cdots  + \dim U_{n}
    \]

\end{corollary}


\begin{proof}
    
    If $ U_1 + \cdots + U_{n} $ is a direct sum, then $ \phi $ is isomorphism, so
    \[
        \dim (U_1 \times \cdots \times U_{n}) = \dim U_1 + \cdots + \dim U_{n}
    \]

    \vspace{1em}
    conversely,
    
    if $ \dim (U_1 \times \cdots \times U_{n}) = \sum_{i = 1}^{n} \dim U_{i} $, then
    \[
        \dim (U_1 + \cdots + U_{n}) = \dim U_1 \times \cdots \times U_{n}
    \]
    so any surjective linear map between them is an isomorphism.
    
\end{proof}

\begin{eg}

    $ U_1 = \{(x, y, 0)\} $ $ \dim = 2 $

    $ U_2 = \{(0, z, z)\} $ $ \dim  = 1 $

    $ U_3 = \{(0, 0, z)\} $ $ \dim  = 1 $

    \vspace{1em}
    $ \dim (U_1 + U_2 + U_3) \neq \dim U_1 + \dim U_2 + \dim U_3 $, so not a direct sum.

\end{eg}
\vspace{2em}
\vspace{2em}

\underbar{\textbf{Dual Space}}

Linear Functional: $ V: $ a vector space over $ F $.
A linear functional is an element of $ L(V, F) $.

\begin{eg} \leavevmode
    
    \begin{enumerate}
        \item $ F = \mathbb{R} $, $ V = \mathbb{R}^2 $, $ \phi: \mathbb{R}^2 \to \mathbb{R} $, $ (x, y) \to 2x - y $
        
        This is a linear Functional

        \item $ V = P(\mathbb{R}) $, $ \phi: P(\mathbb{R}) \to \mathbb{R} $
        \begin{itemize}
            \item $ \phi(p) = p(2) $
            \item $ \phi(p) = p'(3) $
            \item $ \phi(p) = \int_{0}^{1} p(x)dx $
        \end{itemize}

        All of them are linear functional.

        \item $ \phi: F^{n,n} \to F $
        
        $ \phi(A) =  $trace of $ A = A_{1,1} + A_{2,2} \cdots + A_{n,n} $
    \end{enumerate}

\end{eg}

\vspace{2em}
\begin{definition}
    
    The dual of $ V $ is the vector space $ L(V, F) $ i.e. the vector space of all linear functionals on $ V $. 

    Notation: $ V' $

    \vspace{1em}
    $ \dim V' = \dim V \times \dim F = \dim V $

\end{definition}

$ L(V, W) $ isomorphic to $ F^{n \times m} \implies \dim L(V, W) = \dim F_{n,m} = nm = \dim V \times \dim W $ 


\vspace{1em}
Fix a basis $ v_1, \cdots , v_{n} $ for $ V $, A natural basis for $ V' $:
$ 1 \leqslant i \leqslant n $, let $ \phi_{i}: V \to F $ be the linear functional given by
\[
    \phi_{i}(v_{j}) = 
    \begin{cases}
        0 \hspace{1em} j \neq i \\
        1 \hspace{1em} j = i \\
    \end{cases}
\]

Claim: $ \phi_1, \cdots , \phi_{n} $ are linearly independent.

$ \phi_1 ,\cdots , \phi_{n} \in V' \implies \phi_1, \cdots , \phi_{n} $ is a basis for $ V' $

\begin{proof}
    
    If $ a_1\phi_1 + \cdots + a_{n}\phi_{n} = 0 $ in $ V $ i.e. the zero linear linear function,
    then for every $ i \leqslant j\leqslant n $:
    \[
        (a_1\phi_1 + \cdots + a_{n}\phi_{n})(V_{j}) = 0 \implies a_1\phi_1(V_{j}) + \cdots + a_{j}\phi_{j}(V_{j}) + \cdots + a_{n}\phi_{n}(V_{j}) = a_{j}
    \]

    So $ a_{j} = 0 $ for every $ i \leqslant j \leqslant n $, which implies $ a_1 = ... = a_n = 0 $

\end{proof}

\begin{eg}
    
    $ F = \mathbb{R}, V = \mathbb{R}^2 $, $ v_1 = (1,0), v_2 = (0,1) $
    What is the dual basis?

    $ \phi_1, \phi_2: \mathbb{R}^2 \to \mathbb{R} $

    $ \phi_1(1,0) = 1 $, $ \phi_1(0,1) = 0$
    
    $ \phi_1(x,y) = \phi_1(x(1,0) + y(0,1)) = x\phi_1(1,0) + y \phi_1(0,1) = x $ 
    
    $ \phi_2(0,1) = 1, \phi_2(1,0) = 0 $

\end{eg}

\vspace{2em}
\underbar{\textbf{Dual Map}}

$ T: V \to W $ $ \phi W \to F $

dual map: $ T': W' \to V' $

$ T'(\phi) = \phi \circ T $ $ v_1, \cdots , v_{n} $ is basis for $ V $, $ w_1, \cdots , w_{m} $ is basis for $ W $.

$ M(T) $ is $ m \times n $ matrices

$ T' : W' \to V' $ $ \phi_1 ,\cdots  ,\phi_{n} $ dual basis for $ V' $, $ \psi_1, \cdots , \psi_{n} $ dual basis for $ W' $

Relation between $ M(T) $ and $ M'(T) $.


\begin{eg}

    Suppose $ M(T) = \begin{bmatrix}
      3 & 0 \\
      2 & 1 \\
      -4 & 2 \\
    \end{bmatrix} $

    Show $ M(T') = \phi_1 [3, 2, -4], \phi_2 [0,1,2] $

    $ T' \psi_1 = 3 \phi $. 

    $ T'(\psi_1) = 3\phi_1 $

    \[
        M(T) = \begin{bmatrix}
          3 & 0 \\
          2 & 1 \\
          -4 & 2 \\
        \end{bmatrix} 
        \implies Tv_1 = 3v_1 + 2v_2 - 4v_3, Tv_2 = v_2 + 2v_3
    \]

    $ T'(\psi_1) = \psi_1 \circ T $

    $ \begin{cases}
        \psi_1(w_1) = 1 \\
        \psi_1(w_2) = 0
    \end{cases} $

    \vspace{1em}
    $ T'(\psi_1) = \psi_1 \circ T \in L(V, F) $, so 
    
    \begin{align*}
        T'(\psi_1)(v_1) = \psi_1 \circ T(v_1) = \psi_1(3w_1 + 2w_2 - 4w_3) = 3 \\
        T'(\psi_1)(v_2) = \psi_1 \circ T(v_2) = \psi_1 (w_{2} + 2w_3) = 0
    \end{align*}

    $ \implies T'(\psi_1) = 3\phi_1 $

\end{eg}


\vspace{2em}
\begin{theorem}
    
    If $ \dim V = n $, $ \dim  W = m $, $ v_1, \cdots , v_{n} $ a basis for $ V $. $ w_1, \cdots , w_{m} $ basis for $ W $, $ \phi_1, \cdots , \phi_{n} $ dual basis for $ V' $, $ \psi_1 \cdots , \psi_{m} $ dual basis for $ W' $. $ T \in L(V, W) $.

    \[
        (M(T))^{t} = M(T')
    \]

\end{theorem}

\vspace{2em}
\begin{definition}
    
    If $ A $ is an $ n \times m $ matrix, $ A^T $ is the $ m \times n $ matrix given by
    \[
        (A^t)_{i,j} = A_{j,i}
    \]

\end{definition}


\begin{definition}
    
    $ A \in F^{n,m} $,

    Column rank of $ A $: dimension of the subspace of $ F^{n,1} $ spanned by the columns of $ A $.

    \vspace{1em}
    Row rank of $ A $: dimension of the subspace of $ F^{1, m} $ spanned by row of $ A $.

\end{definition}