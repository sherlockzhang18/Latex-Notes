\lesson{16}{Thr Oct 23 2025 14:30}{}

\begin{theorem} \leavevmode
    
    $ T \in L(V) $, $ V: $ finite-dimensional
    \begin{itemize}
        \item 0 is an eigenvalue for $ T \iff T $ is not injective $ \iff T $ is not invertible. 
        \item $ \lambda $ is an eigenvalue for $ T \iff T - \lambda I $ is not injective $ \iff T - \lambda I $ is not invertible.
    \end{itemize}

\end{theorem}

\vspace{1em}
\begin{definition} Eigenvector \leavevmode
    
    $ 0 \neq v \in V $ is called an eigenvector for $ T $ corresponding to $ \lambda $ if
    \[
        Tv = \lambda v
    \]

    Any scalar multiple of eigenvector is also an eigenvector.

\end{definition}

\begin{eg}
    
    Find all eigenvalues of $ T \in L(\mathbb{F}^3) $

    $ T(x, y, z) = (2y, 0, 5z) $

    \vspace{1em}
    
    $ \implies (2y, 0, 5z) = (\lambda x, \lambda y, \lambda z) \implies 2y = \lambda x, 0 = \lambda y, 5z = \lambda z $

    $ 0 = \lambda y \implies \lambda = 0 \text{ or } y = 0 $

    if $ y = 0 $, then $ \lambda x = \lambda y = 0 $, so $ \lambda = 0 $ or $ x = 0 $

    If $ x = y = 0 $, then $ z \neq 0 $ and $ 5z = \lambda z $, so $ \lambda = 5 $

\end{eg}

\vspace{2em}
\begin{theorem} \leavevmode

    Suppose $ T \in L(V) $, and $ \lambda_1, \cdots , \lambda_{k} $ are $ k $ \underbar{distinct} eigenvalues with eigenvectors $ v_1, \cdots , v_{k} $ with
    \[
        Tv_{i} = \lambda_{i} v_{i}
    \]
    then $ v_1, \cdots , v_{k} $ are linearly independent.

\end{theorem}

\vspace{1em}
\begin{corollary} \leavevmode
    
    If $ \dim V = n $, then $ T \in L(V) $ has at most $ n $ distinct eigenvalues.

\end{corollary}

\vspace{1em}
\begin{proof} \leavevmode
    
    \noindent Assume $ v_1, \cdots , v_{k} $ are linearly dependent. By linear dependence lemma, there is $ 1 \leqslant j \leqslant k $ such that $ v_{j} \in \text{span }(v_1, \cdots , v_{j-1}) $.

    \vspace{1em}
    Pick $ j $ the smallest index with thism property such that 
    \begin{align*}
        &v_1, \cdots , v_{j}:\\
        &v_{j} \in \text{span } (v_1, \cdots ,v_{j-1}) \text{ and } v_1, \cdots , v_{j-1} \text{ is linearly independent.}
    \end{align*}

    This implies that $ v_{j} = a_1v_1 + \cdots + a_{j-1}v_{j-1} $

    $ \implies Tv_{j} = a_1Tv_1 + \cdots + a_{j-1}Tv_{j-1} = a_1\lambda_1v_1 + \cdots + a_{j-1}\lambda_{j-1}v_{j-1} $

    \vspace{1em}
    On the other hand, 
    \[
        Tv_{j} = \lambda_{j}v_{j} = \lambda_{j}(a_1v_1 + \cdots + a_{j-1}v_{j-1}) = \lambda_{j}a_1v_1 + \cdots + \lambda_{j} a_{j-1}v_{j-1}
    \]

    \vspace{1em}
    Using the two equations above:
    \[
        a_1\lambda_1v_1 + \cdots + a_{j-1}\lambda_{j-1}v_{j-1} = \lambda_{j}a_1v_1 + \cdots + \lambda_{j} a_{j-1}v_{j-1}
    \]
    $ \implies $
    \[
        a_1(\lambda_1 - \lambda_{j})v_1 + \cdots + a_{j-1}(\lambda_{j-1} - \lambda_{j})v_{j-1} = 0
    \]

    as $ v_1, \cdots , v_{j-1} $ are linearly independent, 
    
    $ a_1(\lambda_1 - \lambda_{j}) = \cdots = a_{j-1}(\lambda_{j-1} - \lambda_{j}) = 0 $

    Since $ \lambda_1, \cdots , \lambda_{j} $ are distinct, we have $ a_1 = \cdots = a_{j-1} = 0 $, which implies $ v_{j} = 0 $. Contradiction.
\end{proof}

\vspace{2em}

\begin{definition} Notation for $ T^m $ \leavevmode
    
    $ V $ vector space over $ F $, $ T \in L(V) $

    \begin{itemize}
        \item $ T^2 = T \circ T, T^m = T \circ T \circ \cdots \circ T $, $ m \geqslant 1 $
        \item $ T^0 = I $
        \item $ T^nT^m = T^{n + m} $
        \item $ (T^n)^m = T^{(mn)} $
    \end{itemize}

\end{definition}

\vspace{1em}
\begin{definition} Notation for $ p(T) $ \leavevmode
    
    if $ p(x) \in P(F) $, $ p(x) = a_{n}x^n + \cdots + a_1x + a_0 $, $ a_0, \cdots , a_{n} \in F $, then

    \[
        p(T) \in L(V), p(T) = a_{n}T^n + \cdots + a_1T + a_0 I
    \]

\end{definition}


\vspace{1em}
If $ p, q \in P(F)$, then $ pq \in P(F) $ as well.

\[
    p(x) = a_{n}x^n + \cdots + a_1x + a_0 \hspace{2em} q(x) = b_{m}x^m + \cdots + b_1x + b_0
\]
\[
    pq(x) = a_{n}b_{m}x^{n+m} + \cdots + (a_1b_0 + a_0b_1)x + a_0b_0
\]

\vspace{1em}
\begin{definition} Product of Polynomial \leavevmode
    
    \begin{itemize}
        \item $ (pq)(T) = p(T)q(T) $
        \item $ p(T) q(T) = q(T)p(T) $
    \end{itemize}

\end{definition}

\vspace{1em}
\begin{explanation}
    
    $ p(x) = x - 1 $, $ q(x) = x + 1 $, 
    then 
    \begin{align*}
        p(T) = T - I \hspace{2em} p(T) v = (T-I) v = Tv - v \\
        q(T) = T + I \hspace{2em} q(T) v = (T + I) v = Tv + v    
    \end{align*}
    \[
        p(T)q(T) = p(T) \circ q(T)
    \]
    \begin{align*}
        (p(T)q(T)) v 
        &= p(T)(q(T) v) 
        = p(T)(Tv + v) \\
        &= T(Tv + v) - (Tv + v) \\
        &= T(Tv) + Tv - Tv - v \\
        &= Tv^2 - v
    \end{align*}
    \begin{align*}
        (q(T)p(T)) v 
        &= q(T)(p(T) v)
        = q(T)(Tv - v) \\
        &= T(Tv - v) + (Tv - v) \\
        &= T(Tv) - Tv + (Tv - v) \\
        &= Tv^2 - v
    \end{align*}
    \[
        (pq) (x) = p(x)q(x) = (x - 1)(x + 1) = x^2 - 1
    \]
    \begin{align*}
        &pq(T) = T^2 - 1 \\
        \implies \hspace{1em} & pq(T)(v) = (T^2 - I) v = Tv^2 - v
    \end{align*}

\end{explanation}

\begin{eg} \leavevmode
    
    $ T \in L(\mathbb{R}^2), T(x, y) = (-y, x) $ (Rotation of 90 degrees)

    What is $ T^2, T^3, T^4, T^5 $?

    \vspace{1em}
    Solution:
    \begin{itemize}
        \item $ T^2 (x, y) = T(-y, x) = (-x, -y) $
        \item $ T^3 (x, y) = T(T^2 (x, y)) = T(-x, -y) = (y, -x) $
        \item $ T^4 (x, y) = T(T^3 (x, y)) = T(y, -x) = (x, y) $
    \end{itemize}

    $ \implies T^4 = I $.

\end{eg}

\begin{eg}
    
    If $ T \in L(V) $. If $ T^2 = I $, then show $ T $ has an eigenvalue.

    \vspace{2em}
    If $ \lambda $ is an eigenvalue, then $ Tv = \lambda v $ for some $ v \neq 0 $. So $ T^2v = T(Tv) = T(\lambda v) = \lambda Tv = \lambda^2v $.

    So $ v  = \lambda^2 v $, so $ (1 - \lambda)^2v = 0, v \neq 0 $. So $ \lambda^2 = 1, $ so $ \lambda = \pm 1 $

    \vspace{1em}
    $ \lambda = 1 $ is not an eigenvalue $ \iff $ $ T - I $ is injective.

    $ \lambda = -1 $ is not an eigenvalue $ \iff $ $ T + I $ is injective.

    \vspace{1em}
    $ T^2 - I = 0 \implies (T-I)(T+I) = 0 $

    \vspace{1em}
    Since $ T^2 - I = 0 $, we have $ (T - I)(T + I) = 0 $. But the composition of two injective linear maps is injective. Let $ S_1 = (T - I) $ and $ S_2 = (T + I) $. If $ S_1S_2 = 0 $at least one of $ S_1 $ or $ S_2 $ is not injective. 

    \begin{itemize}
        \item If $ S_1 = T - I $ is not injective, then 1 is an eigenvalue
        \item If $ S_2 = T + I $ is not injective, then -1 is an eigenvalue
    \end{itemize}


\end{eg}