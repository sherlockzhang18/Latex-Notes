\lesson{1}{Thu Aug 26 2025 14:30}{}
\underbar{\textbf{Syllabus}}: Weekly homework due on Thursday.


\setcounter{chapter}{0}
\chapter{Vector Space}

\section{$ \mathbb{R}^n \text{ and } \mathbb{C}^n $}



Properties of Complex Number:

\begin{itemize}
    \item commutativity
    \item associativity
    \item distributivity
    \item identity
    \item additive inverse
    \item multiplicative inverse
\end{itemize}

In general, if $ 0 \ne \alpha = a+bi $, then 
\begin{align*}
      & (a+bi)(\frac{a}{a^2 + b^2} + \frac{-b}{a^2 + b^2}i) \\
    = & (\frac{a^2}{a^2 + b^2} + \frac{b^2}{a^2 + b^2}) + (\frac{-ab}{a^2 + b^2} + \frac{ab}{a^2 + b^2}i)\\
    = & 1+0i = 1
\end{align*}

\begin{definition} \leavevmode

    A \underbar{\textbf{Field}} ($ \mathbb{F} $) is a set with distinct elements 0,1 and two operations: \underbar{addition} and \underbar{multiplication} with the following properties.
    \begin{enumerate}
        \item \underbar{commutativity}: $ x, y \in \mathbb{F}$, $ x + y = y + x $, $ xy = yx $
        \item \underbar{associativity}: $ x, y, z \in \mathbb{F} $, $ x+(y+z) = x+(y+z) $, $ x(yz) = (xy)z $
        \item \underbar{distribution}: $ x(y+z) = xy + xz $
        \item \underbar{additive identity}: $ 0+x = x \hspace{1em} \forall x $
        \item \underbar{multiplicative identity}: $ 1 \cdot x = x \forall x $
        \item \underbar{additive inverse}: $ \forall x \in \mathbb{F} $, there is an unique element $  y \in \mathbb{F} $ such that $ x+y = 0 $. Notion $  -x = y $
        \item \underbar{multiplicative inverse}: $ 0 \ne x \in \mathbb{F} $, then ther is a unique $  z \in \mathbb{F} $ such that $ xz = 1 $. (Notion: $ z = \frac{1}{x} $)
    \end{enumerate}
\end{definition}

\begin{eg} \leavevmode

    $ \mathbb{R}, \mathbb{C}, \mathbb{Q} $ are fields

    $ \mathbb{Z} $ is not a field: 2 doesn't have a multiplicative inverse
\end{eg}

\begin{eg} \leavevmode
    $ \mathbb{F}_2 = \{0,1\} $ a field with only 2 elements
    \begin{itemize}
        \item $ 0+0=0 $
        \item $ 0+1=1+0=1 $
        \item $ 1+1=0 $
        \item $ 0 \cdot 1  = 1 \cdot 0 = 0$
        \item $ 0 \cdot 0 = 0 $
        \item $ 1 \cdot 1 = 1 $
    \end{itemize}
\end{eg}

\vspace{2em}
\begin{notation}
    If $ F $ is a field, then
    \[
        F^n = \{ (x_{1}, \cdots , x_n), x_i \in F\}
    \]
\end{notation}

\vspace{2em}
\begin{eg}
    $ \mathbb{R}^2, \mathbb{R}^4, \mathbb{C}^5$
\end{eg}

\begin{definition} \leavevmode
    
    \textbf{addition} in $ F^n $:
    $ (x_{1}, \cdots , x_n) + (y_{1}, \cdots y_n) = (x_{1} + y_{1}, \cdots , x_n+y_n) $

    \vspace{1em}
    \textbf{Scalar Multiplication}:
    $ (x_1, \cdots , x_n) \in F^n, \lambda \in F $

    $ \lambda(x_1, \cdots , x_n) = (\lambda x_1, \cdots , \lambda x_n) $
\end{definition}


\vspace{2em}
\vspace{2em}
\underbar{\textbf{from now on}} $ \mathbb{F} = \mathbb{R} $ or $ \mathbb{F = \mathbb{C}} $


\section{Vector Space}
\begin{definition}
    Fix a field $ F $ ($ F = \mathbb{R} $ or $ F = \mathbb{C} $) . A \underbar{\textbf{vector space}} over $ F $ is a set $ V $ along with addition and scalar multiplication such that the following hold
    \begin{itemize}
        \item Communitativity: $ v,w \in V,  v+w = w+v $
        \item associativity: $ v,w,u \in V, (v+u)+ w = v+(u+w) $
        \item distribution: $ a \in F, v,w \in V, a(v+w) = av+aw $
        \item additive identity
        \item additive inverse
        \item multiplication identity
    \end{itemize}
\end{definition}