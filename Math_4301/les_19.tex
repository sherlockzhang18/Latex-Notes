\lesson{19}{Tue Nov 6 2025 14:30}{}

\begin{eg} \leavevmode
    
    $ T \in L(\mathbb{R}^2), T(x, y) = (x - 3y, -y) $.

    \vspace{1em}
    $ p(x) = x^2 - 2x + 2 $

    \vspace{1em}
    Find $ p(T) $

    \begin{align*}
        P(T) 
        = T^2 - 2T + 2I
        = (x, y) - (2x-6y, -2y) + (2x, 2y)
        = (x+6y, 5y)
    \end{align*}

    \vspace{1em}
    Or we can do it with matrices:

    with respect to the standard basis:
    \[
        T = \begin{bmatrix}
          1 & -3 \\
          0 & -1 \\
        \end{bmatrix}
    \]
    then 
    \[
        M(P(T)) = T^2 - 2T + 2I = 
        \begin{bmatrix}
          1 & 6 \\
          0 & 5 \\
        \end{bmatrix}
    \]

\end{eg}

\begin{theorem}[Change of basis] \leavevmode
    
    Let $ T: V \to W $ with $ v_1, \cdots , v_{n} $ and $ v_1', \cdots , v_{n}' $ be the basis of $ V $, and $ w_1, \cdots , w_{n} $ and $ w_1', \cdots , w_{n}' $ be the basis of $ W $. Let $ A = M(T) $ with basis of $ v_1 $ and $ w_1 $, and $ B = M(T) $ with basis of $ w_1 $ and $ w'_1 $. Then 
    \[
        B = M A N
    \]
    where $ N $ is change of basis from $ v_{i} $ to $ v'_i $, and $ M $ is change of basis from $ w'_i $ to $ w_{i} $.

\end{theorem}


\begin{theorem} \leavevmode

    $ T \in L(V) $.
    \begin{itemize}
        \item $ M(T) $ with respect to $ v_1, \cdots , v_{n} $ is upper-triangular $ \iff  \text{span }(v_1, \cdots , v_{i})$ is invariant under $ T $ for every $ i, 1\leqslant i \leqslant n $.
        \item If $ F = \mathbb{C}, \dim V = n $, then there is always a basis such that $ M(T) $ is upper-triangular.
        \item Suppose $ F = \mathbb{R} \text{ or } \mathbb{C} $, and $ M(T) $ is upper-triangular, then $ T $ is invertible $ \iff $ non of the diagonal entries of $ M(T) $ is zero.
    \end{itemize}
    
\end{theorem}

\begin{proof} \leavevmode
    
    to be filled

\end{proof}

\begin{theorem}

    $ T \in L(V, W) $, $ v_1, \cdots , v_{n} $ basis for $ V $, $ w_1, \cdots , w_{m} $ basis for $ W $.
    \[
        \begin{bmatrix}
          b_1 \\
          \vdots \\
          b_{m} \\
        \end{bmatrix} (\text{coordinates of }Tv)
         = 
        M(T)
        =
        \begin{bmatrix}
          a_1 \\
          \vdots \\
          a_{n} \\
        \end{bmatrix}(\text{coordinates of }v)
    \]

    $ v \in V $, $ v = a_1v_1 + \cdots + a_{n}v_{n} $

    $ Tv = a_1Tv_1 + \cdots + a_{n}Tv_{n} $, where $ Tv_1 $ is first column of $ M(T) $.

\end{theorem}

\vspace{1em}
\begin{corollary} \leavevmode
    
    if $ T\in L(V) $ and $ M(T) $ is upper-triangular, then 
    \[
        \{\text{eigenvalues of }T \} = \{ \text{ diagonal entries of } M(T)\}
    \]

\end{corollary}

\begin{proof}
    $ \lambda $ is an eigenvalue $ \iff T - \lambda I $ is not invertible.

    But $ M(T - \lambda I) $ is also upper-triangular. So 0 is on the diagonal of $ M(T - \lambda I) \iff \lambda$ is on the diagonal of $ M(T) $.
\end{proof}

\begin{eg}
    
    $ T \in L(\mathbb{R}^2), T(x, y) = (2 x + 3y, -y) $ 

    \[
        M(T) = \begin{bmatrix}
          2 & 3 \\
          0 & -1 \\
        \end{bmatrix}
    \]
    then eigenvalues are $ 2, -1 $.

\end{eg}

\section{Diagonalizable Operators}

\begin{definition}
    
    $ T \in L(V) $.
    \vspace{1em}
    
    $ T $ is diagonalizable if there is a basis with respect to which $ M(T) $ is a diagonal matrix.
    \[
        M(T) = \begin{bmatrix}
          * & 0 & 0 \\
          0 & \ddots & 0 \\
          0 & 0 & * \\
        \end{bmatrix}
    \]

\end{definition}

\begin{eg}
    
    $ T \in L(\mathbb{R}^2), T(x, y) = (-y, x) $ has no eigenvalue and therefore not diagonalizable.

    \vspace{1em}
    However, if over $ \mathbb{C} $,

\end{eg}

\begin{theorem}
    $ T $ is diagonalizable if and only If
    \begin{itemize}
        \item There is a basis of $ V $ consisting of eigenvectors.
    \end{itemize} 

    i.e.
    If 
    \[
        M(T) = \begin{bmatrix}
          a_1 & 0 & 0 \\
          0 & \ddots & 0 \\
          0 & 0 & a_{n} \\
        \end{bmatrix}
    \]
    then $ Tv_{i} = a_{i}v_{i} (v_{i} \neq 0) \implies v_{i} $ is an eigenvector.
\end{theorem}
