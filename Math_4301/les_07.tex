\lesson{7}{Tue Sep 16 2025 14:30}{}

\begin{theorem} \textbf{Linear Map Lemma} (3.4)\leavevmode
    
    $ V, W $ are two vector spaces over $ F $, assume $ v_1, \cdots , v_{n} $ is a basis for $ V $, and $ w_1, \cdots , w_{n} $ are arbitrary vectors in $ W $. Then, there exists a \underbar{unique} linear map $ T: V \to W $ such that $ T(v_{i}) = w_{i} $

\end{theorem}

\vspace{1em}
\begin{proof} \leavevmode
    
    If $ v $ is a vector in $ V $, we can write $ v $ as 
    \[
        v = a_1v_1 + \cdots  + a_{n}v_{n}
    \] in a unique way. Define
    \[
        T(v) = a_1w_1 + \cdots + a_{n}w_{n}
    \]
    Then this is well-defined. This $ T $ is a linear map.

    Prove for linear map here: $ \cdots  $

    \vspace{2em}
    For uniqueness, note that if $ T': V \to W $ is another such linear map for all $ v \in V $, if 
    \[
        v = a_1v_1 + \cdots + a_{n} v_{n}
    \] then, 
    \begin{align*}
        T(v) &= T(a_1v_1 + \cdots + a_{n}v_{n}) \\
        &=a_1T(v_1) + \cdots + a_{n}T(v_{n}) \\
        & = a_1w_1 + \cdots + a_{n} w_{n} \\
    \end{align*}

    Similarly, we can write
    \begin{align*}
        T'(v) &= T'(a_1v_1 + \cdots + a_{n}v_{n}) \\
        &=a_1T'(v_1) + \cdots + a_{n}T'(v_{n}) \\
        & = a_1w_1 + \cdots + a_{n} w_{n} \\
    \end{align*}

    So $ T(v) = T'(v) $ for all $ v \in V $.

\end{proof}


\vspace{2em}
\begin{theorem} \leavevmode
    
    if $ v_1, \cdots , v_{n} $ is linearly independent in $ V $ and $ w_1, \cdots , w_{n} \in W $. There is linear map 
    \[
        T: V \to W \text{ with } T(v_{i}) = w_{i}, \hspace{1em} i = 1, \cdots , n
    \]

\end{theorem}


\vspace{2em}
\begin{eg}
    
    Find two linear maps in $ L(\mathbb{R}^2, \mathbb{R}^2) $ with
    \[
        T(1,0) = (2,3)
    \]

    \vspace{1em}
    answer:
    \begin{itemize}
        \item $ T(x,y) = (2x, y + 3x) $
        \item $ T(x,y) = (2x, 3x) $
        \item $ T(x+1, y+3) $ is not linear as $ T(0) \neq 0 $
    \end{itemize}

\end{eg}

\section{Null Spaces and Ranges}
\vspace{2em}
\begin{definition} (3.11)
    
    Let $ T \in L(V, W) $. The \textbf{null space} of $ T$ is 
    \[
        \text{null } T: = \{v \in V | T(v) = 0 \}
    \]

\end{definition}


\begin{eg}
    \begin{enumerate}
        \item $ T: P(\mathbb{R}) \to P(\mathbb{R}) $, \hspace{1em} $ T(p(x)) = p'(x) $
        
        Then \[
            \text{null} T = \{p(x) | p'(x) = 0 \} = \text{ set of constant of polynomials}
        \]

        \item $ T: \mathbb{R}^3 \to \mathbb{R}^3 $, \hspace{1em} $ T(x_1, x_2, x_3) = (x_1 + x_2, 2x_3 + x_2, 2x_3-x_1) $.
        
        then $ \text{null} T = \{(x_1,x_2, x_3) | T(x_1, x_2, x_3) = 0\} $.

        \vspace{1em}
        By solving the above three equation, we got $ (x_1, x_2, x_3) = x_1 \cdot (1, -1, \frac{1}{2}) \in \text{span}(1, -1, \frac{1}{2}) $

        Also, $ T(a(1, -1, \frac{1}{2})) = T(a, -a, \frac{a}{2}) = (0,0,0) $. So $ \text{null}T = \text{span}(1, -1, \frac{1}{2}) $.
    \end{enumerate}
\end{eg}

\vspace{2em}
\begin{theorem} 3.13 \leavevmode
    If $ T \in L(V, W) $, then $ \text{null} T $ is a subspace of $ V $.
\end{theorem}

\begin{proof}
    \begin{itemize}
        \item If $ v_1, v_2 \in \text{null} T $, then $ T(v_1) = T(v_2) = 0 $, so 
        \[
            T(v_1 + v_2) = T(v_1) + T(v_2) = 0 + 0 = 0
        \]

        \item if $ v \in \text{null} T $, then $ T(v) = 0 $, so 
        \[
            T(av) = aT(v) = a \cdot 0 = 0
        \] so $ av \in \text{null} T $
    \end{itemize}
\end{proof}

\vspace{2em}
\begin{definition} 3.14 \leavevmode
    $ T \in L(V, W) $ is \underbar{injective} if $ T $ is one-to-one: $ T(v) = T(w) $ implies $ v = w $
\end{definition}


\vspace{2em}
\begin{theorem} 3.15 \leavevmode
    $ T \in L(V, W) $, T is injective $ \iff  \text{null} T = \{0\} $
\end{theorem}

\begin{proof} \leavevmode

    $ \implies $ clear

    \vspace{1em}
    $ \impliedby $: if $ T(v) = T(w) $, then
    \[
        T(v - w) = T(v) - T(w) = 0
    \]
    So, $ v - w \in \text{null} T \implies v - w = 0 $. So $ v = w $.

\end{proof}

\vspace{2em}
\begin{definition} 3.16 \leavevmode
    
    $ T \in L(V, W) $, $ \text{range}T := \{T(v) | v \in V\} $

\end{definition}

\vspace{2em}
\begin{theorem}3.18 \leavevmode

    $ \text{range} T $ is a subspace of $ W $.
    
\end{theorem}

\begin{proof} \leavevmode

    \begin{itemize}
        \item if $ w_1, w_2 \in \text{range} T$, then $ w_1 = T(v_1), w_2 = T(v_2) $,
        so 
        \[
            w_1 + w_2 = T(v_1) + T(v_2) = T(v_1 + v_2) \implies w_1 + w_2 \in \text{range}T
        \]

        \item if $ w \in \text{range}T $, then $ w = T(v) $, so $ aw = aT(v) = T(av) $.
        
        So $ aw \in \text{range}T $
    \end{itemize}

\end{proof}


\vspace{2em}
\begin{eg} \leavevmode

    \begin{itemize}
        \item $ T: \mathbb{R}^3 \to \mathbb{R}^3 $, \hspace{1em} $ T(x_1, x_2, x_3) = (x_1 + x_2, 2x_3 + x_2, 2x_3-x_1) $.
        
        Find a basis for $ \text{range}T $.

        $ T(x_1, x_2, x_3) = x_1(1,0,-1) + x_2(1, 1, 0) + x_3(0,2,2) $

        $ \in \text{span}\{(1, 0, -1), (1, 1, 0), (0,2,2)\}$.

        \vspace{1em}
        as $ (1, 1, 0) = (1, 0, -1) + \frac{1}{2}(0,2,2) $

        $ T(x_1,x_2,x_3) \in \text{span}\{(1,0,-1), (0,2,2)\} = \text{range} T$

        \vspace{1em}
        Therefore, $ \text{range} T $ is 2-dimensional.
    \end{itemize}
\end{eg}

\vspace{2em}
\begin{theorem} \textbf{Fundamental Theorem of Linear Map} (3.21) \leavevmode
    
    If $ T \in L(V, W) $ and $ V $ is finite-dimensional, then
    \[
        \dim \text{null} T + \dim \text{range} T = \dim V
    \]

\end{theorem}