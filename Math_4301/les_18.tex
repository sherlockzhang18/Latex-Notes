\lesson{18}{Thr Oct 30 2025 14:30}{}

\begin{prop}
    
    Let $ T \in L(V) $, $ v_1, \cdots , v_{n} $ be a basis for $ V $, then the following are equivalent:

    \begin{enumerate}
        \item $ M(T, (v_1, \cdots , v_{n})) $ is upper-triangular.
        \item $ Tv_{j} \in \text{span }(v_1, \cdots ,v_{j}) $ for every $ 1 \leqslant j \leqslant n $.
        \item $ \text{span }(v_1, \cdots ,v_{j}) $ is invariant under $ T $ for each $ j = 1 ,\cdots ,n $.
    \end{enumerate}

\end{prop}

\vspace{1em}
\begin{proof} \leavevmode

    \begin{itemize}
        \item 
            $ 1 \iff 2 $:
            Draw the matrix then 
            

        \item $ 2 \iff 3 $:
            
            $ \implies $: 
            
            $ w \in \text{span }(v_1, \cdots , v_{j}) $, $ w = a_1v_1 + \cdots + a_{j}v_{j} $. So
            \[
                Tw = a_1Tv_1 + \cdots + a_{j}Tv_{j} \in \text{span }(v_1, \cdots , v_{j}) \text{ by (2)}.
            \]

            \vspace{2em}
            $ \impliedby $

            $ v_{j} \in \text{span }(v_1, \cdots , v_{j}) $, so $ Tv_{j} \in \text{span }(v_1, \cdots ,v_{j}) $
    \end{itemize}

\end{proof}

\vspace{2em}
\begin{theorem}
    
    If $ V $ is a finite-dimensional complex vector space and $ T \in L(V) $, then there is a basis with respect to which the matrix is upper-triangular.

\end{theorem}

\begin{note}
    If $ M(T, (v_1, \cdots , v_{n})) $ is upper-triangular, then $ Tv_1 = A_{1,1}v_1 $ and $ v_1 \neq 0 $. So $ v_1 $ is an eigenvector and $ A_{1,1} $ is an eigenvalue.
\end{note}

\begin{proof}
    
    Induction on $ \dim V $. Let $ n = \dim V $:

    \vspace{1em}
    Base case: $ n = 1 $: Clear. 

    \vspace{1em}
    Induction Step: $ 1, \cdots , n-1 \implies n $
    Then $ T \in L(V) $ has an eigenvalue $ \lambda $.

    \begin{itemize}
        \item 
            First suppose $ \lambda = 0 $ so $ T $ is not injective, so $ \dim \text{null }T \geqslant 1 $, so $ \dim \text{range }T \leqslant n - 1 $.

            $ \text{range }T $ is invariant under $ T $, so we use $ T $ as operator.

            \vspace{1em}
            Let $ U = \text{range }T $, and $ T' $ be the restriction of $ T $ to $ U $. So $ T' \in L(U) $. By induction hypothesis, there is a basis $ u_1, \cdots , u_{n} $ for $ U $ such that $ M(T', (u_1, \cdots , u_{m})) $ is upper triangular.

            Extend this to a basis $ u_1, \cdots , u_{m}, v_1, \cdots , v_{l} $ for $ V $.

            Claim: $ M(T, (u_1, \cdots , u_{m}, v_1, \cdots ,v_{l})) $ is upper-triangular.
            \begin{itemize}
                \item $ Tu_{i} = T'u_{i} \in \text{span } (u_1, \cdots ,u_{i}) $ is upper-triangular
                \item $ Tv_{j} \in \text{range } T = \text{span }(u_1, \cdots ,u_{m}) $ for $ 1 \leqslant j\leqslant l $.
            \end{itemize}

        \item 
            Now suppose $ \lambda $ is arbitrary. Then look at $ S = T - \lambda I  \in L(V) $. Then $ S $ is not injective, so $ S $ has 0 as eigenvalue. By the previous case, there is a basis $ w_1, \cdots , w_{m} $ for $ V $ such that $ M(S, (w_1, \cdots , w_{m})) $ is upper-triangular.

            Then $ T = S + \lambda I $, so
            \[
                M(T) = M(S) + \lambda M(I) = \text{upper-tri } + I = \text{upper-tri }
            \]
        
    \end{itemize}

\end{proof}


\underbar{\textbf{Question:}} $ T \in L(V) $, $ M(T, (v_1, \cdots , v_{n})) $ upper-triangular with
\[
    M(T,(v_1, \cdots ,v_{n})) = 
    \begin{bmatrix}
      A_{1,1} & A_{1,2} & \cdots  & A_{1,n} \\
      0 & A_{2,2} &  & \vdots \\
      0 & 0 & \ddots &  \\
      0 & 0 & 0 & A_{n,n} \\
    \end{bmatrix}
\]

When is $ T $ invertible (equivalently, when is $ T $ injective?)

\vspace{1em}
\underbar{\textbf{Answer:}} when $ A_{1,1}, A{2,2}, \cdots ,A_{n,n} $ are all non-zero.

\begin{proof} \leavevmode
    
    \begin{enumerate}
        \item 
            Suppose $ A_{i,i} = 0 $. Then $ Tv_{i} = A_{1,i}v_1 + \cdots + A_{i-1, i}v_{i-1} \in \text{span }(v_1, \cdots , v_{i-1}) $.

            \vspace{1em}
            But also,
            \begin{align*}
                &Tv_1 \in \text{span }(v_1) \subseteq \text{span }(v_1, \cdots , v_{i-1}) \\
                &\vdots \\
                &Tv_{i-1} \in \text{span }(v_1, \cdots , v_{i-1}) \subseteq \text{span }(v_1, \cdots , v_{i-1})
            \end{align*}

            $ Tv_1, \cdots , Tv_{i} \in \text{span }(v_1, \cdots , v_{i-1}) $, so $ Tv_1, \cdots , Tv_{i} $ are not linearly independent.

            Suppose $ a_1Tv_1 + \cdots + a_{i}Tv_{i} = 0 $, $ a_1, \cdots ,a_{i} $ not all zero.

            Then $ T(a_1v_1 + \cdots + a_{i}v_{i}) = 0 $, so $ a_1v_1 + \cdots + a_{i}v_{i} \in \text{null }(T) $. $ a_1v_1 + \cdots + a_{i}v_{i} $ is non-zero since $ v_1, \cdots ,v_{i} $ are linearly independent.

            So $ \text{null }T \neq \{0\} $. So $ T $ is not injective.

        \item 
            Suppose $ A_{1,1}, \cdots , A_{n,n} $ are all non-zero. We show $ \text{null }T = \{0\} $.
    \end{enumerate}

\end{proof}