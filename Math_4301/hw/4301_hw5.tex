\documentclass[a4paper]{article}

% \usepackage[margin=1in]{geometry} // reduce margin

% Basic packages
\usepackage[utf8]{inputenc}
\usepackage[T1]{fontenc}
\usepackage{textcomp}
\usepackage{url}
\usepackage{booktabs}
\usepackage{enumitem}
\usepackage[dvipsnames]{xcolor}
\usepackage{xifthen}

% Math packages
\usepackage{amsmath, amsfonts, mathtools, amsthm, amssymb}
\usepackage{mathrsfs}
\usepackage{cancel}
\usepackage{bm}
\usepackage{systeme}
\usepackage{stmaryrd} % for \lightning

% Math shortcuts
\newcommand\N{\ensuremath{\mathbb{N}}}
\newcommand\R{\ensuremath{\mathbb{R}}}
\newcommand\Z{\ensuremath{\mathbb{Z}}}
\renewcommand\O{\ensuremath{\emptyset}}
\newcommand\Q{\ensuremath{\mathbb{Q}}}
\newcommand\C{\ensuremath{\mathbb{C}}}
\newcommand\F{\ensuremath{\mathbb{F}}}
\DeclareMathOperator{\sgn}{sgn}
\DeclareMathOperator{\Ker}{ker}
\DeclareMathOperator{\im}{Im}

% Logic symbols
\let\svlim\lim\def\lim{\svlim\limits}
\let\implies\Rightarrow
\let\impliedby\Leftarrow
\let\iff\Leftrightarrow
\let\epsilon\varepsilon
\newcommand\contra{\scalebox{1.1}{$\lightning$}}

% Useful commands
\definecolor{correct}{HTML}{009900}
\newcommand\correct[2]{\ensuremath{\:}{\color{red}{#1}}\ensuremath{\to }{\color{correct}{#2}}\ensuremath{\:}}
\newcommand\green[1]{{\color{correct}{#1}}}

% Horizontal rule
\newcommand\hr{
    \noindent\rule[0.5ex]{\linewidth}{0.5pt}
}

% Simple theorem environments (without fancy boxes for homework)
\theoremstyle{definition}
\newtheorem{definition}{Definition}
\newtheorem{theorem}{Theorem}
\newtheorem{lemma}{Lemma}
\newtheorem{proposition}{Proposition}
\newtheorem{corollary}{Corollary}
\newtheorem*{remark}{Remark}
\newtheorem*{note}{Note}
\newtheorem*{example}{Example}

% Problem environment
\newcounter{problem}
\newenvironment{problem}[1][]
{
    \stepcounter{problem}
    \section*{Problem \theproblem\ifx\relax#1\relax\else: #1\fi}
}
{}

% Subproblem environment
\newcounter{subproblem}[problem]
\newenvironment{subproblem}[1][]
{
    \stepcounter{subproblem}
    \subsection*{(\alph{subproblem})\ifx\relax#1\relax\else\ #1\fi}
}
{}

% Solution environment
\newenvironment{solution}
{
    \noindent\textbf{Solution:}\\
}
{
    
}

% Headers
\usepackage{fancyhdr}
\pagestyle{fancy}
\fancyhf{}
\fancyhead[L]{Sherlock Zhang}
\fancyhead[C]{4301 linear algebra - Homework 5}
\fancyhead[R]{\today}
\fancyfoot[C]{\thepage}

% Title info
\title{Math 4301 - Homework 5}
\author{Sherlock Zhang}
\date{\today}

\begin{document}

\maketitle

% =============================================================================
% HOMEWORK PROBLEMS START HERE
% =============================================================================

\begin{problem}[3B.19]

    Suppose $ W $ is finite-dimensional and $ T \in L(V, W) $. Prove that $ T $ is injective if and only if there exists $ S \in L(W, V) $ such that $ ST $ is the identity operator on $ V $.

\end{problem}

\vspace{2em}
\begin{solution}

    $ \impliedby $

    Suppose $ ST $ is the identity operator on $ V $, then for all $ v \in \text{null}T $,
    \[
        v = ST(v) = S(0) = 0
    \]
    So $ \text{null}T = \{0\} $. So $ T $ is injective.

    \vspace{2em}
    $ \implies $

    Known $ T $ is injective, let $ v_1, \cdots , v_{n} $ be a basis in $ V $, then let $ Tv_1, Tv_2, \cdots , Tv_{n} $ be linear independent in $ \text{range}T $ since $ T $ is injective. Then extend it into a basis for $ W $ as $ Tv_1, \cdots , Tv_{n}, w_1, \cdots , w_{m} $. So $ S $ can be defined as:
    \[
        S(Tv_{i}) = v_{i} \hspace{2em} S(w_{j}) = 0 \hspace{2em} i = 1, \cdots , n, j = 1, \cdots , m
    \]

    Therefore, such $ ST $ will be 
    \[
        ST(v) = a_1ST(v_1) + \cdots + a_{n} ST(v_{n}) = a_1v_1 + \cdots + a_{n}v_{n} = v
    \]

\end{solution}

\newpage
\begin{problem}[3C.2]
    
    Suppose $ T \in L(V, W) $, where $ V $ and $ W $ are finite-dimensional and nonzero. Prove that $ \dim \text{range}T = 1 $ if and only if there exist a basis of $ V $ and a basis of $ W $ such that with respect to these bases, all entries of $ M(T) $ equal 1.

\end{problem}

\vspace{2em}
\begin{solution}
    
    $ \impliedby $

    If all the entries of $ M(T) $ equals to 1, then it means for the two bases $ v_1, \cdots , v_{n} $ and $ w_1, \cdots w_{m} $ selected, 
    \[
        Tv_1 = Tv_2 = \cdots = Tv_{n} = w_1 + w_2 \cdots + w_{m}
    \]

    Since $ w_{i} $ is a basis in $ W $, their sum cannot be zero. So $ \dim \text{range}T = 1 $.


    \vspace{2em}
    $ \implies $

    Prove by construction: As $ \dim \text{range}T = 1 $, we get
    \[
        T(v) = f(v)w
    \] where $ f(v) $ is a linear function, and $ w \in \text{range}T $.

    Let $ v_1, \cdots , v_{n} $ be the basis for $ V $. Want $ f(v_1) =\cdots = f(v_{n}) = 1 $. 

    Pick any $ v_1 \neq 0 \in V $, and scale $ f $ so that $ f(v_1) = 1 $. Then, pick a basis for $ \text{null}T $. Since $ \dim \text{null} T = \dim V - \dim \text{range}T = n-1 $, there is $ u_2, u_3, \cdots , u_{n} $ that forms a basis for $ \text{null}T $. They are linearly independent and the span does not include $ v_1 $, so adding $ v_1 $ also makes the list linearly independent. Therefore, $ v_1, u_2, \cdots , u_{n} $ forms a basis for $ V $.

    Define the new basis as
    \[
        v_1, v_1 + u_2, v_1 + u_3, \cdots v_1 + u_{n}
    \]

    then, for any vector $ v $ in this basis, $ f(v) = 1 $, which means $ T(v) = w $ for all $ v $ bases. Then, we only need to scale the basis for $ w $ so that $ w = w_1 + w_2 + w_3 \cdots w_{m} $. After scaling, we found the basis for $ M(T) = I $.

\end{solution}


\newpage
\begin{problem}[3C.3]
    
    Supose $ v_1, \cdots , v_{n} $ is a basis of $ V $ and $ w_1, \cdots , w_{m} $ is a basis for $ W $.

\end{problem}

\begin{subproblem}

    Show that if $ S, T \in L(V, W) $, then $ M(S+T) = M(S) + M(T) $.

\end{subproblem}

\vspace{2em}
\begin{solution}
    
    Suppose $ S $ has entries $ A_{i,j} $ and $ T $ has entries $ B_{i,j} $.

    Then, 
    \[
        S(v_{k}) = A_{1,j}w_1 + \cdots + A_{m,j}w_{m} \hspace{2em}
        T(v_{k}) = B_{1,j}w_1 + \cdots + B_{m,j}w_{m}
    \]

    Thus, $ S + T(v_{k}) = S(v_{k}) + T(v_{k}) = A_{1,j}w_1 + \cdots + A_{m,j}w_{m} + B_{1,j}w_1 + \cdots + B_{m,j}w_{m} $, which is $ (A_{1,j} + B_{1,j})w_1 + \cdots + (A_{m,j} + B_{m,j})w_{m} $. That is $ M(S+T) = M(S) + M(T) $.

\end{solution}

\vspace{2em}
\begin{subproblem}
    
    Show that if $ \lambda \in F $ and $ T \in L(V, W) $, then $ M(\lambda T) = \lambda M(T) $.

\end{subproblem}

\vspace{2em}
\begin{solution}
    
    Suppose $ T $ has entries $ B_{i,j} $.

    then, 

    \[
        T(v_{k}) = B_{1,j}w_1 + \cdots + B_{m,j}w_{m}
    \]

    thus, 
    \[
        (\lambda T)(v_{k}) = \lambda (T(v_{k})) = (\lambda B_{1,j})w_1 + \cdots + (\lambda B_{m,j})w_{m}
    \]

    Therefore, $ M(\lambda T) $ has entries $ \lambda B_{i,j} $. It is the same as $ M(\lambda T) = \lambda M(T) $

\end{solution}

\newpage
\begin{problem}[3C.10]

    Give an example of 2-by-2 matrices $ A $ and $ B $ such that $ AB \neq BA $.
    
\end{problem}

\vspace{2em}
\begin{solution}
    
    $ A = 
    \begin{bmatrix}
      1 & 0 \\
      0 & 0 \\
    \end{bmatrix} $ and $ B = 
    \begin{bmatrix}
      1 & 1 \\
      0 & 0 \\
    \end{bmatrix} $

    \vspace{2em}
    \[
        AB = 
        \begin{bmatrix}
          1 & 1 \\
          0 & 0 \\
        \end{bmatrix}
        \hspace{2em}
        BA = \begin{bmatrix}
          1 & 0 \\
          0 & 0 \\
        \end{bmatrix}
    \]

\end{solution}

\newpage
\begin{problem}[3D.1]

    Suppose $ T \in L(V, W) $ is invertible. Show that $ T^{-1} $ is invertible and
    \[
        (T^{-1})^{-1} = T.
    \]
    
\end{problem}

\vspace{2em}
\begin{solution}
    
    Let $ S = T^{-1} $, then by definition of invertibility, $ ST = TS = I $. This shows that $ S $ is invertible and $ S^{-1} = T $, which is exactly what the problem stated.

\end{solution}

\newpage
\begin{problem}[3D.4]

    Suppose $ V $ is finite-dimensional and $ \dim V > 1 $. Prove that the set of noninvertible linear maps from $ V $ to itself is not a subspace of $ L(V) $.
    
\end{problem}

\vspace{2em}
\begin{solution}
    
    Fix the basis of $ V $  to be $ v_1, \cdots , v_{n} $ $ n>1 $.
    \[
        A(v_1) = v_1, A(v_{i}) = 0 \text{ for } i > 1
    \]
    \[
        B(v_1) = 0, A(v_{i}) = v_{i} \text{ for } i > 1
    \]

    Both $ A $ and $ B $ are not invertible as they are not injective as
    for both $ A $ and $ B $, $ A(v_1 + v_2) = A(v_1) $.

    But their sum is $ I $, which is invertible. Therefore, it is not a subspace because it isn't closed under addition.

\end{solution}

\newpage
\begin{problem}[3D.11]
    
    Suppose $ V $ is finite-dimensional and $ S, T \in L(V) $. Prove that
    \[
        ST \text{ is invertible } \iff S \text{ and } T \text{ are invertible. }
    \]

\end{problem}

\vspace{2em}
\begin{solution}
    
    $ \impliedby $

    If $ S $ and $ T $ are invertible, then $ S^{-1}S = I $ and $ T^{-1}T = I $.

    Then
    \[
        T^{-1}S^{-1}ST = T^{-1} I T = T^{-1}T = I
    \]

    \vspace{2em}
    $ \implies $
    If $ S $ is not invertible, then $ S $ is not surjective. Then, 
    \[
        \dim \text{range} ST \leqslant \dim \text{range}S \leqslant \dim V
    \]
    Then this shows that $ ST $ is not surjevtive.

    A similar argument can be made to $ T $ as well. 

    So we proved both $ S, T $ has to be invertible.

\end{solution}

\newpage
\begin{problem}[3F.9]

    Suppose $ m $ is a positive integer. Show that the dual basis of the basis $ 1, x, \cdots , x^m $ of $ P_{m}(\mathbb{R}) $ is $ \psi_0, \psi_1, \cdots \phi_{m} $, where

    \[
        \phi(p) = \frac{p^{(k)}(0)}{k!}.
    \]

\end{problem}




\vspace{2em}
\begin{solution}
    
    Assume $ k \in \{0, \cdots , m\} $ and since for any $ j \neq k $

    \[
        \phi_{k}(x^k) = \frac{(x^k)^{(k)}(0)}{k!} = 1 \text{ and } \phi_{k}(x^j) = \frac{(x^j)^{(k)}(0)}{k!} = \frac{0}{k!} = 0
    \]

    Thus, the linear maps of $ \phi_{k} $ and $ p $ agree on the basis $ 1, x, \cdots , x^m $.

\end{solution}


\end{document}