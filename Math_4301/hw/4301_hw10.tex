\documentclass[a4paper]{article}

% \usepackage[margin=1in]{geometry} // reduce margin

% Basic packages
\usepackage[utf8]{inputenc}
\usepackage[T1]{fontenc}
\usepackage{textcomp}
\usepackage{url}
\usepackage{booktabs}
\usepackage{enumitem}
\usepackage[dvipsnames]{xcolor}
\usepackage{xifthen}

% Math packages
\usepackage{amsmath, amsfonts, mathtools, amsthm, amssymb}
\usepackage{mathrsfs}
\usepackage{cancel}
\usepackage{bm}
\usepackage{systeme}
\usepackage{stmaryrd} % for \lightning

% Math shortcuts
\newcommand\N{\ensuremath{\mathbb{N}}}
\newcommand\R{\ensuremath{\mathbb{R}}}
\newcommand\Z{\ensuremath{\mathbb{Z}}}
\renewcommand\O{\ensuremath{\emptyset}}
\newcommand\Q{\ensuremath{\mathbb{Q}}}
\newcommand\C{\ensuremath{\mathbb{C}}}
\newcommand\F{\ensuremath{\mathbb{F}}}
\DeclareMathOperator{\sgn}{sgn}
\DeclareMathOperator{\Ker}{ker}
\DeclareMathOperator{\im}{Im}

% Logic symbols
\let\svlim\lim\def\lim{\svlim\limits}
\let\implies\Rightarrow
\let\impliedby\Leftarrow
\let\iff\Leftrightarrow
\let\epsilon\varepsilon
\newcommand\contra{\scalebox{1.1}{$\lightning$}}

% Useful commands
\definecolor{correct}{HTML}{009900}
\newcommand\correct[2]{\ensuremath{\:}{\color{red}{#1}}\ensuremath{\to }{\color{correct}{#2}}\ensuremath{\:}}
\newcommand\green[1]{{\color{correct}{#1}}}

% Horizontal rule
\newcommand\hr{
    \noindent\rule[0.5ex]{\linewidth}{0.5pt}
}

% Simple theorem environments (without fancy boxes for homework)
\theoremstyle{definition}
\newtheorem{definition}{Definition}
\newtheorem{theorem}{Theorem}
\newtheorem{lemma}{Lemma}
\newtheorem{proposition}{Proposition}
\newtheorem{corollary}{Corollary}
\newtheorem*{remark}{Remark}
\newtheorem*{note}{Note}
\newtheorem*{example}{Example}

% Problem environment
\newcounter{problem}
\newenvironment{problem}[1][]
{
    \stepcounter{problem}
    \section*{Problem \theproblem\ifx\relax#1\relax\else: #1\fi}
}
{}

% Subproblem environment
\newcounter{subproblem}[problem]
\newenvironment{subproblem}[1][]
{
    \stepcounter{subproblem}
    \subsection*{(\alph{subproblem})\ifx\relax#1\relax\else\ #1\fi}
}
{}

% Solution environment
\newenvironment{solution}
{
    \noindent\textbf{Solution:}\\
}
{
    
}

% Headers
\usepackage{fancyhdr}
\pagestyle{fancy}
\fancyhf{}
\fancyhead[L]{Sherlock Zhang}
\fancyhead[C]{4301 linear algebra - Homework 10}
\fancyhead[R]{\today}
\fancyfoot[C]{\thepage}

% Title info
\title{Math 4301 - Homework 10}
\author{Sherlock Zhang}
\date{\today}

\begin{document}

\maketitle

% =============================================================================
% HOMEWORK PROBLEMS START HERE
% =============================================================================

\begin{problem}[5B.3]

    Suppose $ n $ is an integer with $ n > 1 $ and $ T \in L(F^n) $ is defined by
    \[
        T(x_1, \cdots ,x_{n}) = (x_1 + \cdots + x_{n}, \cdots , x_1 + \cdots + x_{n}).
    \]

\end{problem}

\begin{subproblem}

    Find all eigenvalues and eigenvectors of $ T $.

\end{subproblem}

\vspace{2em}
\begin{solution}

    \[
        M(T) = \begin{bmatrix}
          1 & \cdots & 1 \\
          \vdots & \ddots & \vdots \\
          1 & \cdots  & 1 \\
        \end{bmatrix}
    \]
    Since $ T(-x_2-x_3\cdots -x_{n}, x_2, \cdots x_{n}) \in \text{null }T $, $ \text{null }T \neq \{0\} $. Therefore, 0 is an eigenvector since $ T $ is not injective. The eigenvector is $ -x_2-x_3\cdots -x_{n}, x_2, \cdots x_{n} $.

    \vspace{1em}

    Also, $ T(1, \cdots ,1) = (n, \cdots , n) $, so $ n $ is also an value, with $ (1, \cdots ,1) $ as eigenvector.

    Since $ \text{range }T = \text{span }(1,\cdots ,1) $, we know that if $ \lambda \neq 0 $ is an eigenvalue, and $ x $ is a corresponding eigenvector, then there must be a constant $ c $ such that
    \[
        Tx = \lambda x = c \cdot (1, \cdots , 1)
    \]

    Then, $ x = c \cdot (1, \cdots , 1) \lambda^{-1} $.
    So $ x $ is linearly dependent to $ (1, \cdots ,1) $, which means $ \lambda = n $.

    Therefore, only two eigenvalue above exists.

\end{solution}

\begin{subproblem}
    
    Find  the minimal polynomial of $ T $.

\end{subproblem}

\vspace{2em}
\begin{solution}

    \[
        T(T - nI) 
        =
        \begin{bmatrix}
          1 & \cdots & 1 \\
          \vdots & \ddots & \vdots \\
          1 & \cdots  & 1 \\
        \end{bmatrix}
        \cdot 
        \begin{bmatrix}
          1-n & \cdots & 1 \\
          \vdots & \ddots & \vdots \\
          1 & \cdots  & 1-n \\
        \end{bmatrix}
        =
        0
    \]
    Since there are two distinct eigenvalues, $ \text{deg }p $ is at least 2. Since $ T(T - nI) = 0 $, then $ p(x) = x(x - n) $.

\end{solution}

\newpage
\begin{problem}[5B.6]
    
    Suppose $ T \in L(F^2) $ is defined by $ T(w, z) = (-z, w) $. Find the minimal polynomial of $ T $.

\end{problem}

\vspace{2em}
\begin{solution}
    
    The minimal polynomial cannot be degree 1 since $ T - \lambda I $ cannot be the zero operator since $ T $ is not a multiple of identity operator. Therefore, the degree is at least 2. As $ T^2(w, z) = T(-z, w) = (-w, -z) = -(w, z) $, $ T^2 = -I $. So $ T^2 + I = 0 $, $ p(x) = x^2+1 $.

\end{solution}

\newpage
\begin{problem}[5B.11]

    Suppose $ V $ is a two-dimensional vector space, $ T \in L(V) $, and the matrix of $ T $ with respect to some basis of $ V $ is $ \begin{bmatrix}
      a & c \\
      b & d \\
    \end{bmatrix} $.
    
\end{problem}

\begin{subproblem}
    
    Show that $ T^2 - (a+d)T + (ad-bc)I = 0 $.

\end{subproblem}

\vspace{1em}
\begin{solution}
    
    Let 
    \[
      A = 
      \begin{bmatrix}
        a & c \\
        b & d
      \end{bmatrix}.
    \]
    Then
    \[
      A^2 =
      \begin{bmatrix}
        a & c \\
        b & d
      \end{bmatrix}
      \begin{bmatrix}
        a & c \\
        b & d
      \end{bmatrix}
      =
      \begin{bmatrix}
        a^2 + bc & c(a+d) \\
        b(a+d) & bc + d^2
      \end{bmatrix}.
    \]
    Also,
    \[
      (a+d)A =
      (a+d)
      \begin{bmatrix}
        a & c \\
        b & d
      \end{bmatrix}
      =
      \begin{bmatrix}
        a(a+d) & c(a+d) \\
        b(a+d) & d(a+d)
      \end{bmatrix}.
    \]
    Hence
    \[
      A^2 - (a+d)A =
      \begin{bmatrix}
        a^2 + bc - a(a+d) & 0 \\
        0 & bc + d^2 - d(a+d)
      \end{bmatrix}
      =
      \begin{bmatrix}
        bc - ad & 0 \\
        0 & bc - ad
      \end{bmatrix}.
    \]
    Adding $(ad-bc)I$ gives
    \[
      A^2 - (a+d)A + (ad-bc)I =
      \begin{bmatrix}
        bc - ad & 0 \\
        0 & bc - ad
      \end{bmatrix}
      +
      \begin{bmatrix}
        ad - bc & 0 \\
        0 & ad - bc
      \end{bmatrix}
      =
      \begin{bmatrix}
        0 & 0 \\
        0 & 0
      \end{bmatrix}.
    \]
    Thus $A$ satisfies $A^2 - (a+d)A + (ad-bc)I = 0$. Since $A$ is the matrix
    of $T$ with respect to a basis of $V$, this equation is equivalent to
    \[
      T^2 - (a+d)T + (ad-bc)I = 0.
    \]

\end{solution}

\begin{subproblem}
    
    Show that the minimal polynomial of $ T $ equals
    \[
        \begin{cases}
            z - a & \text{if } b = c = 0 \text{ and } a = d, \\
            z^2 - (a+d)z + (ad - bc) & \text{otherwise}.
        \end{cases}
    \]

\end{subproblem}

\begin{solution}

    If $ b = c = 0 \text{ and } a = d $, then $ T = a I $ where $ c \in F $. Then, $ T - aI = 0 $, which shows that in this case the minimal polynomial should be $ p(z) = z - a $. 
    
    On the other hand, in other cases, $ T \neq cI $ for any $ c \in F $, so there is no 1 degree minimal polynomial exists. $ \text{deg }p \geqslant 2 $. Since $ T^2 - (a+d)T + (ad-bc)I = 0 $, we know that $ T^2 - (a+d)T + (ad-bc)I = 0 $, then $ p(z) = z^2 - (a+d)z + (ad-bc) $ is monic and unique minimal polynomial for other cases.
    
\end{solution}

\newpage
\begin{problem}
    
    Suppose $ V $ is finite-dimensional and $ T \in L(V) $ has minimal polynomial $ 4 + 5z - 6z^2 - 7z^3 + 2z^4 + z^5 $. Find the minimal polynomial of $ T^{-1} $.

\end{problem}

\vspace{2em}
\begin{solution}

    Let the minimal polynomial of $T$ be
    $m(z)=4+5z-6z^2-7z^3+2z^4+z^5$. Then $m(T)=0$, i.e.
    $T^5+2T^4-7T^3-6T^2+5T+4I=0$. The constant term is $4\neq 0$, so $T$ is
    invertible. Multiplying this equation on the left by $T^{-5}$ gives
    \[
    4T^{-5}+5T^{-4}-6T^{-3}-7T^{-2}+2T^{-1}+I=0.
    \]
    Thus $T^{-1}$ is annihilated by the polynomial
    $n(z)=4z^5+5z^4-6z^3-7z^2+2z+1$, which we can also write as
    $n(z)=z^5 m(1/z)$.

    By the result from class that the minimal polynomial of $T^{-1}$ is
    $z^{\deg m_T} m_T(1/z)$ (up to a nonzero scalar), $n$ is, up to scalar,
    the minimal polynomial of $T^{-1}$. Making it monic, we obtain
    \[
    m_{T^{-1}}(z)
    = z^5+\tfrac54 z^4-\tfrac32 z^3-\tfrac74 z^2+\tfrac12 z+\tfrac14
    \]

\end{solution}

\newpage
\begin{problem}[5B.20]
    
    Suppose $ T \in L(F^4) $ is such that the eigenvalues of $ T $ are $ 3, 5, 8 $. Prove that $ (T-3I)^2(T-5I)^2(T - 8I)^2 = 0 $.

\end{problem}

\vspace{2em}
\begin{solution}
    
    By theorem in class, we know that minimal polynomial of $ T $
    \[
        p(x) = s(x)(x -3)(x - 5)(x - 8)
    \]
    Since $ T $ only has three eigenvalues, and $ \text{deg }p \leqslant \dim V = 4 $, $ s(x) = \{1, (x - 3), (x - 5), (x - 8)\} $. So the polynomial of $ (T-3I)^2(T-5I)^2(T - 8I)^2 $ which is $ q(x) = (x - 3)^2(x-5)^2(x- 8)^2 $ is a multiple of the minimal polynomial $ p(x) $, so $ q(x )= 0 $.

\end{solution}

\newpage

\begin{problem}[5D.2]
    
    Suppose $ T \in L(V) $ has diagonal matrix $ A $ with respect ot some basis of $ V $. Prove that if $ \lambda \in F $, then $ \lambda $ appears on the diagonal of $ A $ precisely $ \dim E(\lambda, T) $ times.

\end{problem}

\vspace{2em}
\begin{solution}

    Let $\{v_1,\dots,v_n\}$ be a basis of $V$ with respect to which the
    matrix of $T$ is diagonal. Then
    \[
        A = \operatorname{diag}(a_1,\dots,a_n)
    \]
    for some $a_1,\dots,a_n\in F$, and we have $T(v_j)=a_j v_j$ for each $j$.

    Fix $\lambda\in F$ and let
    \[
        J = \{ j : a_j = \lambda \}.
    \]
    For each $j\in J$ we have $T(v_j)=\lambda v_j$, so $v_j\in E(\lambda,T)$.
    Hence $\operatorname{span}\{v_j : j\in J\} \subseteq E(\lambda,T)$.

    \vspace{1em}
    Conversely, let $v\in E(\lambda,T)$. Write
    $v = x_1 v_1 + \cdots + x_n v_n$ for some scalars $x_1,\dots,x_n$.
    The equation $T(v)=\lambda v$ becomes
    \[
        T(v) = a_1 x_1 v_1 + \cdots + a_n x_n v_n
        = \lambda (x_1 v_1 + \cdots + x_n v_n),
    \]
    so for each $j$ we have $(a_j - \lambda)x_j = 0$. Thus, if $a_j\neq\lambda$
    then $x_j=0$. Therefore all nonzero coefficients $x_j$ occur only for
    indices $j\in J$, and $v$ is a linear combination of $\{v_j : j\in J\}$.
    This shows $E(\lambda,T)\subseteq \operatorname{span}\{v_j : j\in J\}$.

    Hence
    \[
        E(\lambda,T) = \operatorname{span}\{v_j : j\in J\},
    \]
    so $\dim E(\lambda,T)$ equals the number of indices $j$ with $a_j=\lambda$,
    which is exactly the number of times $\lambda$ appears on the diagonal of $A$.

\end{solution}

\newpage
\begin{problem}[5D.6]

    Suppose $ T \in L(F^5) $ and $ \dim E(8,T) = 4 $. Prove that $ T - 2I $ or $ T - 6I $ is invertible.

\end{problem}

\vspace{2em}
\begin{solution}

    Prove contrapositive. 
    
    Assume that neither $T-2I$ nor $T-6I$ is invertible. Then $2$ and $6$ are eigenvalues of $T$, so $\dim E(2,T) \ge 1$ and $\dim E(6,T) \ge 1$.

    Whether $8$ is an eigenvalue or not, theorem in class implies that the sum $E(8,T) + E(2,T) + E(6,T)$ is a direct sum of eigenspaces. Hence
    \[
        \dim E(8,T) + \dim E(2,T) + \dim E(6,T) \le \dim F^5 = 5.
    \]
    Because $\dim E(2,T) \ge 1$ and $\dim E(6,T) \ge 1$, we obtain
    \[
        \dim E(8,T) \le 5 - \dim E(2,T) - \dim E(6,T)
        \le 5 - 1 - 1 = 3.
    \]

    But this contradicts the assumption that $\dim E(8,T) = 4$.

    Therefore our assumption that both $T-2I$ and $T-6I$ are noninvertible must be false. Thus at least one of them is invertible.

\end{solution}



\end{document}
