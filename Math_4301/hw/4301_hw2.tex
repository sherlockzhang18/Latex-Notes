\documentclass[a4paper]{article}

% \usepackage[margin=1in]{geometry} // reduce margin

% Basic packages
\usepackage[utf8]{inputenc}
\usepackage[T1]{fontenc}
\usepackage{textcomp}
\usepackage{url}
\usepackage{booktabs}
\usepackage{enumitem}
\usepackage[dvipsnames]{xcolor}
\usepackage{xifthen}

% Math packages
\usepackage{amsmath, amsfonts, mathtools, amsthm, amssymb}
\usepackage{mathrsfs}
\usepackage{cancel}
\usepackage{bm}
\usepackage{systeme}
\usepackage{stmaryrd} % for \lightning

% Math shortcuts
\newcommand\N{\ensuremath{\mathbb{N}}}
\newcommand\R{\ensuremath{\mathbb{R}}}
\newcommand\Z{\ensuremath{\mathbb{Z}}}
\renewcommand\O{\ensuremath{\emptyset}}
\newcommand\Q{\ensuremath{\mathbb{Q}}}
\newcommand\C{\ensuremath{\mathbb{C}}}
\newcommand\F{\ensuremath{\mathbb{F}}}
\DeclareMathOperator{\sgn}{sgn}
\DeclareMathOperator{\Ker}{ker}
\DeclareMathOperator{\im}{Im}

% Logic symbols
\let\svlim\lim\def\lim{\svlim\limits}
\let\implies\Rightarrow
\let\impliedby\Leftarrow
\let\iff\Leftrightarrow
\let\epsilon\varepsilon
\newcommand\contra{\scalebox{1.1}{$\lightning$}}

% Useful commands
\definecolor{correct}{HTML}{009900}
\newcommand\correct[2]{\ensuremath{\:}{\color{red}{#1}}\ensuremath{\to }{\color{correct}{#2}}\ensuremath{\:}}
\newcommand\green[1]{{\color{correct}{#1}}}

% Horizontal rule
\newcommand\hr{
    \noindent\rule[0.5ex]{\linewidth}{0.5pt}
}

% Simple theorem environments (without fancy boxes for homework)
\theoremstyle{definition}
\newtheorem{definition}{Definition}
\newtheorem{theorem}{Theorem}
\newtheorem{lemma}{Lemma}
\newtheorem{proposition}{Proposition}
\newtheorem{corollary}{Corollary}
\newtheorem*{remark}{Remark}
\newtheorem*{note}{Note}
\newtheorem*{example}{Example}

% Problem environment
\newcounter{problem}
\newenvironment{problem}[1][]
{
    \stepcounter{problem}
    \section*{Problem \theproblem\ifx\relax#1\relax\else: #1\fi}
}
{}

% Subproblem environment
\newcounter{subproblem}[problem]
\newenvironment{subproblem}[1][]
{
    \stepcounter{subproblem}
    \subsection*{(\alph{subproblem})\ifx\relax#1\relax\else\ #1\fi}
}
{}

% Solution environment
\newenvironment{solution}
{
    \noindent\textbf{Solution:}\\
}
{
    
}

% Headers
\usepackage{fancyhdr}
\pagestyle{fancy}
\fancyhf{}
\fancyhead[L]{Sherlock Zhang}
\fancyhead[C]{4301 linear algebra - Homework 2}
\fancyhead[R]{\today}
\fancyfoot[C]{\thepage}

% Title info
\title{Math 4301 - Homework 2}
\author{Sherlock Zhang}
\date{\today}

\begin{document}

\maketitle

% =============================================================================
% HOMEWORK PROBLEMS START HERE
% =============================================================================

\begin{problem}[1C.15]
    Suppose $ U $ is a subspace of $ V $. What is $ U+U $?
\end{problem}

\vspace{2em}
\begin{solution}
    
    $ U = U+U $

    \begin{enumerate}
        \item $ U \subseteq U+U $
        
        $ \forall u \in U $, $ u = u + 0 $, and $ u + 0 $ is in $ U+U $. so $ U \subseteq U+U $

        \vspace{2em}
        \item $ U+U \subseteq U $
        
        as $ U $ is a subspace of $ V $, it is closed under addition. Therefore, $ \forall u_1 + u_2 \in U+U $, both $ u_1 $ and $ u_2 $ are in $ U $. By $ U $ closed under addition, $ u_1 + u_2 \in U $. Thus $ U \subseteq U + U $.
    \end{enumerate}

    Therefore $ U = U+U $.

\end{solution}

\newpage
\begin{problem}[1C.23]
    
    Prove or give a counterexample: if $ V_1, V_2, U $ are subspaces of $ V $ such that 
    \[
        V = V_1 \bigoplus U \text{ and } V = V_2 \bigoplus U
    \]
    then $ V_1 = V_2 $

\end{problem}

\vspace{2em}
\begin{solution}

    The counterexample is $ V_1 = \{(x_1, x_2) | x_1 = 0\} $, $ V_2 = \{(x_1, x_2) | x_2 = 0\} $, $ U = \{(x_1, x_2) | x_1 + x_2 = 0\} $, then $ V = \mathbb{R}^2 $.

    It's trivial to show that $ V_1, V_2, U $ are subspaces of $ V $ and 
    \[
        V = V_1 \bigoplus U \text{ and } V = V_2 \bigoplus U
    \]

\end{solution}

\newpage
\begin{problem}[2A.5]
    
    Find a number $ t $ such that
    \[
        (3,1,4),(2,-3,5),(5,9,t)
    \]
    is not linearly independent in $ \mathbb{R}^3 $

\end{problem}

\vspace{2em}
\begin{solution}
    
    let $ v_1 = (3,1,4), v_2 = (2,-3,5), v_3 = (5,9,t) $
    If $ v_1, v_2, v_3 $ are linearly dependent, then there exist $ a_1, a_2, a_3 \in F $ such that $ a_1v_1 + a_2v_2 + a_3v_3 = 0 $

    First need to find a combination of $ a_1, a_2, a_3 $ such that the first two element can be 0.

    \begin{align*}
        3a_1 + 2a_2 + 5a_3 = 0\\
        a_1 - 3a_2 + 9a_3 = 0\\
    \end{align*}

    one solution will be $ a_1 = -3, a_2 = 2, a_3 = 1 $

    Then, also have $ 4a_1 + 5a_2 + ta_3 = 0 $ \hspace{2em} $ t = 2 $.

\end{solution}

\newpage
\begin{problem}[2A.7]\end{problem}

\begin{subproblem}
    
    Show that if we think of $ \mathbb{C} $ as a vector space over $ \mathbb{R} $, then the list $ 1+i, 1-i $ is linearly independent.

\end{subproblem}

\vspace{2em}
\begin{solution}
    
    as $ \mathbb{C} $ is a vector space over $ \mathbb{R} $, the solution to
    \[
        a(1+i) + b(1-i)
    \]
    can only be $ a = b = 0 $ as $ a, b \in \mathbb{R} $. Therefore, the list is linearly independent in $ \mathbb{R} $.

\end{solution}

\vspace{5em}
\begin{subproblem}
    Shwo that if we think of $ \mathbb{C} $ as a vector space over $ \mathbb{C} $, then the list $ 1+i, 1-i $ is linearly dependent.
\end{subproblem}

\vspace{2em}
\begin{solution}
    
    as $ \mathbb{C} $ is a vector space over $ \mathbb{C} $, there exists non-trivial solution
    \[
        a(1+i) + b(1-i)
    \]
    if $ a = (-1+i) b = (1+i) $. Therefore, the list $ 1+i, 1-i $ is linearly dependent. 

\end{solution}

\newpage
\begin{problem}[2A.10]
    
    Prove or give a counterexample: If $ v_1, \cdots , v_m $ is a linearly independent list of vectors in $ V $ and $ \lambda \in F $ with $ \lambda \neq 0 $, then $ \lambda v_1, \lambda v_2, \cdots , \lambda v_m $ is linearly independent.

\end{problem}


\vspace{2em}
\begin{solution}
    
    Need to solve $ \lambda a_1v_1 + \lambda a_2v_2 + \cdots + \lambda a_mv_m = 0 $, where $ a_1, \cdots , a_m \in F $. 

    Given $ \lambda \neq 0 $, multiply both side by $ \lambda^{-1} $, we get
    \[
        a_1v_1 + a_2v_2 + \cdots + a_mv_m = 0
    \]

    Given $ v_1, \cdots , v_m $ is a linearly independent list, solution to this equation is trivial, which means $ a_1 = a_2 = \cdots = a_m = 0 $. This implies $ \lambda \in F $ with $ \lambda \neq 0 $, then $ \lambda v_1, \lambda v_2, \cdots , \lambda v_m $ is linearly independent.

\end{solution}

\newpage
\begin{problem}[2A.11]
    
    Prove or give a counterexample: If $ v_1, \cdots , v_m $ and $ w_1, \cdots , w_m $ are linearly independent lists of vectors in $ V $, then the list $ v_1+w_1, \cdots , v_m+w_m $ is linearly independent.

\end{problem}

\vspace{2em}
\begin{solution}
    
    counterexample: Assume $ v_1 = (1,0), v_2 = (0,1) $, $ w_1 = (-1,0) w_2 = (0, -1) $. It is trivial that $ \{v_1, v_2\} $ and $ \{w_1,w_2\} $ are linearly independent in $ \mathbb{R}^2 $. However, $ v_1 + w_1 = v_2 + w_2 = (0,0) $, which is not linearly independent.

\end{solution}

\newpage
\begin{problem}[2A.13]
    
    Suppose $ v_1, \cdots , v_m $is linearly independent in $ V $ and $ w \in V $. Show that
    \[
        v_1, \cdots , v_m, w \text{ is linearly independent} \iff w \notin \text{span}(v_1, \cdots , v_m)
    \]

\end{problem}

\vspace{2em}
\begin{solution}
    
    $ \impliedby $ 
    
    Prove contrapositive. Assume $ v_1, \cdots , v_m, w $ is linearly dependent. By linear dependence lemma, one of the vectors must be in the span of previous vectors. But $ v_1, \cdots ,v_m $ are linearly independent, so the only case will be $ w \in \text{span}(v_1, \cdots , v_m) $.

    \vspace{2em}
    $ \implies $

    Prove contrapositive. Assume $ w \in \text{span}(v_1, \cdots , v_m) $. Then 
    \[
        w = a_1v_1 + a_2v_2 + \cdots + a_mv_m, \text{ where } a_i \in F
    \]
    Then, using the same coefficients, $ a_1v_1 + a_2v_2 + \cdots + a_mv_m - w = 0 $, which implies $ v_1, \cdots , v_m, w \text{ is linearly dependent} $.

    \vspace{2em}
    Therefore, $ v_1, \cdots , v_m, w \text{ is linearly independent} \iff w \notin \text{span}(v_1, \cdots , v_m) $.
    
\end{solution}

\newpage
\begin{problem}[2A.20]
    
    Suppose $ p_0, p_1, \cdots , p_m $ are polynomials in $ P_m(F) $ such that $ p_k(2) = 0 $ for each $ k \in (0, \cdots , m) $. Prove that $ p_0, p_1, \cdots , p_m $ is not linearly independent in $ P_m(F) $.

\end{problem}

\vspace{2em}
\begin{solution}
    
    as $ p_k(2) = 0 $ for each $ k \in (0, \cdots , m) $, for $ p_0(2) = 0 $, $ p_0 = 0 $. Therefore, the equation
    \[
        a_0p_0 + a_1p_1 + \cdots + a_mp_m = 0
    \]
    has non-trivial solution. Let $ a_0 $ be any value in $ F $ and $ a_1 = a_2 = \cdots = a_m = 0 $. Then
    \[
        a_0p_0 + a_1p_1 + \cdots + a_mp_m = a_0 \cdot 0 = 0
    \]

    \vspace{1em}
    Thereofore, $ p_0, p_1, \cdots , p_m $ is not linearly independent in $ P_m(F) $.

\end{solution}

\end{document}
