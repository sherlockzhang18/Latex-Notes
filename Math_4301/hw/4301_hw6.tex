\documentclass[a4paper]{article}

% \usepackage[margin=1in]{geometry} // reduce margin

% Basic packages
\usepackage[utf8]{inputenc}
\usepackage[T1]{fontenc}
\usepackage{textcomp}
\usepackage{url}
\usepackage{booktabs}
\usepackage{enumitem}
\usepackage[dvipsnames]{xcolor}
\usepackage{xifthen}

% Math packages
\usepackage{amsmath, amsfonts, mathtools, amsthm, amssymb}
\usepackage{mathrsfs}
\usepackage{cancel}
\usepackage{bm}
\usepackage{systeme}
\usepackage{stmaryrd} % for \lightning

% Math shortcuts
\newcommand\N{\ensuremath{\mathbb{N}}}
\newcommand\R{\ensuremath{\mathbb{R}}}
\newcommand\Z{\ensuremath{\mathbb{Z}}}
\renewcommand\O{\ensuremath{\emptyset}}
\newcommand\Q{\ensuremath{\mathbb{Q}}}
\newcommand\C{\ensuremath{\mathbb{C}}}
\newcommand\F{\ensuremath{\mathbb{F}}}
\DeclareMathOperator{\sgn}{sgn}
\DeclareMathOperator{\Ker}{ker}
\DeclareMathOperator{\im}{Im}

% Logic symbols
\let\svlim\lim\def\lim{\svlim\limits}
\let\implies\Rightarrow
\let\impliedby\Leftarrow
\let\iff\Leftrightarrow
\let\epsilon\varepsilon
\newcommand\contra{\scalebox{1.1}{$\lightning$}}

% Useful commands
\definecolor{correct}{HTML}{009900}
\newcommand\correct[2]{\ensuremath{\:}{\color{red}{#1}}\ensuremath{\to }{\color{correct}{#2}}\ensuremath{\:}}
\newcommand\green[1]{{\color{correct}{#1}}}

% Horizontal rule
\newcommand\hr{
    \noindent\rule[0.5ex]{\linewidth}{0.5pt}
}

% Simple theorem environments (without fancy boxes for homework)
\theoremstyle{definition}
\newtheorem{definition}{Definition}
\newtheorem{theorem}{Theorem}
\newtheorem{lemma}{Lemma}
\newtheorem{proposition}{Proposition}
\newtheorem{corollary}{Corollary}
\newtheorem*{remark}{Remark}
\newtheorem*{note}{Note}
\newtheorem*{example}{Example}

% Problem environment
\newcounter{problem}
\newenvironment{problem}[1][]
{
    \stepcounter{problem}
    \section*{Problem \theproblem\ifx\relax#1\relax\else: #1\fi}
}
{}

% Subproblem environment
\newcounter{subproblem}[problem]
\newenvironment{subproblem}[1][]
{
    \stepcounter{subproblem}
    \subsection*{(\alph{subproblem})\ifx\relax#1\relax\else\ #1\fi}
}
{}

% Solution environment
\newenvironment{solution}
{
    \noindent\textbf{Solution:}\\
}
{
    
}

% Headers
\usepackage{fancyhdr}
\pagestyle{fancy}
\fancyhf{}
\fancyhead[L]{Sherlock Zhang}
\fancyhead[C]{4301 linear algebra - Homework 6}
\fancyhead[R]{\today}
\fancyfoot[C]{\thepage}

% Title info
\title{Math 4301 - Homework 6}
\author{Sherlock Zhang}
\date{\today}

\begin{document}

\maketitle

% =============================================================================
% HOMEWORK PROBLEMS START HERE
% =============================================================================

\begin{problem}[3F.1]

    Explain why each linear functional is surjective or is the zero map.

\end{problem}

\vspace{2em}
\begin{solution}

    Suppose $ \phi \in V' $, then if $ \phi $ is not the zero map, then there exist some $ v \in V $ such that $ \phi(v) = a  $, where $ a \neq 0 $.

    Therefore, since $ \phi $ is linear, for any $ b \in F $, there exists
    \[
        \phi(\frac{v}{a} \times b) = \frac{\phi(v)}{a} \times b = \frac{a}{a} \times b = b
    \]

    Thereofore, $ \phi $ is either zero map or surjective.
    
\end{solution}

\newpage
\begin{problem}[3F.3]

    Suppose $ V $ is finite-dimensional and $ v \in V $ with $ v \neq 0 $. Prove that there exists $ \phi \in V' $  such that $ \phi(v) = 1 $.
    
\end{problem}

\vspace{2em}
\begin{solution}
    
    let $ v_1 = v $ and extend into a basis for $ V $: 
    \[
        v_1, \cdots ,  v_{n}
    \]

    Then define $ \phi \in V' $ as
    \[
        \phi(v_{k}) = 1 \hspace{2em} \text{for }1 \leqslant k \leqslant n
    \]

    Then $ \phi(v) = 1 $.

\end{solution}

\newpage
\begin{problem}[3F.11]
    
    Suppose $ v_1, \cdots , v_{n} $ is a basis of $ V $ and $ \phi_1, \cdots ,\phi_{n} $ is the corresponding dual basis of $ V' $. Suppose $ \psi \in V' $. Prove that
    \[
        \psi = \psi(v_1)\phi_1 + \cdots + \psi(v_{n})\phi_{n}
    \]

\end{problem}

\vspace{2em}
\begin{solution}
    
    \begin{align*}
        &\psi(v_1)\phi_1 + \cdots + \psi(v_{n})\phi_{n} (v) \\
        &= \psi(v_1)\phi_1 + \cdots + \psi(v_{n})\phi_{n} (a_1v_1 + \cdots + a_{n}v_{n}) \\
        &= \psi(v_1)\phi_1 (a_1v_1 + \cdots + a_{n}v_{n}) + \cdots + \psi(v_{n})\phi_{n} (a_1v_1 + \cdots + a_{n}v_{n}) \\
        &=\psi (v_1) a_1 + \cdots + \psi (v_{n}) a_{n} \\
        &= \psi(a_1v_1 + a_2v_2 + \cdots + a_{n}v_{n})
    \end{align*}

\end{solution}

\newpage
\begin{problem}[3F.14]
    
    Define $ T: \mathbb{R}^3 \to \mathbb{R}^2 $ by
    \[
        T(x, y, z) = (4x + 5y + 6z, 7x + 8y + 9z)
    \]
    Suppose $ \phi_1, \phi_2 $ denotes the dual basis of the standard basis of $ \mathbb{R}^2 $ and $ \psi_1, \psi_2, \psi_3 $ denotes the dual basis of the standard basis of $ \mathbb{R}^3 $.

\end{problem}

\begin{subproblem}

    Describe the linear functionals $ T'(\phi_1) $ and $ T'(\phi_2) $

\end{subproblem}

\vspace{1em}
\begin{solution}
    
    \begin{align*}
        T'(\phi_1)(x, y, z) = \phi_1 T(T(x, y, z)) = \phi_1(4x + 5y + 6z, 7x + 8y + 9z) = 4x + 5y + 6z \\
        T'(\phi_2)(x, y, z) = \phi_2 T(T(x, y, z)) = \phi_2(4x + 5y + 6z, 7x + 8y + 9z) = 7x + 8y + 9z
    \end{align*}

\end{solution}

\vspace{1em}
\begin{subproblem}
    
    Write $ T'(\phi_1) $ and $ T'(\phi_2) $ as linear combination of $ \psi_1, \psi_2, \psi_3 $.

\end{subproblem}

\vspace{1em}
\begin{solution}
    
    \[
        T'(\phi_1) = 4\psi_1 + 5\psi_2 + 6\psi_3 \hspace{2em} T'(\phi_2) = 7\psi_1 + 8\psi_2 + 9\psi_3
    \]

\end{solution}

\newpage
\begin{problem}[3F.15]

    Define $ T $ : $ P(\mathbb{R}) \to P(\mathbb{R}) $ by
    \[
        (Tp)(x) = x^2p(x) + p''(x)
    \]
    for each $ x \in \mathbb{R} $.

\end{problem}

\begin{subproblem}
    
    Suppose $ \phi \in P(\mathbb{R})' $ is defined by $ \phi(p) = p'(4) $. Describe the linear function $ T'(\phi) $ on $ P(\mathbb{R}) $.

\end{subproblem}

\vspace{1em}
\begin{solution}
    
    \[
        T'(\phi) (p) = \phi(Tp) = \phi(x^2p + p'') = 8p(4) + 16p'(4) + p'''(4)
    \]

\end{solution}

\vspace{2em}
\begin{subproblem}
    
    Suppose $ \phi \in P(\mathbb{R})' $ is defined by $ \phi(p) = \int_{0}^{1} p $. Evaluate $ (T'(\phi))(x^3) $

\end{subproblem}

\vspace{1em}
\begin{solution}
    
    \[
        (T'(\phi))(x^3) = \phi(Tx^3) = \phi(x^5 + 6x) = \int_{0}^{1} x^5 + 6x dx = \frac{19}{6}
    \]

\end{solution}

\newpage
\begin{problem}[3F.16]
    
    Suppose $ W $ is finite-dimensional and $ T \in L(V, W) $. Prove that
    \[
        T' = 0 \iff T = 0
    \]

\end{problem}

\vspace{2em}
\begin{solution}
    
    $ \impliedby $
    
    \vspace{1em}
    If $ T = 0 $ and $ \phi \in W' $, then $ T'(\phi) = \phi \circ T = \phi \circ 0 = 0 $. 

    \vspace{2em}
    $ \implies $

    \vspace{1em}
    If $ T' = 0 $, then $ \text{null }T' = W' $. Since $ \text{null }T' = (\text{range }T)^0 $, $ (\text{range T})^0 = W' $ as well.

    This means for all $ f \in W' $ $ f(v) = 0 $ for all $ v \in \text{range }T $. This can only happen when $ \text{range }T = \{0\} $. So $ T = 0 $.

\end{solution}

\newpage
\begin{problem}[3F.21]
    
    Suppose $ V $ is finite-dimensional and $ U $ and $ W $ are subspaces of $ V $.

\end{problem}

\begin{subproblem}
    
    Prove that $ W^0 \subseteq U^0 $ if and only if $ U \subseteq W $

\end{subproblem}

\vspace{1em}
\begin{solution}
    
    $ \impliedby $

    \vspace{1em}
    If $ U \subseteq W $, then for $ f \in W^0 $, for all $ u \in U \subseteq W $, $ f(u) = 0 $. So $ W^0 \subseteq U^0 $.

    \vspace{2em}
    $ \implies $

    \vspace{1em}
    Define $ (U^0)^0 as \{v \in V | \phi(v) = 0 \text{ for all } \phi \in U^0\} $. Then since $ W^0 \subseteq U^0 $, if $ v \in (U^0)^0 $, then it is also in $ (W^0)^0 $. So $ (U^0)^0 \subseteq (W^0)^0 $. By definition, it is equivalent to $ U \subseteq W $.

\end{solution}

\begin{subproblem}
    
    Prove that $ W^0 = U^0 $ if and only if $ U = W $

\end{subproblem}

\vspace{2em}
\begin{solution}
    
    Since $ W^0 = U^0 $, $ W^0 \subseteq U^0 $ and $ U^0 \subseteq W^0 $. Therefore, we geometry
    \[
        U \subseteq W \text{ and } W \subseteq U \implies U = W
    \]

\end{solution}

\end{document}