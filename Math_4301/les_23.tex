\lesson{23}{Tue Nov 25 2025 14:30}{}

\begin{lemma} \leavevmode
    
    \begin{enumerate}
        \item 
            $ \text{null }(T - \lambda I) \subseteq \text{null }(T - \lambda I)^2 \subseteq \text{null }(T - \lambda I)^3 \subseteq \cdots \subseteq \text{null }(T - \lambda I)^n \subseteq \cdots  $, where $ n = \dim V $.
        
        \item if $ \text{null }(T - \lambda I)^j = \text{null }(T - \lambda I)^{j+1} $, then
        \[
            \text{null }(T - \lambda I)^j = \text{null }(T - \lambda I)^{j+1} = \text{null }(T - \lambda I)^{j+2} = \cdots 
        \]
    \end{enumerate}

\end{lemma}

\begin{corollary}
    
    \[
        G(\lambda, T) = \text{null }(T - \lambda I)^n \hspace{1em} n = \dim V
    \]

\end{corollary}

\begin{eg}
    
    $ T(x, y, z) = (2z, 3y, 0) $, $ T \in L(\mathbb{C}^3) $.
    \[
        M(T) = \begin{bmatrix}
          0 & 0 & 2 \\
          0 & 3 & 0 \\
          0 & 0 & 0 \\
        \end{bmatrix}
    \]
    with eigenvalues: 0, 3.

    $ T^2(x, y, z) = (0, 9y, 0) $

    $ T^3(x, y, z) = (0, 27y, 0) $

    \vspace{1em}
    \begin{itemize}
        \item 
            $ \lambda = 0 $
            $ E(0,T) = \text{null }T = \text{span }\{(1, 0, 0)\} $, which has dimension 1.

            $ \text{null }T^2 = \text{span }\{(1, 0, 0), (0, 0, 1)\} $, which has dimension 2.

            $ \text{null }T^3 = \text{span }\{(1, 0, 0), (0, 0, 1)\} $, which has dimension 2.

            $ \implies \text{null }T^2 = \text{null }T^3 = \text{null }T^4 = \cdots  $

            $ \implies G(0,T) = \text{null }T^2 $.

        \item 
            $ \lambda = 3 $

            $ E(3, T) = \text{null }(T - 3I) = \text{span }\{(0,1,0)\} $:

            \[
                (T - 3I)(x, y, z) = (2z, 3y, 0) - 3(x, y, z) = (2z - 3x, 0, -3z)
            \]
            $ \implies \text{null }(T - 3I) = \text{span }\{(0,1,0)\} $ with dimension 1.

            \[
                (T - 3I)^2(x, y, z) = (T - 3I)(2z - 3x, 0, -3z) = (-12z + 9x, 0, 9z)
            \]
            $ \implies\text{null }(T - 3I)^2 = \text{span }\{(0,1,0)\} $ with dimension 1.

            $ \text{null }(T - 3I) ^2 = \text{null }T - 3I$

            $ \implies G(3, T) = \text{null }(T - 3I)  = \text{span }\{(0,1,0)\}$
        
    \end{itemize}
\end{eg}

\begin{theorem} \leavevmode
    
    If $ \lambda_1, \cdots , \lambda_{m} $ are distinct eigenvalues for $ T $ and $ v_1, \cdots , v_{m} $ are generalized eigenvectors corresponding to $ \lambda_1, \cdots , \lambda_{m} $, then $ v_1, \cdots ,v_{m} $ are linearly independent.

\end{theorem}

\begin{theorem} \leavevmode
    
    If $ V $ is a complex vector space of $ \dim n $ with $ \lambda_1, \cdots ,\lambda_{m} $ as distinct eigenvalues, then
    \begin{enumerate}
        \item $ V = G(\lambda_1, T) \bigoplus \cdots \bigoplus G(\lambda_{m}, T) $. (There is always enough generalized eigenvectors)
        \item each $ G(\lambda_{i}, T) $ is invariant under $ T $.
    \end{enumerate}

\end{theorem}

\begin{proof} \leavevmode

    2: 

    Suppose $ v \in G(\lambda_{i}, T) $, so $ (T - \lambda_{i}I)v = 0 $.

    Need to show $ (T - \lambda_{i}I)^n(Tv) =T(T - \lambda_{i}I)^n v = 0 $.

    But $ (T - \lambda_{i}I)^nT = T(T - \lambda_{i}I)^n $.

    \vspace{1em}
    1:
    Induction on $ n $.
    $ 1, \cdots , n-1 \implies n $ Pick an eigenvalue $ \lambda_1 $.

    Then, 
    \[
        V = \text{null }(T - \lambda_1I)^n \bigoplus \text{range }(T - \lambda_1I)^n
    \]
\end{proof}

\begin{remark}
    
    This is not true for $ \mathbb{R} $.
    $ T(x, y) = (y, -x) $ has no eigenvalue

\end{remark}

\begin{corollary}
    $ V $ has basis consisting of generalized eigenvectors.
\end{corollary}

\begin{eg}
    
    Previous example: $ T(x, y, z) = (2z, 3y, 0) $ where $ T \in L(\mathbb{C}^3) $.

    \[
        \mathbb{C}^3 = G(0,T) \bigoplus G(3, T)
    \]
    where $ G(0, T) = \text{span }\{(1, 0, 0), (0, 0, 1)\} $ and $ G(3, T) = \text{span }\{(0,1,0)\} $.

    Multiplicity of $ \lambda_{i} $: $ \dim G(\lambda_{i}, T) $.

\end{eg}

\begin{corollary}[for theorem 8.3]
    
    The sum of multiplicities of eigenvalues of $ T = \dim V $.

\end{corollary}

\begin{definition}[Block Diagonal Matrix]
    
    A block diagonal matrix is a square matrix of the form
    \[
        \begin{bmatrix}
          A_1 &  & 0 \\
           & \ddots &  \\
          0 &  & A_{n} \\
        \end{bmatrix}
    \]
    where $ A_1, \cdots ,A_{n} $ are square matrices lying along the diagonal and all other entries of the matrix equal 0.

\end{definition}

\begin{eg} \leavevmode
    
    $ \begin{bmatrix}
      1 & 2 & 0 & 0 & 0 & 0 \\
      0 & 3 & 0 & 0 & 0 & 0 \\
      0 & 0 & 5 & 0 & 0 & 0 \\
      0 & 0 & 0 & -1 & 3 & 4 \\
      0 & 0 & 0 & 0 & -1 & 0 \\
      0 & 0 & 0 & 0 & 0 & 2 \\
    \end{bmatrix} $

    Block diagonal matrices with upper triangular blocks.

\end{eg}

\begin{theorem}
    
    If $ T \in L(V) $, $ F = \mathbb{C} $, $ \lambda_1, \cdots ,\lambda_{m} $ distinct eigenvalues of $ T $, then there is a basis for $ V $ with respect to which
    \[
        M(T) 
        = 
        \begin{bmatrix}
          A_1 &  & 0 \\
           & \ddots &  \\
          0 &  & A_{n} \\
        \end{bmatrix}
    \]
    where $ A_{j}: d_{j} \times d_{j} $ upper-triangular with diagonal entries = $ \lambda_{j} $.

    $ d_{j} = $ multiplicity of $ \lambda_{j} $.

\end{theorem}