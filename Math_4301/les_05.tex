\lesson{5}{Tue Sep 9 2025 14:30}{}

\begin{eg} \leavevmode

    \begin{enumerate}
        \item $ P_{m}(F) = $ vector space of polynomials of degree $ \leqslant m $ with coefficients from $ F $.

        \[
            P_{m}(F) = \{a_{m}x^m + \cdots + a_1 x + a_0\}
        \]
        A basis is $ 1, x, x^2, \cdots ,x^m $ for $ P_{m}(F) $

        \vspace{1em}
        \item $ m = 3 $ \hspace{3em} $ 1, x, x^2, x^3 $ a basis for $ P_3(\mathbb{R}) $. 
    \end{enumerate}
\end{eg}


\section{Dimension}
\vspace{1em}
\begin{lemma}
    Let $ V $ be a vector space, then the length of every linear independent list is $ \leqslant  $ the length of every spanning list.

    \vspace{1em}
    \noindent $ v_1, \cdots , v_{m} $ linearly independent, \hspace{4em} $ w_1, \cdots ,w_{n}: $ spanning $ V = \text{span}(w_1, \cdots , w_{n}) $, then 
    \[
        m \leqslant n
    \]
\end{lemma}

\vspace{1em}
\begin{proof} sketch of proof \leavevmode

    $ v_1, w_1, w_2, \cdots ,w_{n} $ not linearly independent.
    
    ($ v_1 \in V = \text{span}(w_1, \cdots , w_{n}) \implies v_1 = a_1w_1 + a_2w_2 + \cdots + a_{n}w_{n} \implies v_1 - a_1w_1 - \cdots - a_{n}w_{n} = 0 $ ).

    \vspace{1em}
    By the linear dependence lemma, we can remove one of the vectors without changing the space and we can choose this vector to be in the span of the previous ones. This vector cannot be $ v_1 $ (otherwise $ v_1 = 0 $ contradicting $ v_1, \cdots ,v_{n} $ is linearly independent.) So we get a new spanning list of length $ n $: $ v_1, w_1, \cdots , w_{j-1}, w_{j+1}, \cdots ,w_{n} $
    
    \vspace{1em}
    Continue with $ v_2 $: $ v_2, v_1, w_1, \cdots , w_{j-1}, w_{j+1}, \cdots w_{n} $ is not linearly independent. Apply linear dependence lemma to get a spanning list of length $ n $. $ v_2, v_1, w_1, \cdots ,w_{j-1}, w_{j+1}, \cdots , w_{k-1}, w_{k+1}, \cdots , w_{n} $.

    \vspace{1em}
    Continue until all the $ v_{i} $ are inserted in a list of length $ n $. This shows $ m \leqslant n $.
\end{proof}


\vspace{1em}
\begin{corollary}
    Any two bases in a finite-dimensional vector space have the same length.
\end{corollary}

\vspace{1em}
\begin{proof}
    Suppose we have two basis (both spanning and linearly independent as they are basis):
    \begin{enumerate}
        \item $ v_1, v_2, \cdots , v_{n} $ spanning
        \item $ w_1, w_2, \cdots , w_{m} $ linearly independent
    \end{enumerate}
    By Lemma 2, $ m \leqslant n $. Reverse it, then we got $ n \leqslant m $.
\end{proof}

\begin{remark}
    The common length of vector space $ V $ is $ \dim (V) $.

    \vspace{1em}
    \begin{itemize}
        \item $ \dim (\mathbb{R}^n) = n $
        \item $ \dim (P_3(\mathbb{R})) = 4 $
        \item $ \dim (P_{m}(F)) = m+1 $
        \item convention: $ \dim (0) = 0 $
    \end{itemize}

\end{remark}

\vspace{1em}
\begin{eg}
    If $ V = \text{span }(v_1 = (1, 0, 2), v_2 = (-1, 1, \frac{1}{2})) $ in $ \mathbb{R}^3 $. What is $ \dim (V) $?

    \vspace{1em}
    Answer: 2

    $ v_1, v_2 $: span $ V $ and they are linearly independent, so the form a basis for $ V $.
\end{eg}

\vspace{1em}
\begin{theorem}
    If $ V $ is a finite-dimensional vector space, then any linearly independent list can be extended to a basis.
\end{theorem}

\begin{proof}
    Suppose $ v_1, \cdots , v_{n} $ is linearly independent and suppose $ \dim V = m $. We know $ n \leqslant  m $.

    If $ v_1, \cdots , v_{n} $ is spanning, we are done. Otherwise, there is $ w \in V $ such that $ w \notin $ span$ (v_1, \cdots , v_{n}) $. Then $ v_1, \cdots , v_{n} , w $ is linearly independent by the linear dependence lemma (none of the vectors is in the span of the previous ones.) 
    
    Now continue with $ v_1, \cdots , v_{n}, w $. If span, we are done. Otherwise, we can extend it to a linearly independent list of length $ n + 2 $

    \vdots

    in $ m-n $ steps, we will get to a linearly independent list which has to be spanning.
\end{proof}

\vspace{1em}
\begin{eg} \leavevmode

    \noindent Let $ v_1, v_2, v_3 $ be linearly independent in $ \mathbb{R}^3 $. What is span $ (v_1, v_2, v_3) $? 

    \vspace{1em}
    \noindent Answer: span is $ \mathbb{R}^3 $. 

    \noindent reason: $ \dim \mathbb{R}^3 = 3 $, and $ v_1, v_2, v_3 $ are linearly independent, so they are a part of a basis. But every basis has length 3, so $ v_1, v_2, v_3 $ forms a complete basis for $ \mathbb{R}^3 $.
\end{eg}

\begin{theorem}
    Let $ \dim V = n. $ Let $ v_1, \cdots , v_{n} $ be a list of vector in $ V $. Then the followings are equivalent:
    \begin{enumerate}
        \item $ v_1, \cdots , v_{n} $ is spanning.
        \item $ v_1, \cdots , v_{n} $ is a basis.
        \item $ v_1, \cdots , v_{n} $ is linearly independent.
    \end{enumerate}
\end{theorem}

\vspace{1em}
\begin{proof} \leavevmode

    \noindent From 1 to 2: Every spanning list can be reduced to a basis, but every basis has length n, so $ v_1, \cdots , v_{n} $ is a basis.

    \vspace{1em}
    \noindent From 3 to 2: Every linearly independent list can be extended to a basis but every basis has length $ n $.
\end{proof}

\vspace{1em}
\begin{eg}
    in $ \mathbb{R}^2 $, $ \dim (\mathbb{R}^2) = 2 $. If $ v_1, v_2 $ are linearly independent, they form a basis. 
\end{eg}

\vspace{1em}
\begin{theorem} 2.37 \leavevmode
    
    If $ V $ is a finite-dimensional vector space, and $ U $ is a subspace of $ V $, then $ U $ is also finite-dimensional and $ \dim U \leqslant \dim V $
\end{theorem}

\vspace{1em}
\begin{proof} \leavevmode

    \noindent Suppose $ \dim V = n $. If $ U = \{0\} $, then $ \dim U = 0 \leqslant \dim V $.
    So suppose $ U \neq \{0\} $. Let $ u_1, \cdots , u_{m} $ be the largest linearly independet list in $ U $. since $ u_1, \cdots , u_{m} $ is also linearly independent in $ V $, we have $ m \leqslant n $.
    Claim $ u_1, \cdots , u_{m} $ is a basis for $ U $.

    If $ \text{span}(u_1, \cdots , u_{m}) \neq U $, we can choose $ u_{m+1} \in U, u_{m+1} \notin \text{span}(u_1, \cdots , u_{m}) $. Then, $ u_1, \cdots , u_{m}, u_{m+1} $ is a linearly independent list of vectors in $ U $.

    Contradiction. So $ \dim U = m \leqslant n = \dim V $.
\end{proof}