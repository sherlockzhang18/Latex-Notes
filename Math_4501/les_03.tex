\lesson{3}{Fri Aug 29 2025 15:00}{}

\underbar{\textbf{Bijection Method:}} obtain two sequences $ (a_n)_{n=0}^\infty $ and $ (b_n)_{n=0}^\infty $ where $ a_0 = a, b_0 = b $.

$ |a_n - b_n|  \leqslant 2^{-n} |a-b| \implies \lim_{r \to \infty} a_n = \lim_{r \to \infty} b_n $ if limits exists

$ |a_{n+1} - a_n| \leqslant  2^{-n} |a-b| $, $ |b_{n+1} - b_n| \leqslant  2^{-n} |a-b| $, $ \implies (a_n) $ and $ (b_n) $ are Cauchy. (Cauchy sequences in $ \mathbb{R} $ converges to a point in $ \mathbb{R} $).

\vspace{2em}
\begin{proof} \underbar{\textbf{Fixed point theorem}} \leavevmode

    Let $ f(x) = x - g(x) $, then $ \exists \xi $ such that $ f(\xi) = 0 $ if $ (a-g(a))(b-g(b)) \leqslant 0$. So since $ g(a) \geqslant a $, $ g(b) \leqslant b $, by previous theorem, there exist $ \xi $.
\end{proof}


\vspace{2em}
\begin{definition}
    Simple iteration: given $ x_0 \in [a,b] $, define $ (x_k)_{k=0}^\infty $
    \[
        x_{k+1} = g(x_k)
    \]
\end{definition}

\vspace{1em}
If a simple iteration converges, then it converges to a fixed point.

\vspace{1em}
Let $ c = \lim_{k \to \infty} x_{k+1} = \lim_{k \to \infty} g(x_k) = g (\lim_{k \to \infty} x_k) = g(c)$ (Since $ g $ is continuous)

\begin{definition}
    $ g: [a,b] \to \mathbb{R} $ continuous $ \iff $ g is continuous at $ x_0 \in [a,b] $.

    \vspace{1em}
    Topological Definition:
    For all $ \epsilon >0 $, there exists a $ \delta > 0 $ such that $ |x-x_0| < \delta, x \in [a,b] \implies |g(x) - g(x_0) | < \epsilon$

    \vspace{1em}
    g is \underbar{\textbf{lipschitz continuous}} if for all $ x, y \in [a,b] $, then $ |g(x) - g(y)| \leqslant L|x-y| $ for some $ L>0 $.
\end{definition}

\vspace{1em}
\begin{definition}
    g is a \underbar{\textbf{contraction}} if there exists an $ L $ such that $ | g(x) - g(y) | \leqslant L|x-y| $, for all $ x,y \in [a,b] $ and $ 0 \leqslant  L <1 $
\end{definition}

\vspace{1em}
\begin{theorem} \leavevmode
    $ g[a,b] \to [a,b] $ is a contraction. (Contraction $ \implies $ Lipschitz $ \implies $ pointwise continuous)
    
    \begin{itemize}
        \item g has a unique fixed point $ \xi \in [a,b] $
        \item simple iteration ($ x_{k+1} = g(x_k) $) converges to the fixed point $ \xi $
    \end{itemize}
\end{theorem}

\vspace{1em}
\begin{proof} \leavevmode

    \noindent\textbf{Uniqueness}: Suppose $ \xi_{1}, \xi_2 $ are fixed points.
    \begin{align*}
        &|\xi_1-\xi_2| = |g(\xi_1) - g(\xi_2)| \leqslant L|\xi_1-\xi_2| \\
        \implies &(1-L)|\xi_1-\xi_2| \leqslant 0 \\ 
        \implies &|\xi_1-\xi_2| \leqslant 0 \\
        \implies &|\xi_1-\xi_2| = 0
    \end{align*}

    \noindent\textbf{Convergence}: Let $ \xi \in [a,b] $ be the unique fixed point.

    \vspace{1em}
    \noindent Reminder: Series convergence: $ (x_n)_{n=0}^\infty \subseteq \mathbb{R}$ converges if for any $ \epsilon > 0 $, $ \exists N \in \mathbb{N} $ such that $ n \geqslant N \implies |x_n-\xi| < \epsilon $

    \vspace{1em}
    \begin{align*}
        |x_k - \xi| &= |g(x_{k-1}) - g(\xi)| \\
        &\leqslant  L|x_{k-1} - \xi| = L|g(x_{k-2})-g(\xi)| \\
        &\leqslant L^2|x_{k-2} -\xi|   \\
        & \vdots \\
        &\leqslant  L^k|x_0-\xi| \\
    \end{align*}

\end{proof}