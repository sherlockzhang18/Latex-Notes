\lesson{2}{Wed Aug 27 2025 15:00}{Introduction}

\begin{itemize}
    \item Solving nonlinear equations: 
    \begin{itemize}
        \item $ \overrightarrow{f}: \mathbb{R}^m  \to \mathbb{R}^m$, need to solve $ \overrightarrow{f}(\overrightarrow{x}) = 0 $ (m equations, m unknowns).
    \end{itemize}
    
    \item In case $ \overrightarrow{f} $ linear:
    \begin{itemize}
        \item $ \overrightarrow{A}\cdot \overrightarrow{x} = 0 \hspace{2em} \overrightarrow{A} = \left[\overrightarrow{f}(\overrightarrow{e_{1}}) | \overrightarrow{f}(\overrightarrow{e_{2}}) | \cdots | \overrightarrow{f}(\overrightarrow{e_{n}}) \right] $
    \end{itemize}
    
    \item closely related eigenvalue problem.
    $ \overrightarrow{A} \cdot \overrightarrow{x} = \lambda \overrightarrow{x} \hspace{2em} \overrightarrow{x} \neq 0, \lambda \in \mathbb{C} $. (related to root finding for polynomials)
    
    \item polynomial approximations:
    \begin{itemize}
        \item $ f:[0,1] \to \mathbb{R} $, continuous function.
        Want to find polynomial approximation $ P_n(x) = c_0+c_1x + \cdots + c_nx^n $ of degree $ n $

        \item Then, key question is:
        \begin{enumerate}
            \item how to compute $ c_0, \cdots , c_n $ given $ f $?
            \item If $ f $ is continuous differentiable, does $ P_n' $ approximate $ f' $?
            \item If $ f $ is integrable does $ \int p_n dx $ approximate $ \int f dx $
        \end{enumerate}

        \item Deeper Question: Is the approximation efficient?
    \end{itemize}
\end{itemize}


\setcounter{chapter}{0}
\chapter{}
\section{Scalar Problems}

\begin{itemize}
    \item (P1) $ f:[a,b] \to \mathbb{R} $ continuous, find $ \xi \in [a,b] $ such that $ f(\xi) = 0 $
    \item (P2) $ g:[a,b] \to \mathbb{R} $ continuous, find $ \xi \in [a,b] $ such that $ g(\xi) = \xi, g(\xi) - \xi = 0 $
\end{itemize}

P1 and P2 are equivalent, $ f(x): = g(x) - x $ what's the advantage of viewing P(1) as P(2) ?

\vspace{2em}
\underbar{\textbf{When does a solution exists?}}

\begin{itemize}
    \item trivial case: $ f(x) = 0 $ for all $ x \in [a,b] $
    \item WLOG $ f(a) f(b) <0 \implies \exists \xi \in [a,b] $ such taht $ f(\xi) = 0 $. (By Intermediate Value Theorem)
    
    Intermediate Value Theorem (IVP)
    \begin{proof} \underbar{\textbf{Bisection Algorithm}}

        WLOG $ f(a) < 0, f(b) > 0 $

        Set two sequences $ (a_n), (b_n) (c_n) $, $ n = 0, 1, 2, \cdots  $

        Set $ a_0 = a, b_0 = b $

        \underbar{For $ n = 0,1,2 $} define $ c_{n+1} = \frac{a_n + b_n}{2} $
        \begin{itemize}
            \item if $ f(c_{n+1}) = 0 $, then break. set $ \xi = c_{n+1} $
            \item if $ f(c_{n+1}) < 0 $, set $ a_{n+1}  = c_{n+1}, b_{n+1} = b_n $
            \item if $ f(c_{n+1}) > 0 $, set $ b_{n+1}  = c_{n+1}, a_{n+1} = a_n $
        \end{itemize}

        \[
            |a_n - b_n| \leqslant 2^{-n} \cdot |a_0-b_0|
        \]

        So $ f(a_n) < 0 \forall n, f(b_n) > 0 \forall n $

        $ \lim_{n \to \inf} f(a_n) \leqslant 0 \lim_{n \to \inf} f(b_n) \geqslant 0 $

        $ \lim_{n \to \infty} a_{n}= \lim_{n \to \infty} b_{n} $ if limit exists.

        Limit exists because both sequences are Cauchy (or are bounded montonous). So,
        \[
            0 \geqslant \lim_{n \to \infty} f(a_n) = f(\lim_{n \to \infty} a_n) = f(\xi)
        \]

        So there exists $ \xi\in [a,b] $ such that $ f(\xi) = 0 $ (Existence, but not uniqueness)
        
    \end{proof}
\end{itemize}

(P2) $ g: [a,b] \to \mathbb{R} $ continuous, find $ \xi: [a,b] $ such that $ g(\xi) = \xi $. When does a solution exist?  $ g(\xi) - \xi = 0 $. Set $ f(x) = g(x) - x $, then $ f(\xi) = 0 $ exists if $ f(a)f(b) < 0 $, that is $ \left(g(a) - a\right) \left(g(b) - b\right) \leqslant 0 $. 

\vspace{1em}
$ g(a) < a $ and $ g(b) > b $, or \underbar{$ g(a) > a $ and $ g(b) < b $}.(is satisfied if $ g:[a,b] \to [a,b] $)

\setcounter{theorem}{0}
\begin{theorem} \underbar{\textbf{Fixed point theorem}} \leavevmode

    if $ g:[a,b] \to [a,b] $ is continuous then there exists $ \xi \in [a,b] $ such that $ g(\xi) = \xi $
\end{theorem}