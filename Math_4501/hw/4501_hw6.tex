\documentclass[a4paper]{article}

\usepackage[margin=1.3in]{geometry} % reduce margin

% Basic packages
\usepackage[utf8]{inputenc}
\usepackage[T1]{fontenc}
\usepackage{textcomp}
\usepackage{url}
\usepackage{booktabs}
\usepackage{enumitem}
\usepackage[dvipsnames]{xcolor}
\usepackage{xifthen}

% Math packages
\usepackage{amsmath, amsfonts, mathtools, amsthm, amssymb}
\usepackage{mathrsfs}
\usepackage{cancel}
\usepackage{bm}
\usepackage{systeme}
\usepackage{stmaryrd} % for \lightning

% Math shortcuts
\newcommand\N{\ensuremath{\mathbb{N}}}
\newcommand\R{\ensuremath{\mathbb{R}}}
\newcommand\Z{\ensuremath{\mathbb{Z}}}
\renewcommand\O{\ensuremath{\emptyset}}
\newcommand\Q{\ensuremath{\mathbb{Q}}}
\newcommand\C{\ensuremath{\mathbb{C}}}
\newcommand\F{\ensuremath{\mathbb{F}}}
\DeclareMathOperator{\sgn}{sgn}
\DeclareMathOperator{\Ker}{ker}
\DeclareMathOperator{\im}{Im}

% Logic symbols
\let\svlim\lim\def\lim{\svlim\limits}
\let\implies\Rightarrow
\let\impliedby\Leftarrow
\let\iff\Leftrightarrow
\let\epsilon\varepsilon
\newcommand\contra{\scalebox{1.1}{$\lightning$}}

% Useful commands
\definecolor{correct}{HTML}{009900}
\newcommand\correct[2]{\ensuremath{\:}{\color{red}{#1}}\ensuremath{\to }{\color{correct}{#2}}\ensuremath{\:}}
\newcommand\green[1]{{\color{correct}{#1}}}

% Horizontal rule
\newcommand\hr{
    \noindent\rule[0.5ex]{\linewidth}{0.5pt}
}

% Simple theorem environments (without fancy boxes for homework)
\theoremstyle{definition}
\newtheorem{definition}{Definition}
\newtheorem{theorem}{Theorem}
\newtheorem{lemma}{Lemma}
\newtheorem{proposition}{Proposition}
\newtheorem{corollary}{Corollary}
\newtheorem*{remark}{Remark}
\newtheorem*{note}{Note}
\newtheorem*{example}{Example}

% Problem environment
\newcounter{problem}
\newenvironment{problem}[1][]
{
    \stepcounter{problem}
    \section*{Problem \theproblem\ifx\relax#1\relax\else: #1\fi}
}
{}

% Subproblem environment
\newcounter{subproblem}[problem]
\newenvironment{subproblem}[1][]
{
    \stepcounter{subproblem}
    \subsection*{(\alph{subproblem})\ifx\relax#1\relax\else\ #1\fi}
}
{}

% Solution environment
\newenvironment{solution}
{
    \noindent\textbf{Solution:}\\
}
{
    
}

% Headers
\usepackage{fancyhdr}
\pagestyle{fancy}
\fancyhf{}
\fancyhead[L]{Sherlock Zhang}
\fancyhead[C]{4501 - Homework 6}
\fancyhead[R]{\today}
\fancyfoot[C]{\thepage}

% Title info
\title{Math 4501 - Homework 6}
\author{Sherlock Zhang}
\date{\today}

\begin{document}

\maketitle

% =============================================================================
% HOMEWORK PROBLEMS START HERE
% =============================================================================

\setcounter{problem}{1}
\begin{problem}

    Suppose that $ \varphi_{j}, j = 0, 1, \cdots , $ form a system of orthogonal polynomials with weight function $ w(x) $ on the interval $ (a, b) $. Show that, for some value of the constant $ C_{j} $, $ \varphi_{j+1}(x) - C_{j}x\varphi_{j}(x) $ is a polynomial of degree $ j $, and hence that
    \[
        \varphi_{j+1}(x) - C_{j}x\varphi_{j}(x) = \sum_{k = 0}^{j}\alpha_{jk}\varphi_{k}(x), \hspace{1em} \alpha_{jk} \in \mathbb{R}. 
    \]
    Use orthogonality properties to show that $ \alpha_{jk} = 0 $ for $ k < j - 1 $, and deduce that the polynomials satisfy a recurrence relation of the form
    \[
        \varphi_{j+1}(x) = (C_{j}x + D_{j})\varphi_{j}(x) + E_{j}\varphi_{j-1}(x) = 0, \hspace{1em} j \geqslant 1
    \]

\end{problem}

\vspace{2em}
\begin{solution}

    Since both $\varphi_{j+1}$ and $x\varphi_j$ are polynomials of degree $j+1$, we may choose a constant $C_j$ so that the leading terms cancel.  With this choice of $C_j$, the polynomial
    \[
        \psi_j(x) := \varphi_{j+1}(x) - C_j x \varphi_j(x)
    \]
    has degree at most $j$.  Because $\{\varphi_0,\dots,\varphi_j\}$ form an orthogonal basis for the space
    of polynomials of degree $\le j$, we may expand
    \[
        \psi_j(x) = \sum_{k=0}^j \alpha_{j,k}\,\varphi_k(x),
        \qquad
        \alpha_{j,k} = \frac{\langle \psi_j, \varphi_k\rangle}{\langle \varphi_k,\varphi_k\rangle}.
    \]

    now use orthogonality. 
    For $k \le j-2$, the inner products
    \[
        \langle \varphi_{j+1}, \varphi_k\rangle = 0,
        \qquad
        \langle x\varphi_j, \varphi_k\rangle = 0
    \]
    follow results in class, the first because $j+1\ne k$, and the second because $x\varphi_k$ has degree $\le j-1$ and is therefore orthogonal to $\varphi_j$.
    Hence
    \[
        \alpha_{j,k} = 0, \qquad k = 0,\dots,j-2.
    \]

    Thus only the coefficients with indices $j$ and $j-1$ may be nonzero, and we obtain
    \[
        \varphi_{j+1}(x) - C_j x \varphi_j(x)
        = \alpha_{j,j}\,\varphi_j(x) + \alpha_{j,j-1}\,\varphi_{j-1}(x).
    \]
    Setting $D_j := \alpha_{j,j}$ and $E_j := \alpha_{j,j-1}$, we rearrange this to obtain the three-term recurrence
    \[
        \varphi_{j+1}(x)
        = (C_j x + D_j)\varphi_j(x) + E_j\,\varphi_{j-1}(x),
        \qquad j \ge 1.
    \]

\end{solution}

\newpage
\begin{problem}

    The function $ H(x) $ is defined by
    \[
        H(x) = 
        \begin{cases}
            1 & \text{if } x > 0, \\
            0 & \text{if } x = 0, \\
            -1 & \text{if } x < 0.
        \end{cases}
    \]
    Construct the best polynomial approximation $ p_{n} $ of degrees $ n = 0, 1, 2 $ in the 2-norm to this function over the interval $ (-1, 1) $ with weight function $ w(x) \equiv 1 $. If you let $ n \to \infty $ will $ p_{n} $ converge to $ H $?

\end{problem}

\vspace{2em}
\begin{solution}

    We work in the inner product space
    \[
        \langle f,g\rangle = \int_{-1}^1 f(x)g(x)\,dx,
    \]
    with weight function $w(x)\equiv 1$.  The Legendre polynomials
    $\{\varphi_0,\varphi_1,\varphi_2,\dots\}$ form an orthogonal basis on
    $(-1,1)$, and the best polynomial approximation of degree $n$ is the
    orthogonal projection of $H$ onto $\operatorname{span}\{\varphi_0,\dots,\varphi_n\}$.

    Since $H$ is an odd function and $\varphi_0(x)=1$ is even, we have
    $\langle H,\varphi_0\rangle = 0$.  Hence the best approximation of degree
    $0$ is
    \[
        p_0(x) = 0.
    \]

    \medskip
    \noindent\textbf{Degree 1.}
    The degree--1 approximant has the form
    \[
        p_1(x) = \gamma_1 \varphi_1(x),
    \qquad
    \varphi_1(x)=x.
    \]
    By orthogonality,
    \[
        \gamma_1 
        = \frac{\langle H,\varphi_1\rangle}{\langle \varphi_1,\varphi_1\rangle}
        = \frac{\displaystyle \int_{-1}^1 H(x)\,x\,dx}
            {\displaystyle \int_{-1}^1 x^2\,dx}.
    \]
    Because $H$ is odd and $x$ is odd, $H(x)x$ is even, so
    \[
        \int_{-1}^1 H(x)x\,dx
        = 2\int_0^1 x\,dx
        = 1.
    \]
    Also,
    \[
        \int_{-1}^1 x^2\,dx = \frac{2}{3}.
    \]
    Thus
    \[
        \gamma_1 = \frac{1}{2/3} = \frac{3}{2},
    \qquad
        p_1(x) = \frac{3}{2}x.
    \]

    \medskip
    \noindent\textbf{Degree 2.}
    The degree--2 approximant has the form
    \[
        p_2(x)=\gamma_1\varphi_1(x)+\gamma_2\varphi_2(x).
    \]
    Since $H$ is odd and $\varphi_2$ is even, we have $\langle H,\varphi_2\rangle=0$,
    so $\gamma_2=0$.  Therefore $p_2$ coincides with the degree--1 approximant,
    \[
        p_2(x)=\frac{3}{2}x.
    \]

    \medskip
    \noindent\textbf{Convergence.}
    Because $H$ is discontinuous at $x=0$, its projections onto polynomial
    spaces cannot converge uniformly to $H$ on $[-1,1]$, and the sequence
    $\{p_n\}$ exhibits the usual Gibbs--type behaviour near the jump.
    However, as established in class, the orthogonal projections onto Legendre
    polynomials \emph{do} converge to $H$ in the $L^2$ sense:
    \[
        \|p_n - H\|_{L^2(-1,1)} \to 0.
    \]

\end{solution}

\end{document}
