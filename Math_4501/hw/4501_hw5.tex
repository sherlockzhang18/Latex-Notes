\documentclass[a4paper]{article}

\usepackage[margin=1.3in]{geometry} % reduce margin

% Basic packages
\usepackage[utf8]{inputenc}
\usepackage[T1]{fontenc}
\usepackage{textcomp}
\usepackage{url}
\usepackage{booktabs}
\usepackage{enumitem}
\usepackage[dvipsnames]{xcolor}
\usepackage{xifthen}

% Math packages
\usepackage{amsmath, amsfonts, mathtools, amsthm, amssymb}
\usepackage{mathrsfs}
\usepackage{cancel}
\usepackage{bm}
\usepackage{systeme}
\usepackage{stmaryrd} % for \lightning

% Math shortcuts
\newcommand\N{\ensuremath{\mathbb{N}}}
\newcommand\R{\ensuremath{\mathbb{R}}}
\newcommand\Z{\ensuremath{\mathbb{Z}}}
\renewcommand\O{\ensuremath{\emptyset}}
\newcommand\Q{\ensuremath{\mathbb{Q}}}
\newcommand\C{\ensuremath{\mathbb{C}}}
\newcommand\F{\ensuremath{\mathbb{F}}}
\DeclareMathOperator{\sgn}{sgn}
\DeclareMathOperator{\Ker}{ker}
\DeclareMathOperator{\im}{Im}

% Logic symbols
\let\svlim\lim\def\lim{\svlim\limits}
\let\implies\Rightarrow
\let\impliedby\Leftarrow
\let\iff\Leftrightarrow
\let\epsilon\varepsilon
\newcommand\contra{\scalebox{1.1}{$\lightning$}}

% Useful commands
\definecolor{correct}{HTML}{009900}
\newcommand\correct[2]{\ensuremath{\:}{\color{red}{#1}}\ensuremath{\to }{\color{correct}{#2}}\ensuremath{\:}}
\newcommand\green[1]{{\color{correct}{#1}}}

% Horizontal rule
\newcommand\hr{
    \noindent\rule[0.5ex]{\linewidth}{0.5pt}
}

% Simple theorem environments (without fancy boxes for homework)
\theoremstyle{definition}
\newtheorem{definition}{Definition}
\newtheorem{theorem}{Theorem}
\newtheorem{lemma}{Lemma}
\newtheorem{proposition}{Proposition}
\newtheorem{corollary}{Corollary}
\newtheorem*{remark}{Remark}
\newtheorem*{note}{Note}
\newtheorem*{example}{Example}

% Problem environment
\newcounter{problem}
\newenvironment{problem}[1][]
{
    \stepcounter{problem}
    \section*{Problem \theproblem\ifx\relax#1\relax\else: #1\fi}
}
{}

% Subproblem environment
\newcounter{subproblem}[problem]
\newenvironment{subproblem}[1][]
{
    \stepcounter{subproblem}
    \subsection*{(\alph{subproblem})\ifx\relax#1\relax\else\ #1\fi}
}
{}

% Solution environment
\newenvironment{solution}
{
    \noindent\textbf{Solution:}\\
}
{
    
}

% Headers
\usepackage{fancyhdr}
\pagestyle{fancy}
\fancyhf{}
\fancyhead[L]{Sherlock Zhang}
\fancyhead[C]{4501 - Homework 5}
\fancyhead[R]{\today}
\fancyfoot[C]{\thepage}

% Title info
\title{Math 4501 - Homework 5}
\author{Sherlock Zhang}
\date{\today}

\begin{document}

\maketitle

% =============================================================================
% HOMEWORK PROBLEMS START HERE
% =============================================================================

\begin{problem}

    Find a matrix whose characteristic polynomial is equal to a given cubic polynomial
    \[
        p(x) = a_3x^3 + a_2x^2 + a_1x + a_0
    \]

\end{problem}

\vspace{2em}
\begin{solution}

    For a 3 $ \times $ 3 matrix, let the characteristic polynomial $ \text{det }(A - \lambda I) $ be  monic. By the notes in class, we know that it has the form of:
    \[
        \text{det }(A - \lambda I) = \lambda^3 - \text{tr}(A)\lambda^2 + \cdots 
    \]

    So $ a_3 = 1 $. Since $ \text{det }(A - \lambda I) = x^3 + a_2x^2 + a_1x + a_0 = 0 $, which implies $ x^3 = -a_2x^2 - a_1x - a_0 $. 

    Let $ v = \begin{bmatrix}
      x^2 \\
      x \\
      1 \\
    \end{bmatrix} $, then $ xv = \begin{bmatrix}
      x^3 \\
      x^2 \\
      x \\
    \end{bmatrix} $. Replace $ x^3 $ by the equation above, 
    \[
        xv = \begin{bmatrix}
          -a_2x^2 - a_1x - a_0 \\
          x^2 \\
          x \\
        \end{bmatrix}
    \]

    Now only need to find matrix $ A $ such that $ Av = xv $, since $ xv $ is linear combination of $ 1, x, x^2 $.

    The matrix is 
    \[
        A = \begin{bmatrix}
          -a_2 & -a_1 & -a_0 \\
          1 & 0 & 0 \\
          0 & 1 & 0 \\
        \end{bmatrix}
    \]

\end{solution}

\newpage
\begin{problem}



\end{problem}

\setcounter{subproblem}{1}
\begin{subproblem}

    Find the eigenvalues of $ A $ explicitly, then compute their errors.

\end{subproblem}

\vspace{2em}
\begin{solution}
    
    From Chebyshev polynomials discussed in class, we calculate eigenvalues of tridiagonal symmetric Toeplitz matrix by substituting $ \lambda = 2 - 2z $, and the final formula is 
    \[
        z_{m} = \cos\left(\frac{m\pi}{n+1}\right) \hspace{1em} \text{for }m = 1, \cdots , n
    \]

    Using that substitution, we get the new formula to calculate $ \lambda $
    \[
        \lambda_{m} = 2 - 2\cos\left(\frac{m\pi}{n+1}\right) \hspace{1em} \text{for }m = 1, \cdots , n
    \] where in this case $ n = 8 $.

    The calculation is done in jupyter notebook using python.

\end{solution}

\end{document}
