\lesson{21}{Thr Nov 13 2025 14:30}{}

\begin{eg} \leavevmode
    
    $ T \in L(\mathbb{R}^4) $
    \[
        T(x, y, z, w) = (2x + w, 2y, -z, -w)
    \]
    If $ p(x) = x^2 + ax + b $, then $ T^2 + aT + bI = 0 $, so $ T^2 = -aT -bI $

    Calculate $ T^2 = (4x + w, 4y, z, w) $.

    So $ T^2 = T + 2I $, so $ p(x) = x^2 - x - 2 $ is the minimal polynomial.

\end{eg}

\begin{theorem}
    
    The minimal polynomial $ p(x) $ exists, and $ \text{deg }p(x) \leqslant \dim V $.

\end{theorem}

\begin{explanation}
    
    $ \dim V = n  \implies \dim L(V) = n^2 $, so $ I, T, T^2, \cdots , T^{n^2} $ can not be linearly independent. So there must be a non-zero polynomial that can zero out the value.

\end{explanation}

\begin{theorem} $ T \in L(v) $, and $ p(x) $ is the minimal polynomial of $ T $.
    
    \begin{enumerate}
        \item $ \lambda $ is a zero of $ p(x)  \iff  \lambda$ is an eigenvalue of $ T $.
        \item $ T $ is not invertible $ \iff $ the constant term of $ p(x) $ is 0.
        \item $ q(T) = 0 \iff q(x) $ is a multiple of $ p(x) $. i.e $ q(x) = p(x)f(x) $
    \end{enumerate}

\end{theorem}

\begin{proof} \leavevmode
    
    \begin{enumerate}
        \item 
            $ \implies $

            \vspace{1em}
            If $ p(\lambda) = 0 $, then $ p(x) = (x -\lambda)q(x) $

            $ 0 =  p(T) = (T - \lambda I)q(T) $.

            $ q(T) \neq 0 $ because $ \text{deg }q = \text{deg }(p-1) $, so there is $ v \in V $ such that $ w = q(T)v \neq 0 $. 

            So $ 0 = (T - \lambda I)q(T)v = (T - \lambda I) w \implies \lambda $ is an eigenvalue.

            \vspace{2em}
            $ \impliedby $

            \vspace{1em}
            If $ \lambda $ is an eigenvalue, then there is $ v \neq 0 $ such that $ Tv = \lambda v $.

            Then, $ T^nv = \lambda^n v $.

            $ \implies q(T)v = q(\lambda)v $. Since $ p(T) = 0 $, $ p(T)v = 0 $, so $ p(\lambda)v = 0 \implies p(\lambda) = 0 $. 

        \item 
            If $ p(x) = x^n + a_{n-1}x^{n-1} + \cdots + a_1x $ $ \iff $ $ p(0) = 0  \iff $ 0 is an eigenvalue $ \iff T $  is not invertible.

        \item 
            $ \implies $

            \vspace{1em}
            If $ f(x) = p(x)q(x) + r(x) $ with $ \text{deg }r < \text{deg }p $. Then $ f(T) = p(T)q(T) + r(T) $.

            So $ 0 = 0 + r(T) \implies r(T) = 0 $. By $ \text{deg }r < \text{deg }p $, this shows that $ r = 0 $.

            \vspace{1em}
            $ \impliedby $

            \vspace{1em}
            If $ f(x) = p(x)q(x) $, then
            \[
                f(T) = p(T)q(T) = 0q(T) = 0
            \]

    \end{enumerate}

\end{proof}

\begin{eg} \leavevmode
    
    Suppose $ M(T) $ is upper-triangular and $ p(x) $ is the minimal polynomial. $ V $ is a 3-dimensional vector space over $ \mathbb{C} $.
    \begin{enumerate}
        \item What is $ p(x) $ if $ M(T) = \begin{bmatrix}
          1 & 0 & 0 \\
          0 & 1 & 1 \\
          0 & 0 & 2 \\
        \end{bmatrix} $

        \underbar{\textbf{Solution:}}
        \[
            M(T - I) = \begin{bmatrix}
                0 & 0 & 0 \\
                0 & 0 & 1 \\
                0 & 0 & 1 \\
            \end{bmatrix}
        \]
        \[
            M(T - 2I) = \begin{bmatrix}
                -1 & 0 & 0 \\
                0 & -1 & 1 \\
                0 & 0 & 0 \\
            \end{bmatrix}
        \]

        let $ q(x) = (x - 1)(x - 2) $. Is $ q(T) = 0 $? i.e. Is $ (T - I)(T - 2I) = 0 $?
        \[
            \begin{bmatrix}
                0 & 0 & 0 \\
                0 & 0 & 1 \\
                0 & 0 & 1 \\
            \end{bmatrix}\cdot
            \begin{bmatrix}
                -1 & 0 & 0 \\
                0 & -1 & 1 \\
                0 & 0 & 0 \\
            \end{bmatrix}
             = 
             0
        \]


        \item What is $ p(x) $ if $ M(T) = \begin{bmatrix}
          1 & 1 & 5 \\
          0 & 1 & 4 \\
          0 & 0 & -1 \\
        \end{bmatrix} $
    \end{enumerate}

\end{eg}