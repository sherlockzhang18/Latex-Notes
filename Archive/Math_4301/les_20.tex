\lesson{20}{Tue Nov 11 2025 14:30}{}

\begin{eg} \leavevmode
    
    $ T \in L(\mathbb{C}^2) $, $ T(z,w) = (z+w, w) $ is not diagonalizable.

    \vspace{1em}
    $ M(T) $ in standard basis: $ \begin{bmatrix}
      1 & 1 \\
      0 & 1 \\
    \end{bmatrix} $

    so eigenvalue is $ 1 $.

    If $ T $ is diagonalizable, then $ T $ has 2 linearly independent eigenvectors corresponding to 1. It shows that $ T $ is the identity operator, which reaches contradiction.

\end{eg}

\begin{definition}[eigenspace] \leavevmode
    
    $ T \in L(V), \lambda \in F $. Then
    \begin{align*}
        E(\lambda, T) 
        &= \text{null }(T - \lambda I) \\
        &= \{v \in V | Tv = \lambda v\} \\
        &= \{\text{eigenvectors corresponding to } \lambda \}\cup \{0\} \\
    \end{align*}

    \vspace{1em}
    $ E(\lambda, T) $ is a subspace of $ V $.

\end{definition}

This implies $ E(\lambda, T) = \{0\} \iff \lambda$ is not an eigenvalue.


\vspace{2em}
\begin{eg} \leavevmode
    
    $ T \in L(\mathbb{R}^4) $
    \[
        T(x, y, z, w) = (y + 3z, 2z, -z, -w)
    \]

    Find the dimension of all the eigenspaces corresponding to eigenvalues.

    \vspace{1em}
    \[
        M(T) = \begin{bmatrix}
          0 & 1 & 3 & 0 \\
          0 & 0 & 2 & 0 \\
          0 & 0 & -1 & 0 \\
          0 & 0 & 0 & -1 \\
        \end{bmatrix}
    \]
    Upper Triangular Matrix $ \implies $ eigenvalues are 0, -1.

    $ E(0,T) = \text{null }(T) = \{(x, y, z, w): T(x, y, z, w) = 0\} $.

    Solve the function we got $ z = w = y = 0 $.

    $ \implies \text{null }T = \{(x, 0, 0, 0 | x \in \mathbb{R})\} = \text{span }\{(1, 0, 0, 0)\} $.

    \vspace{1em}
    $ E(-1, T) = \text{null }(T + I) = \{(x, y, z, w) | T(x, y, z, w) = -(x, y, z, w)\} $

    Solve the function we got $ x = -2 $, $ y = -2z $

    $ \implies E(-1, T) = \{-z, -2z, z, w\} = \text{span }\{(-1, -2, 1, 0), (0, 0, 0, 1)\} $.

\end{eg}

\vspace{1em}
\begin{theorem}
    
    If $ \lambda_1, \cdots , \lambda_{m} $ are distinct eigenvalues for $ T $, then $ E(\lambda_1, T) +\cdots + E(\lambda_{m}, T) \subseteq V $ is a direct sum.

\end{theorem}

\begin{proof}
    
    If $ v_{i} \in E(\lambda_{i}, T) $ and $ v_1 + \cdots + v_{m} = 0 $. But if $ v_{i} $ are nonzero, then since $ v_1, \cdots , v_{n} $ correspond to distinct eigenvalues , so by a previous result they are linearly independent.

\end{proof}

\begin{theorem}
    
\begin{align*}      
    T \text{ is diagonalizable } 
    & \iff E(\lambda_1, T) + \cdots + E(\lambda_{m}, T) \\
    & \iff \sum_{i=1}^{m} \dim E(\lambda_{i}, T) = \dim V
\end{align*}

\end{theorem}

\begin{corollary}
    
    If $ \dim V = n $ and $ T $ has $ n $ distinct eigenvalues, then $ T $ is diagonalizable.

\end{corollary}

The other way might not be true. eg: $ T = I $ has only 1 eigenvalue: 1 and $ T $ is diagonalizable.

\begin{eg}
    
    $ T \in L(\mathbb{R}^3) $
    \[
        T(x, y, z) = (x + 3y, 2y + 5z, -z)
    \]
    Find a basis with respect to which $ T $ is diagonalizable.

    \vspace{1em}
    $ M(T) = \begin{bmatrix}
      1 & 3 & 0 \\
      0 & 2 & 5 \\
      0 & 0 & -1 \\
    \end{bmatrix} $

    3 eigenvalues means diagonalizable.

\end{eg}

\begin{definition}[monic polynomial]
    
    $ p(x) \in P(F) $ is monic if the highest-degree coefficient equals 1.

\end{definition}

\begin{theorem}[minimal polynomial] \leavevmode
    
    If $ \dim V = n $, $ T \in L(V) $, then there is a unique monic polynomial of smallest degree $ p(x) \in P(F) $ such that $ P(T) = 0 $.

    \vspace{1em}
    $ \text{deg }p(x) = 1 \iff T = cI$ for some $ c $.

\end{theorem}