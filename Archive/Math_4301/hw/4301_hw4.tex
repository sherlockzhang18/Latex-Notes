\documentclass[a4paper]{article}

% \usepackage[margin=1in]{geometry} // reduce margin

% Basic packages
\usepackage[utf8]{inputenc}
\usepackage[T1]{fontenc}
\usepackage{textcomp}
\usepackage{url}
\usepackage{booktabs}
\usepackage{enumitem}
\usepackage[dvipsnames]{xcolor}
\usepackage{xifthen}

% Math packages
\usepackage{amsmath, amsfonts, mathtools, amsthm, amssymb}
\usepackage{mathrsfs}
\usepackage{cancel}
\usepackage{bm}
\usepackage{systeme}
\usepackage{stmaryrd} % for \lightning

% Math shortcuts
\newcommand\N{\ensuremath{\mathbb{N}}}
\newcommand\R{\ensuremath{\mathbb{R}}}
\newcommand\Z{\ensuremath{\mathbb{Z}}}
\renewcommand\O{\ensuremath{\emptyset}}
\newcommand\Q{\ensuremath{\mathbb{Q}}}
\newcommand\C{\ensuremath{\mathbb{C}}}
\newcommand\F{\ensuremath{\mathbb{F}}}
\DeclareMathOperator{\sgn}{sgn}
\DeclareMathOperator{\Ker}{ker}
\DeclareMathOperator{\im}{Im}

% Logic symbols
\let\svlim\lim\def\lim{\svlim\limits}
\let\implies\Rightarrow
\let\impliedby\Leftarrow
\let\iff\Leftrightarrow
\let\epsilon\varepsilon
\newcommand\contra{\scalebox{1.1}{$\lightning$}}

% Useful commands
\definecolor{correct}{HTML}{009900}
\newcommand\correct[2]{\ensuremath{\:}{\color{red}{#1}}\ensuremath{\to }{\color{correct}{#2}}\ensuremath{\:}}
\newcommand\green[1]{{\color{correct}{#1}}}

% Horizontal rule
\newcommand\hr{
    \noindent\rule[0.5ex]{\linewidth}{0.5pt}
}

% Simple theorem environments (without fancy boxes for homework)
\theoremstyle{definition}
\newtheorem{definition}{Definition}
\newtheorem{theorem}{Theorem}
\newtheorem{lemma}{Lemma}
\newtheorem{proposition}{Proposition}
\newtheorem{corollary}{Corollary}
\newtheorem*{remark}{Remark}
\newtheorem*{note}{Note}
\newtheorem*{example}{Example}

% Problem environment
\newcounter{problem}
\newenvironment{problem}[1][]
{
    \stepcounter{problem}
    \section*{Problem \theproblem\ifx\relax#1\relax\else: #1\fi}
}
{}

% Subproblem environment
\newcounter{subproblem}[problem]
\newenvironment{subproblem}[1][]
{
    \stepcounter{subproblem}
    \subsection*{(\alph{subproblem})\ifx\relax#1\relax\else\ #1\fi}
}
{}

% Solution environment
\newenvironment{solution}
{
    \noindent\textbf{Solution:}\\
}
{
    
}

% Headers
\usepackage{fancyhdr}
\pagestyle{fancy}
\fancyhf{}
\fancyhead[L]{Sherlock Zhang}
\fancyhead[C]{4301 linear algebra - Homework 4}
\fancyhead[R]{\today}
\fancyfoot[C]{\thepage}

% Title info
\title{Math 4301 - Homework 4}
\author{Sherlock Zhang}
\date{\today}

\begin{document}

\maketitle

% =============================================================================
% HOMEWORK PROBLEMS START HERE
% =============================================================================

\begin{problem}[3A.3]

    Suppose that $ T \in L(F^n, F^m) $. Show that there exist scalars $ A_{j,k} \in F $ for $ j = 1, \cdots , m $  and $ k = 1, \cdots ,n $ such that
    \[
        T(x_1, \cdots , x_{n}) = (A_{1,1}x_1 + \cdots + A_{1,n} x_{n}, \cdots A_{m,1}x_1 + A_{m,n}x_{n})
    \]
    for every $ (x_1, \cdots , x_{n}) \in F^n $.

\end{problem}

\vspace{2em}
\begin{solution}

    let $ e_1, \cdots ,e_{n} $ be the standard basis for $ F^n $, and $ f_1, \cdots , f_{m} $ be the standard basis for $ F^m $. Then, there exist $ A_{i,j} $
    \[
        T(e_{j}) = \sum_{i = 1}^{m} A_{i,j}f_{i}
    \]

    Then,
    \begin{align*}
        T(x_1, \cdots , x_{n}) 
        &= T\left( \sum_{j = 1}^{n} x_{j}e_{j}\right) \\
        &= \sum_{j = 1}^{n} x_{j}T(e_{j}) \\
        &= \sum_{j = 1}^{n} \left(x_{j} \sum_{i = 1}^{m} A_{i,j}f_{i}\right) \\
        &= \sum_{j = 1}^{n} \sum_{i = 1}^{m} x_{j} A_{i,j} f_{i} \\
        &= \sum_{i = 1}^{m} \sum_{j = 1}^{n} x_{j} A_{i,j} f_{i} \\
        &= \sum_{i = 1}^{m} f_{i} \sum_{j = 1}^{n} x_{j} A_{i,j} \\
    \end{align*}

    The derived answer is the same as $ (A_{1,1}x_1 + \cdots + A_{1,n} x_{n}, \cdots A_{m,1}x_1 + A_{m,n}x_{n}) $.

\end{solution}

\newpage
\begin{problem}[3A.12]
    
    Suppose $ U $ is a subspace of $ V $ with $ U \neq V $. Suppose $ S \in L(U, W) $ and $ S \neq 0 $ (which means that $ Su \neq 0 $ for some $ u \in U $ ). Define $ T: V \to W $ by
    \[
        Tv = 
        \begin{cases}
            Sv & \text{ if } v \in U, \\
            0  & \text{ if } v \in V \text{ and } v \notin U.
        \end{cases}
    \]
    Prove that $ T $ is not a linear map on $ V $.

\end{problem}

\vspace{2em}
\begin{solution}
    
    Let $ u \in U $ such that $ Su \neq 0 $. Let $ v \in V and v \notin U $. Then $ Tv = 0 $. 

    If $ T $ is linear, then $ T(u - v) = Tu - Tv = Su - 0 = Su \neq 0 $. 

    However, $ u-v \notin U $ as if $ u-v $ is in $ U $, then $ v = - (u-v) + u $ will also be in $ U $. Therefore, $ T(u-v) = 0 $. This reaches the contradiction, which shows that $ T $ is not linear.

\end{solution}


\newpage
\begin{problem}[3A.13]

    Suppose $ V $ is finite-dimensional. Prove that every linear map on a subspace of $ V $ can be extended to a linear map on $ V $. In other words, show that if $ U $ is a subspace of $ V $ and $ S \in L(U, W) $, then there exists $ T \in L(V, W) $ such that $ Tu = Su $ for all $ u \in U $.

\end{problem}

\vspace{2em}
\begin{solution}
    
    Let $ u_1, \cdots , u_{n} $ be a basis for $ U $. As $ U $ is a subspace of $ V $, it can be extended to
    \[
        u_1, \cdots , u_{n}, v_1, \cdots v_{m}
    \] 
    which is a basis for $ V $.

    For any $ u \in U $, it can be written as $ u = a_1u_1 + \cdots + a_{n}u_{n} $. 
    
    Define $ T(u) = a_1u_1 + a_2u_2 + \cdots + a_{n}u_{n} $. For any $ 0 < i \leqslant m $ $ T(v_{i}) = 0 $. Thus, for any $ u \in U $, 
    \[
        T(u) = T(a_1u_1 + \cdots + a_{n}u_{n}) = a_1Tu_1 + \cdots + a_{n}Tu_{n} = a_1Su_1 + \cdots + a_{n}Su_{n} = S(u)
    \]

\end{solution}

\newpage
\begin{problem}[3B.5]
    
    Give an example of $ T \in L(\mathbb{R}^4) $ such that $ \text{range}T = \text{null}T $.

\end{problem}

\vspace{2em}
\begin{solution}
    
    Let $ T(x_1, x_2, x_3, x_4) = (0, 0, x_1, x_2) $.
    
    Both range and null are $ (0, 0, a, b) a, b \in \mathbb{R} $.

\end{solution}

\newpage
\begin{problem}[3B.6]
    
    Prove that there does not exist $ T \in L(\mathbb{R}^5) $ such that $ \text{range} T = \text{null}T $.

\end{problem}

\vspace{2em}
\begin{solution}
    
    By Fundamental Theorem of Linear Map, 
    \[
        \dim (\mathbb{R}^5) = \dim \text{null}T + \dim \text{range}T
    \]

    If $ \text{range}T = \text{null}T $, then $ \dim (\mathbb{R}^5) = 2 \dim \text{null}T = 5 $. This does not make sense as the dimension of $ \text{null}T $ should be non-negative integer.

\end{solution}

\newpage
\begin{problem}[3B.9]
    
    Suppose $ T \in L(V, W) $ is injective and $ v_1, \cdots , v_{n} $ is linearly independent in $ V $. Prove that $ Tv_1, \cdots , Tv_{n} $ is linearly independent in $ W $.

\end{problem}

\vspace{2em}
\begin{solution}
    
    As $ T $ is injective, $ \text{null}T = \{0\} $. Assume $ a_1, \cdots , a_{n} $ are scalars, if the only solution to
    \[
        a_1Tv_1 + a_2Tv_2 + \cdots + a_{n}Tv_{n} = 0
    \]
    is trivial, then $ Tv_1, \cdots , Tv_{n} $ is linearly independent in $ W $.

    As $ T $ is linear, 
    \[
        a_1Tv_1 + a_2Tv_2 + \cdots + a_{n}Tv_{n} = T(a_1v_1 + a_2 v_2 + \cdots a_{n}v_{n}) = 0
    \]

    $ \text{null}T = \{0\} $, $ a_1v_1 + \cdots a_{n}v_{n} = 0 $ is the only solution. Since $ v_1, \cdots , v_{n} $ is linearly indepentent, $ a_1 = a_2 = \cdots = a_{n} = 0 $.

    \vspace{1em}
    Thus, $ Tv_1, \cdots , Tv_{n} $ is linearly independent in $ W $.

\end{solution}

\newpage
\begin{problem}[3B.13]
    
    Suppose $ U $ is a three-dimensional subspace of $ \mathbb{R}^8 $ and $ T $ is a linear map from $ \mathbb{R}^8 \to \mathbb{R}^5 $ such that $ \text{null}T = U $. Prove that $ T $ is surjective.

\end{problem}

\vspace{2em}
\begin{solution}
    
    By fundamental theorem of linear map,
    \[
        \dim \text{null}T + \dim \text{range} T = \dim \mathbb{R}^8 = 8
    \]

    As $ \dim U = 3 $ and $ \text{null}T = U $, $\dim \text{null}T = 3 $, so $ \dim \text{range}T = 5 = \dim \mathbb{R}^5 $. As $ \text{range}T $ is a subspace of $ \mathbb{R}^5 $, $ \text{range}T = \mathbb{R}^5 $.
    
    Therefore, $ T $ is surjective.

\end{solution}

\newpage
\begin{problem}[3C.4]
    
    Suppose that $ D \in L(P_3(\mathbb{R}), P_2(\mathbb{R})) $ is the differentiation map defined by $ Dp = p' $. Find a basis of $ P_3(\mathbb{R}) $ and a basis of $ P_2(\mathbb{R}) $ such that the matrix of $ D $ with respect to this bases is
    \[
        \begin{pmatrix}
          1 & 0 & 0 & 0 \\
          0 & 1 & 0 & 0 \\
          0 & 0 & 1 & 0 \\
        \end{pmatrix}
    \]

\end{problem}

\vspace{2em}
\begin{solution}
    
    Let $ x^3, x^2, x, 1 $ be the basis in $ P_3(\mathbb{R}) $, and the result of applying $ D $ is $ 3x^2, 2x, 1, 0 $, which is the basis for $ P_2(\mathbb{R}) $.

\end{solution}

\end{document}
