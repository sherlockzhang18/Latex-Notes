\documentclass[a4paper]{article}

% \usepackage[margin=1in]{geometry} // reduce margin

% Basic packages
\usepackage[utf8]{inputenc}
\usepackage[T1]{fontenc}
\usepackage{textcomp}
\usepackage{url}
\usepackage{booktabs}
\usepackage{enumitem}
\usepackage[dvipsnames]{xcolor}
\usepackage{xifthen}

% Math packages
\usepackage{amsmath, amsfonts, mathtools, amsthm, amssymb}
\usepackage{mathrsfs}
\usepackage{cancel}
\usepackage{bm}
\usepackage{systeme}
\usepackage{stmaryrd} % for \lightning

% Math shortcuts
\newcommand\N{\ensuremath{\mathbb{N}}}
\newcommand\R{\ensuremath{\mathbb{R}}}
\newcommand\Z{\ensuremath{\mathbb{Z}}}
\renewcommand\O{\ensuremath{\emptyset}}
\newcommand\Q{\ensuremath{\mathbb{Q}}}
\newcommand\C{\ensuremath{\mathbb{C}}}
\newcommand\F{\ensuremath{\mathbb{F}}}
\DeclareMathOperator{\sgn}{sgn}
\DeclareMathOperator{\Ker}{ker}
\DeclareMathOperator{\im}{Im}

% Logic symbols
\let\svlim\lim\def\lim{\svlim\limits}
\let\implies\Rightarrow
\let\impliedby\Leftarrow
\let\iff\Leftrightarrow
\let\epsilon\varepsilon
\newcommand\contra{\scalebox{1.1}{$\lightning$}}

% Useful commands
\definecolor{correct}{HTML}{009900}
\newcommand\correct[2]{\ensuremath{\:}{\color{red}{#1}}\ensuremath{\to }{\color{correct}{#2}}\ensuremath{\:}}
\newcommand\green[1]{{\color{correct}{#1}}}

% Horizontal rule
\newcommand\hr{
    \noindent\rule[0.5ex]{\linewidth}{0.5pt}
}

% Simple theorem environments (without fancy boxes for homework)
\theoremstyle{definition}
\newtheorem{definition}{Definition}
\newtheorem{theorem}{Theorem}
\newtheorem{lemma}{Lemma}
\newtheorem{proposition}{Proposition}
\newtheorem{corollary}{Corollary}
\newtheorem*{remark}{Remark}
\newtheorem*{note}{Note}
\newtheorem*{example}{Example}

% Problem environment
\newcounter{problem}
\newenvironment{problem}[1][]
{
    \stepcounter{problem}
    \section*{Problem \theproblem\ifx\relax#1\relax\else: #1\fi}
}
{}

% Subproblem environment
\newcounter{subproblem}[problem]
\newenvironment{subproblem}[1][]
{
    \stepcounter{subproblem}
    \subsection*{(\alph{subproblem})\ifx\relax#1\relax\else\ #1\fi}
}
{}

% Solution environment
\newenvironment{solution}
{
    \noindent\textbf{Solution:}\\
}
{
    
}

% Headers
\usepackage{fancyhdr}
\pagestyle{fancy}
\fancyhf{}
\fancyhead[L]{Sherlock Zhang}
\fancyhead[C]{4301 linear algebra - Homework 8}
\fancyhead[R]{\today}
\fancyfoot[C]{\thepage}

% Title info
\title{Math 4301 - Homework 8}
\author{Sherlock Zhang}
\date{\today}

\begin{document}

\maketitle

% =============================================================================
% HOMEWORK PROBLEMS START HERE
% =============================================================================

\begin{problem}[5A.5]

    Suppose $ T \in L(\mathbb{R}^2) $ is defined by $ T(x, y) = (-3y, x) $. Find the eigenvalues of $ T $.

\end{problem}

\vspace{2em}
\begin{solution}

    Let $ \lambda $ be the eigenvalue of $ T $, then $ T(x, y) = \lambda (x, y) = (-3y, x) $.

    so we got $ \begin{cases}
        \lambda x = -3y \\
        \lambda y = x
    \end{cases} $

    \vspace{1em}
    From the second equation we know $ x = \lambda y $, plug in to the first equation, $ \lambda^2 y = -3 y $. So $ \lambda^2 = -3 $. There is no real solution for $ \lambda $. In $ \mathbb{C} $ there is solution for $ \lambda = \pm \sqrt{3}i $.
    
\end{solution}



\newpage
\begin{problem}[5A.8]

    Suppose $ P \in L(V) $ is such that $ P^2 = P $. Prove that if $ \lambda $ is an eigenvalue of $ P $, then $ \lambda = 0 $ or $ \lambda = 1 $.
    
\end{problem}

\vspace{2em}
\begin{solution}
    
    Let $ \lambda $ be the eigenvalue of $ T $ and corresponding eigenvector $ v $, then
    \[
        \lambda v = Pv = P^2v = P(Pv) = P \lambda v = \lambda P v = \lambda^2 v
    \]

    then we got deduced to
    \[
        \lambda v = \lambda^2 v
    \]
    \[
        (\lambda - 1) \lambda v = 0
    \]
    Since $ v $ is the eigenvector, $ v \neq 0 $, so the only two real solutions for $ \lambda  $ will be $ \lambda = 1 $ and $ \lambda = 0 $.

\end{solution}



\newpage
\begin{problem}[5A.9]
    
    Define $ T: P(\mathbb{R}) \to P(\mathbb{R}) $ by $ Tp = p' $. Find all eigenvalues and eigenvectors of $ T $.

\end{problem}

\vspace{2em}
\begin{solution}
    
    Let $ p = c $ to be non-zero constant, then $ p' = 0 $, so $ \lambda = 0 $ is an eigenvalue for $ T $ . For any $ p $ that has $ \text{deg } p $ larger than 1, then if $ \text{deg }p = a $, then $ \text{deg }p' = a - 1 $, so there is no $ \lambda $ such that $ \lambda p = Tp = p' $. Therefore, the only eigenvalue is $ 0 $. The corresponding eigenvectors are non-zero constant polynomials in $ P(\mathbb{R}) $.

\end{solution}


\newpage
\begin{problem}[5A.19]
    
    Show that the forward shift operator $ T \in L(F^{\infty}) $ defined by
    \[
        T(z_1, z_2, \cdots) = (0, z_1, z_2, \cdots )
    \]
    has no eigenvalues.

\end{problem}

\vspace{2em}
\begin{solution}
    
    Let $ \lambda $ be the eigenvalue of $ T $, and $ v = (z_1, z_2, \cdots ) $ with $ v \neq 0 $ . Then
    \[
        (\lambda z_1, \lambda z_2, \cdots ) = (0, z_1, z_2, \cdots )
    \]

    which transfers to the following equality:
    \[
        \begin{cases}
            \lambda z_1 = 0 \\
            \lambda z_2 = z_1 \\
            \vdots
        \end{cases}
    \]

    If $ \lambda = 0 $, then $ \lambda z_2 = 0 $, which shows that $ z_1 = 0 $. Following the similar arguments above, we can show that $ z_1 = z_2 = \cdots = 0 $. 
    
    \vspace{1em}
    If $ \lambda \neq 0 $, then $ z_1 = 0 $. So $ \lambda z_2 = 0 $, then $ z_2 = 0 $ as well. Following the similar arguments above, we can show that $ z_1 = z_2 = \cdots = 0 $.

    In either case, we both showed that eigenvector $ v = 0 $, which is a contracdiction. Therefore, there is no eigenvalue for $ T $.

\end{solution}

\newpage
\begin{problem}[5A.21]
    
    Suppose $ T \in L(V) $ is invertible.

\end{problem}

\vspace{1em}
\begin{subproblem}

    Suppose $ \lambda \in F $ with $ \lambda \neq 0 $. Prove that $ \lambda $ is an eigenvalue of $ T $ if and only if $ \frac{1}{\lambda} $ is an eigenvalue of $ T^{-1} $.

\end{subproblem}

\vspace{1em}
\begin{solution}
    
    If $ \lambda $ is an eigenvalue of $ T $ and $ v \neq 0 $ is the corresponding eigenvector, then 
    \[
        Tv = \lambda v \iff v = T^{-1} \lambda v \iff T^{-1}v = \frac{v}{\lambda}
    \]

    This is equivalent to the statement $ \lambda $ is an eigenvalue of $ T $ if and only if $ \frac{1}{\lambda} $ is an eigenvalue of $ T^{-1} $.

\end{solution}



\vspace{1em}
\begin{subproblem}

    Prove that $ T $ and $ T^{-1} $ have the same eigenvectors.

\end{subproblem}

\vspace{1em}
\begin{solution}
    
    As $ T $ is invertible, by definition both $ T $ and $ T^{-1} $ is injective, so the null space is $ \{0\} $. So 0 cannot be the eigenvalue. For non-zero eigenvalues, using the equation above:
    \[
        Tv = \lambda v \iff v = T^{-1} \lambda v \iff T^{-1}v = \frac{v}{\lambda}
    \]

    This shows that both $ T $ and $ T^{-1} $ has the same eigenvector.

\end{solution}

\newpage
\begin{problem}[5A.23]
    
    Suppose $ V $ is finite-dimensional and $ S, T \in L(V) $. Prove that $ ST $ and $ TS $ have the same eigenvalues.

\end{problem}

\vspace{2em}
\begin{solution}
    
    Let $ \lambda $ be the eigenvalue of $ ST $, then let $ v \neq 0 $ be the corresponding eigenvector.
    If $ \lambda \neq 0 $, then we know $ ST v = \lambda v $
    \[
        (TS)(Tv) = T(STv) = T(\lambda v) = \lambda Tv
    \]

    If $ \lambda = 0 $, then $ STv = \lambda v = 0 $ for some $ v \neq 0 $. Then $ ST $ is not invertible. By previous homework problem 3D.11, $ ST $ not invertible $ \implies $ $ S $ and $ T $ are not both invertible $ \implies $ $ TS $ is not invertible.

    \vspace{1em}
    Therefore, the null space of $ TS $ is not only $ \{0\} $. There exists $ v \neq 0 $ such that $ TS v = 0 = \lambda v $. Therefore, $ TS $ also have 0 as eigenvalue.

    \vspace{1em}
    Thus, $ ST $ and $ TS $ always have the same eigenvalues.

\end{solution}


\newpage
\begin{problem}[5A.25]
    
    Suppose $ T \in L(V) $ and $ u, w $ are eigenvectors of $ T $ such that $ u + w $ is also an eigenvector of $ T $. Prove that $ u $ and $ w $ are eigenvectors of $ T $ corresponding to the same eigenvalue.

\end{problem}

\vspace{2em}
\begin{solution}

    Assume that $ Tu = \lambda_1 u $, $ Tw = \lambda_2 w $, 
    then 
    \[
        \lambda_3 (u + w) = T(u + w) = T(u) + T(w) = \lambda_1 u + \lambda_2 w
    \]
    Rearrange the equation we get
    \[
        (\lambda_3 - \lambda_1) u + (\lambda_3 - \lambda_2)w = 0
    \]

    If $ \lambda_1 \neq \lambda_2 $, then its corresponding eigenvector $ u $ and $ w $ should be linearly independent. Then the equation above should give $ \lambda_1 = \lambda_2 = \lambda_3 $, which is the trivial solution. Therefore we reached a contradiction here. Thus $ \lambda_1 = \lambda_2 $.

\end{solution}


\newpage
\begin{problem}[5A.30]
    
    Suppose $ T \in L(V) $ and $ (T - 2I)(T - 3I)(T - 4I) = 0 $. Suppose $ \lambda $ is an eigenvalue of $ T $. Prove that $ \lambda = 2 $ or $ \lambda = 3 $ or $ \lambda = 4 $.

\end{problem}

\vspace{2em}
\begin{solution}
    
    Let $ \lambda $ be the eigenvalue of $ T $ and let $ v \neq 0 $ be the eigenvector.
    By definition, 
    \[
        Tv = \lambda v \implies Tv - \lambda v = 0 \implies (T - \lambda I) v = 0
    \]
    Therefore,
    \[
        0 = (T - 2I)(T - 3I)(T - 4I)v = (\lambda - 2)(\lambda - 3)(\lambda - 4) = 0
    \]

    This implies that $ \lambda $ can be $ 2, 3, 4 $.

\end{solution}


\end{document}
