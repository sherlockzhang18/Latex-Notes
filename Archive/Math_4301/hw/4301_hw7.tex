\documentclass[a4paper]{article}

% \usepackage[margin=1in]{geometry} // reduce margin

% Basic packages
\usepackage[utf8]{inputenc}
\usepackage[T1]{fontenc}
\usepackage{textcomp}
\usepackage{url}
\usepackage{booktabs}
\usepackage{enumitem}
\usepackage[dvipsnames]{xcolor}
\usepackage{xifthen}

% Math packages
\usepackage{amsmath, amsfonts, mathtools, amsthm, amssymb}
\usepackage{mathrsfs}
\usepackage{cancel}
\usepackage{bm}
\usepackage{systeme}
\usepackage{stmaryrd} % for \lightning

% Math shortcuts
\newcommand\N{\ensuremath{\mathbb{N}}}
\newcommand\R{\ensuremath{\mathbb{R}}}
\newcommand\Z{\ensuremath{\mathbb{Z}}}
\renewcommand\O{\ensuremath{\emptyset}}
\newcommand\Q{\ensuremath{\mathbb{Q}}}
\newcommand\C{\ensuremath{\mathbb{C}}}
\newcommand\F{\ensuremath{\mathbb{F}}}
\DeclareMathOperator{\sgn}{sgn}
\DeclareMathOperator{\Ker}{ker}
\DeclareMathOperator{\im}{Im}

% Logic symbols
\let\svlim\lim\def\lim{\svlim\limits}
\let\implies\Rightarrow
\let\impliedby\Leftarrow
\let\iff\Leftrightarrow
\let\epsilon\varepsilon
\newcommand\contra{\scalebox{1.1}{$\lightning$}}

% Useful commands
\definecolor{correct}{HTML}{009900}
\newcommand\correct[2]{\ensuremath{\:}{\color{red}{#1}}\ensuremath{\to }{\color{correct}{#2}}\ensuremath{\:}}
\newcommand\green[1]{{\color{correct}{#1}}}

% Horizontal rule
\newcommand\hr{
    \noindent\rule[0.5ex]{\linewidth}{0.5pt}
}

% Simple theorem environments (without fancy boxes for homework)
\theoremstyle{definition}
\newtheorem{definition}{Definition}
\newtheorem{theorem}{Theorem}
\newtheorem{lemma}{Lemma}
\newtheorem{proposition}{Proposition}
\newtheorem{corollary}{Corollary}
\newtheorem*{remark}{Remark}
\newtheorem*{note}{Note}
\newtheorem*{example}{Example}

% Problem environment
\newcounter{problem}
\newenvironment{problem}[1][]
{
    \stepcounter{problem}
    \section*{Problem \theproblem\ifx\relax#1\relax\else: #1\fi}
}
{}

% Subproblem environment
\newcounter{subproblem}[problem]
\newenvironment{subproblem}[1][]
{
    \stepcounter{subproblem}
    \subsection*{(\alph{subproblem})\ifx\relax#1\relax\else\ #1\fi}
}
{}

% Solution environment
\newenvironment{solution}
{
    \noindent\textbf{Solution:}\\
}
{
    
}

% Headers
\usepackage{fancyhdr}
\pagestyle{fancy}
\fancyhf{}
\fancyhead[L]{Sherlock Zhang}
\fancyhead[C]{4301 linear algebra - Homework 7}
\fancyhead[R]{\today}
\fancyfoot[C]{\thepage}

% Title info
\title{Math 4301 - Homework 7}
\author{Sherlock Zhang}
\date{\today}

\begin{document}

\maketitle

% =============================================================================
% HOMEWORK PROBLEMS START HERE
% =============================================================================

\begin{problem}[3F.17]

    Suppose $ V $ and $ W $ are finite-dimensional and $ T \in (L, W) $. Prove that $ T $ is invertible if and only if $ T' \in L(W', V') $ is invertible.

\end{problem}

\vspace{2em}
\begin{solution}

    We are using the corollary in class that
    \begin{align*}
        T \text{ is surjective } \iff T' \text{ is injective} \\
        T \text{ is injective } \iff T' \text{ is surjective} \\
    \end{align*}

    $ \implies $

    \vspace{1em}
    If $ T $ is invertible, then it is injective and surjective.
    By the corollary above, $ T' $ is injective and surjective. So it is invertible.

    $ \impliedby $

    \vspace{1em}
    If $ T' $ is invertibel, then it is injective and surjective.
    By the corollary above, $ T $ is also injective and surjective. So it is invertible.
    
\end{solution}

\newpage
\begin{problem}[3F.19]

    Suppose $ U \subseteq V $. Explain why
    \[
        U^0 = \{\varphi \in V': U \subseteq \text{null }\varphi\}
    \]

\end{problem}

\vspace{2em}
\begin{solution}
    
    As $ U \subseteq \text{null }\varphi $, we know that $ \varphi(u) = 0 $ for all $ u \in U $. This is the definition of $ U^0 $.
    \[
        \varphi(u) = 0 \text{ for all } u \in U \iff U \subseteq \text{null }\varphi
    \]

\end{solution}

\newpage
\begin{problem}[3F.22]

   Suppose $ V $ if finite-dimensional and $ U $ and $ W $ are subspaces of $ V $.

\end{problem}

\vspace{1em}
\begin{subproblem}

    Show that $ (U + W)^0 = U^0 \cap W^0 $.

\end{subproblem}

\vspace{1em}
\begin{solution}
    
    Let $ \varphi \in (U + W)^0 $. Then for any $ u \in U $ and $ w \in W $, we know that $ u, w \in U + W $, so $ \varphi(u) = \varphi(w) = 0 $. Thus $ \varphi \in U^0 \cap V^0 $. Therefore $ (U + W)^0 \subseteq U^0 \cap W^0 $.

    \vspace{2em}
    Let $ \varphi \in U^0 \cap W^0 $, then for any $ u \in U $ and $ w \in W $, $ \varphi(u) = \varphi(w) = 0 $. Any vector $ x $ in $ U + W $ can be written as $ au + bw $. $ \varphi(x) = \varphi(au + bv) = a\varphi(u) + b\varphi(v) = a \cdot 0 + b \cdot 0 = 0 $. Therefore $ (U + W)^0 \supseteq U^0 \cap W^0 $

    \vspace{2em}
    Therefore we got
    \[
        (U + W)^0 = U^0 \cap W^0
    \]

\end{solution}

\vspace{2em}
\begin{subproblem}

    Show that $ (U \cap W)^0 = U^0 + W^0 $.

\end{subproblem}

\vspace{1em}
\begin{solution}
    
    $ (U \cap W)^0 \supseteq U^0 + W^0 $:

    \vspace{1em}
    Let $ \varphi = \alpha + \beta $ with $ \alpha \in U^0 $ and $ \beta \in W^0 $. Then, for any $ x \in U \cap W $, $ \alpha(x) = \beta(x) = 0 $. So,
    \[
        \varphi(x) = \alpha(x) + \beta(x) = 0 + 0 = 0
    \]
    Therefore, $ \varphi \in (U \cap W)^0 $, so $ (U \cap W)^0 \supseteq U^0 + W^0 $.

    \vspace{2em}

    Since $ U^0 + W^0 \subseteq (U \cap W)^0 $ and both are finite-dimensional, if we prove $ \dim (U^0 \cup W^0) = \dim U^0 + \dim W^0 $, then $ (U \cap W)^0 = U^0 + W^0 $.

    \begin{align*}
        \dim (U^0 + W^0) 
        &= \dim U^0 + dim W^0 - \dim (U^0 \cap W^0) \\
        &= (\dim V - \dim U) + (\dim V - \dim W) - \dim ((U + W)^0) \\
        &= 2\dim V - \dim U - \dim W - (\dim V - \dim (U + W)) \\
        &= \dim V - (\dim U + \dim W - \dim (U + W)) \\
        & = \dim V - \dim (U \cap W) \\
        &= \dim (U \cap W)^0
    \end{align*}

\end{solution}

\newpage
\begin{problem}[4.1]

    Suppose $ w, z $ in $ \mathbb{C} $. Verify the following equalities and inequalities.
    
\end{problem}

\vspace{1em}
\begin{subproblem}

    $ z + \overline{z} = 2\text{Re} z $
    
\end{subproblem}

\vspace{1em}
\begin{solution}
    
    let $ z = a + bi $, then $ \overline{z} = a - bi $, where $ a, b \in \mathbb{R} $.
    \[
        z + \overline{z} = a + bi + a - bi = 2a = 2 \text{Re}z
    \]

\end{solution}

\vspace{2em}
\begin{subproblem}
    
    $ z - \overline{z} = 2(\text{Im}z)i $

\end{subproblem}

\vspace{1em}
\begin{solution}
    
    let $ z = a + bi $, then $ \overline{z} = a - bi $, where $ a, b \in \mathbb{R} $.
    \[
        z - \overline{z} = a + bi - (a - bi) = 2bi = 2 (\text{Im}z)i
    \]

\end{solution}

\setcounter{subproblem}{4}
\begin{subproblem}
    
    $ \overline{\overline{z}} = z $

\end{subproblem}

\vspace{1em}
\begin{solution}
    
    let $ z = a + bi $, then $ \overline{z} = a - bi $, where $ a, b \in \mathbb{R} $.

    \[
        \overline{\overline{z}} = \overline{a-bi} = a + bi = z
    \]

\end{solution}

\newpage
\begin{problem}[4.4]
    
    Suppose $ m $ is positive integer. Is the set
    \[
        \{0\} \cup \{p \in P(F) : \text{deg }p = m\}
    \]
    a subspace of $ P(F) $.

\end{problem}

\vspace{2em}
\begin{solution}
    
    No it is not as it is not closed under addition. $ x^m $ and $ 1 - x^m $ are both in the set $ A = \{0\} \cup \{p \in P(F) : \text{deg }p = m\} $

    \[
        x^m + 1 - x^m = 1 \notin A
    \]

\end{solution}

\newpage
\begin{problem}[4.7]
    
    Suppose that $ m $ is a nonnegative integer, $ z_1, \cdots ,z_{m + 1} $ are distinct elements of $ F $, and $ w_1, \cdots ,w_{m+1} \in F $. Prove that there exists a unique polynomial $ p \in P_{m(F)} $ such that
    \[
        p(z_{k}) = w_{k}
    \]
    for each $ k = 1, \cdots, m+1 $.

\end{problem}

\vspace{2em}
\begin{solution}
    
    Define the linear map:
    \begin{align*}
    &T: P_{m}(F) \to F^{m+1} \\    
    &T(p) = (p(z_1), p(z_2), \cdots , p(z_{m+1}))
    \end{align*}

    Then, for $ P_{m}(F) $ we choose $ \{1, x, x^2, \cdots , x^n\} $ as the basis, and for $ F^{m+1} $ we choose the standard basis.

    Then the transformation could be written as a matrix:
    \[
        T = \begin{bmatrix}
          1 & z_1 & z_1^2 & \cdots & z_1^m \\
          1 & z_2 & z_2^2 & \cdots & z_2^m \\
          \vdots & \vdots & \vdots & \ddots & \vdots \\
          1 & z_{m+1} & z_{m+1}^2 & \cdots & z_{m+1}^m \\
        \end{bmatrix}
    \]

    This is a Vandermonde matrix, and the determinant of the matrix is 
    \[
        \text{det }T = \prod_{1 \leqslant i \leqslant j \leqslant m+1 } (z_{j} - z_{i})
    \]

    Since $ z_1 ,\cdots , z_{m+1} $ are distinct elements, the determinant is not zero. Therefore the matrix is invertible.

    \vspace{1em}
    By the theorem in class, there exists a unique $ p $ such that
    \[
        T(p) = (w_1, \cdots , w_{m+1})
    \]
    , which is identical to 
    \[
        p(z_{k}) = w_{k}
    \]


\end{solution}

\newpage
\begin{problem}[4.9]
    
    Prove that every ploynomial of odd degree with real coefficients has a real zero.

\end{problem}

\vspace{2em}
\begin{solution}
    
    By the corollary in class, let $ p \in P(\mathbb{R}) $ be a ploynomial with odd degree, its factorization should have the form
    \[
        p(x) = c(x - \lambda_1)\cdots (x - \lambda_{k})(x^2 + r_{1}x + s_{1}) \cdots (x^2 + r_{l} + s_{l})
    \]
    where $ c, \lambda_1, \cdots \lambda_{k} \in \mathbb{R} $ with $ k > 0 $ and $ x^2 + r_{i}x + s_{i} $ has no real zeros for $ 1 \leqslant i \leqslant l $. Therefore, $ \text{deg }p = k + 2l $. Since $ \text{deg }p $ is odd, and $ \text{deg }p = k + 2l $, $ k $ has to be odd. Then it has at least one real zero since $ k > 0 $.
    
\end{solution}

\end{document}
