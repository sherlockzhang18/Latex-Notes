\lesson{14}{Thur Oct 16 2025 14:30}{}

\section{Zeros of Polynomials}
\begin{definition} Polynomials \leavevmode
    
    $ p(x) = a_{n}x^n + \cdots + a_1x + a_0 $. \hspace{2em} deg $ p(x) = n $

    \vspace{1em}
    $ p(x) = 0 $ polynomial \hspace{2em} deg $ p(x) = -\infty $

\end{definition}

\section{Division Algorithm for Polynomials}
\begin{definition} Division algorithm for polynomials \leavevmode

    Suppose $ p(x), s(x) \in P(F) $, $ s \neq 0 $, then, there are unique polynomials $ q(x), r(x) \in P(F) $ such that 
    \[
        p(x) = s(x) q(x) + r(x) \hspace{2em} \text{deg }r(x) < \text{deg }s(x)
    \]

\end{definition}

\vspace{1em}
\begin{eg} \leavevmode
    
    $ p = (x^4) - 2x - 1 $, $ s = 2x^2 - 3 $.

    let $ p, s \in P(\mathbb{R}) $

    \[
        x_4 - 2x -1 (p) = (2x^2 - 3)(s) (\frac{1}{2x^2} + \frac{3}{4})(q) + (-2x + \frac{5}{4})(r)
    \]
    where $ \text{deg }r = 1 < \text{deg }s = 2 $

\end{eg}

\begin{note}
    
    If $ f(x), g(x) \neq 0 $, then
    \[
        \text{deg }f(x)g(x) = \text{deg }f(x) + \text{deg }g(x)
    \]

\end{note}

\begin{proof} \leavevmode

    Let $ n = \text{deg }p(x) $, $ m = \text{deg }s(x) $.

    If $ n < m $, then $ p(x) = s(x) \times 0 + p(x) $, $ \text{deg }p < \text{deg }s $

    \vspace{1em}
    if $ n \geqslant m $, then consider
    \[
        T:P_{n - m}(F) \times P_{m-1}(F) \to P_n(F)
    \]
    \[
        T(q,r) = qs + r
    \]

    \begin{itemize}
        \item $ T $ is a linear map
        \item T is injective if $ qs + r = 0 $ and $ q \neq 0 $, then $ qs $ has degree = $ \text{deg }q + \text{deg }s \geqslant \text{deg }s = m $, but $ r $ has $ \text{deg }r \leqslant m-1 $, so the sum cannot be zero. Contradiction.
        
        So $ q = 0, r = 0 $, so $ T $ is injective.
    \end{itemize}

    \vspace{1em}
    Fundamental theorem of linear map

    \begin{align*}
        \dim P_{n-m}(F) \times P_{m-1}(F) 
        &= 0 + \dim \text{range }T  \\
        &= \dim P_{n-m}(F) + \dim P_{m-1}(F) \\
        &= n-m+1+m \\
        &= n+1
    \end{align*}

    but $ \dim P_{n}(F) = n+1 $ and $ \text{range }T \subseteq P_{n}(F) $, so $ \text{range }T = P_{n}(F) $, so $ T $ is surjective.

    So for every $ p \in P_{n}(F) $, there are $ (q, r) $ such that 
    \[
        p = qs + r
    \]
    uniqueness follows from injectivity of $ T $.
\end{proof}

\section{Factorization of Polynomials over $ \mathbb{C} $}
\vspace{2em}
\begin{theorem} Fundamental Theorem of algebra \leavevmode

    Every nonconstant polynomial with complex coefficients has a zero in $ \mathbb{C} $.

    \vspace{1em}
    (This is not true for real polynomials ) $ p(x) = x^2 + 1 $.
    
\end{theorem}

\vspace{1em}
\begin{corollary}
    
    Every polynomial in $ P(\mathbb{C}) $ factors completely: $ P(x) \in P(\mathbb{C}) $
    \[
        p(x) = c(x - \lambda_1) \cdots (x - \lambda_{n}) \hspace{2em} c, \lambda_1, \cdots , \lambda_{n} \in \mathbb{C}, n = \text{deg }p
    \]

\end{corollary}

\begin{explanation} \leavevmode
    
    If $ \lambda $ is a zero of $ p $. Then divide $ p $ by $ x - \lambda $
    \[
        p(x) = (x - \lambda) q + r \hspace{2em} \text{deg }r \leqslant 0 \implies r \text{ is a constant}
    \]

    \vspace{1em}
    plug in $ x = \lambda $
    \[
        0 = 0 + r \implies r = 0
    \]
    so $ p(x) = (x - \lambda) q(x) \hspace{2em} \text{deg }q = n-1$ .

    Continue with $ q $.

\end{explanation}

\vspace{1em}
\underbar{\textbf{Property of complex conjugate:}}

\begin{itemize}
    \item $ \overline{\overline{\lambda}} = \lambda $
    \item $ \overline{\lambda_1 + \lambda_2} = \overline{\lambda_1} + \overline{\lambda_2} $
    \item $ \overline{\lambda_1\lambda_2} = \overline{\lambda_1}\overline{\lambda_2} $
    \item $ \lambda = \overline{\lambda} \iff \lambda \in \mathbb{R} $.
\end{itemize}

\vspace{1em}
\begin{prop}
    
    If $ p(x) $ has real coefficient and if $ \lambda \in \mathbb{C} $ such that $ p(\lambda) = 0 $, then $ p(\overline{\lambda}) = 0 $, where $ \overline{\lambda}  $ is the complex conjugate of $ \lambda $.

\end{prop}
\vspace{1em}

\begin{proof} \leavevmode

    Suppose $ p(x) = a_{n}x^n + \cdots + a_1x + a_0 $, $ a_{i} \in \mathbb{R}, \lambda \in \mathbb{C} $. If $ p(\lambda) = 0 $, then

    \begin{align*}
        &a_{n}\lambda^n + \cdots + a_1\lambda + a_0 = 0 \\
        \implies &\overline{a_{n}\lambda^n + \cdots + a_1\lambda + a_0} = 0 \\
        \implies &\overline{a_{n}}\overline{\lambda}^n + \cdots + \overline{a_1}\overline{\lambda} + \overline{a_0} = 0 \\
        \implies &a_{n}\overline{\lambda}^n + \cdots + a_1\overline{\lambda} + a_0 = 0 \\
        \implies &p(\overline{\lambda}) = 0
    \end{align*}

\end{proof}