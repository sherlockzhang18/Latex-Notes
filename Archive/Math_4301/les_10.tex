\lesson{10}{Thur Sep 25 2025 14:30}{}

\begin{definition} \leavevmode
    
    \begin{itemize}
        \item An isomorphism is an invertible linear map.
        \item $ V $ and $ W $ are isomorphic if there is an isomorphism $ T: V \to W $.
    \end{itemize}
    
\end{definition}

\vspace{2em}
\begin{eg}
    
    $ V = \{(x,y,z) | x + 2y - z = 0 \} \subset \mathbb{R}^3 $. $ V $ is a subspace of $ \mathbb{R}^3 $.

    Then $ V $ and $ \mathbb{R}^2 $ are isomorphic 
    
    $ T: V \to \mathbb{R}^2 $, $ T(x, y, z) = (x, y) $.

    \vspace{2em}
    injective: if $ T(x, y, z) = 0 $, then $ x = y = 0 $, so $ z - x + 2y = 0 $. So $ \text{null}T = \{0\} $.

    \vspace{1em}
    surjective: For any $ (x, y) \in \mathbb{R}^2 $, $ T(x, y, x+2y) = (x, y) $

    \vspace{2em}
    $ T^{-1}: \mathbb{R}^2 \to V $, $ T^{-1}(x,y) = (x, y, x+2y) $.

\end{eg}

\vspace{2em}
\begin{theorem}
    
    If $ V $ and $ W $ are finite-dimensional, then $ V $ and $ W $ are isomorphic if and only if $ \dim V = \dim W $.

\end{theorem}

\begin{proof}

    Suppose $ T: V \to W $ is invertible, so $ T $ is injective and surjective. Then,
    \[
        \dim \text{null}T ( = 0) + \dim \text{range} T ( = \dim W ) = \dim V \implies \dim W = \dim V 
    \]

    \vspace{2em}
    Conversely if $ \dim W = \dim V = n $,

    Let $ v_1, \cdots , v_{n} $ be a basis for $ V $, and $ w_1, \cdots , w_{n} $ be a basis for $ W $. Then there is $ T \in L(V, W) $ with $ Tv_{i} = w_{i} $

    \begin{itemize}
        \item $ T $ is injective
        
        If $ v \in \text{null}T $, then $ v = a_1v_1 + \cdots + a_{n}v_{n} $, so 
        \[
            0 = Tv = T(a_1v_1 + \cdots + a_{n}v_{n}) = a_1w_1 + \cdots + a_{n}w_{n}
        \]
        So $ a_1 = a_2 = \cdots =a_{n} = 0 $.


        \item $ T $ is surjective
        
        $ w \in W $, then $ w = b_1w_1 + \cdots + b_{n}w_{n} $. Then
        \begin{align*}
            T(b_1v_1 + \cdots + b_{n}v_{n}) 
            &= b_1Tw_1 + \cdots + b_{n}Tw_{n} \\
            &= b_1w_1 + \cdots + b_{n} w_{n} \\
            &= w \\
        \end{align*}
    \end{itemize}

\end{proof}

\begin{eg}
    $ P_{m}(\mathbb{R}) $ and $ \mathbb{R}^{m+1} $ are isomorphic: $ \dim P_{m}(\mathbb{R}) = \dim \mathbb{R}^m+1 = m+1 $.

    \vspace{1em}
    $ T: (a_{m}x^m + \cdots + a_1x + a_0) = (a_{m}, \cdots , a_0) $.
\end{eg}

\begin{eg}
    
    Suppose $ \dim V = n $, $ \dim W = m $. Find a basis $ v_1, \cdots , v_{n} $ for $ V $ and $ w_1, \cdots , w_{m} $ for $ W $. Then $ L(V, W) \xrightarrow{\phi} F^{m,n} $
    \[
        \phi(T) = U(T)
    \]

    Then $ \phi $ is linear.

    \vspace{1em}
    Claim: $ \phi $ is an isomorphism.
    \begin{itemize}
        \item $ \phi $ is injective: if $ U(T) = 0 $ matrix, then $ Tv_{i} = 0 $ for all $ i $. So $ T = 0 $.
        \item $ \phi $ is surjective: If $ A = (A_{ij}) $, there is a unique $ T \in L(V, W) $ such that 
        
        \[
            Tv_{j} = A_{1,j}w_1 + \cdots + A_{m,j}w_{m} \hspace{2em} j = 1, \cdots , n
        \]

        Then $ U(T) = A $, so $ \phi(T) = A $.
    \end{itemize}

\end{eg}

\vspace{2em}
\begin{corollary}
    
    \begin{align*}
        \dim L(V,W) 
        &= \dim F^{m,n}\\
        &= mn \\
        &= (\dim V)(\dim W) \\
    \end{align*}

\end{corollary}

\begin{theorem} \leavevmode
    
    Suppose $ V, W $ are finite-dimensional and $ \dim V = \dim W $, $ T \in L(V, W) $.

    Then, $ T $ is injective $ \iff $ $ T $ is invertible $ \iff T $ is surjective.

\end{theorem}

\begin{proof} \leavevmode
    
    \begin{enumerate}
        \item $ \implies $ if $ T $ injective, then $ \dim \text{null}T = \{0\} $, and $ \dim \text{null}T + \dim \text{range}T = \dim V = \dim W $.
        
        This implies $ \text{range}T = W \implies T $ is surjective.

        \item $ \impliedby $ If $ T $ is surjecive, then $ \text{range}T = W $ , so $ \dim \text{range}T = \dim W = \dim V $
        \[
            \dim \text{null}T + \dim \text{range}T = \dim V \implies \text{null}T = \{0\}
        \]
        So $ T $ is surjective.
    \end{enumerate}

\end{proof}

\begin{note}
    
    In particular if $ V = W: T \in L(V,V) $ and $ V $ is finite-dimensional

    \[
        T \text{ surjective } \iff T \text{ injective } \iff T \text{ invertible }
    \]

    \vspace{1em}
    Last time: $ T: F^\infty \to F^\infty $

    $ T(a_1, a_2, \cdots ) = (0, a_1, a_2, \cdots ) $

    \vspace{1em}
    injective but not surjective.
\end{note}

Another version of this theorem: 
\begin{theorem}
    \begin{tikzcd}
        V \arrow[r, bend left=20, "T"]
        &W \arrow[l, bend left=20, "S"]
    \end{tikzcd}

    Suppose $ \dim V = \dim W $, then
    \[
        TS = I_{W} \iff ST = I_{V}
    \]
\end{theorem}

\begin{proof}

    the condition $ TS = I_{W} $ alone shows that $ S $ is surjective because if $ Sw = 0 $, the $ w = TSw = T \cdot 0 = 0 $

    Using the previous theorem, $ S $ is invertible, there exists $ S^{-1 }$.

    Claim $ T = S^{-1} $: $ TS = I_{W} \implies TSS^{-1} = I_{W}S^{-1} = S^{-1} $
    
    $ \implies T = S^{-1} $

\end{proof}

\section{Product and Quotient of Vector Spaces}

\begin{definition} Product of Vector Spaces \leavevmode

    $ V_1, \cdots , V_{m} $ vector spaces over $ F $. 

    \begin{itemize}
        \item Product is defined as:
        \[
            V_1 \times \cdots \times V_{m} = \{(u_1, u_2, \cdots , u_{m}) | u_{i} \in V_{i}, i = 1, \cdots , m \}
        \]

        \item Addition is defined as
        \[
            (u_1, \cdots , u_{m}) + (u_1', \cdots , u_{m}') = (u_1+ u_1', \cdots , u_{m}+ u_{m}')
        \]

        \item Scalar multiplication:
        \[
            a(u_1, \cdots , u_{m}) = (au_1, \cdots , au_{m})
        \]
    \end{itemize}

\end{definition}

\vspace{2em}
\begin{theorem}
    $ V_1 \times \cdots \times V_{m} $ is a vector space. 

    \vspace{2em}
    Basis for $ V_1 \times \cdots \times V_{m} $:

    Pick a basis $ u_{i,1}, u_{i,2}, \cdots , u_{i,n_{i}} $ for $ V_{i} $ (so $ \dim V_{i} = n_{i} $).

    Then the vectors:
    \[
        w_{i,j} = (0,\cdots ,0, \cdots ,u_{i,j}, \cdots 0) \hspace{2em} (1\leqslant j \leqslant n_{i})
    \]

    Then $ w_{i,j} (1\leqslant i \leqslant m, 1\leqslant j \leqslant n_{i})$ form a basis for $ V_1 \times \cdots \times V_{m} $. 

    So 
    \[
        \dim  V_1 \times \cdots \times V_{m}  = \dim V_1 + \cdots + \dim V_{m}
    \]
\end{theorem}

