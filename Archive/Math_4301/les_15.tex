\lesson{15}{Tue Oct 21 2025 14:30}{}

\section{Factorization of Polynomials over $ \mathbb{R} $}
\begin{corollary} \leavevmode
    
    Let $ p(x) \in P(\mathbb{R}) $, then by corollary above,
    \[
        p(x) = c(x - \lambda_1) \cdots (x-\lambda_{k})(x - \lambda_{k+1}) \cdots (x - \lambda_{m})
    \],
    where $ c, \lambda_1, \cdots , \lambda_{k} \in \mathbb{R} $, $ \lambda_{k + 1}, \cdots, \lambda_{m} \in \mathbb{C}  $

    \vspace{1em}
    For each $ k + 1 \leqslant j \leqslant m $, $ \overline{\lambda_{j}} $is also a zero of $ p(x) $.

\end{corollary}

\vspace{2em}
\underbar{\textbf{What is $ (x - \lambda_{j})(x - \overline{\lambda_{j}}) $}}

Let $ \lambda_{j} = b + ci, \overline{\lambda_{j}} = b-ci $, then
\begin{align*}
    (x - \lambda_{j})(x - \overline{\lambda_{j}})
    &= (x - (b + ci)) (x - (b - ci)) \\
    &= x^2 - 2bx + b^2 + c^2 \\
\end{align*}

\begin{note}
    
    If $ p(x) \in P(\mathbb{R}) $, then
    \[
        p(x) = c(x - \lambda_1) \cdots (x - \lambda_{k})(x^2 + r_{k+1}x + s_{k+1}) \cdots (x^2 + r_{l} + s_{l})
    \]
    such that $ c, \lambda_1, \cdots , \lambda_{k}, r_{j}, s_{j} \in \mathbb{R} $, $ k + 1 \leqslant j \leqslant l $, and $ x^2 + r_{i} + s_{i} $ with $ k + 1 \leqslant i \leqslant l $ has no real zero.

\end{note}

\vspace{2em}
\begin{theorem} \leavevmode
    
    if $ p(x) \in P(F) $ is a polynomial of degree $ n $, then $ p(x) $ has at most $ n $ zeros in $ F $.

\end{theorem}

\vspace{1em}
\begin{proof} \leavevmode
        
    Induction on $ n $ and division algorithm:

    \vspace{1em}
    if $ \lambda \in F $ is a zero of $ p $, then $ p(x) = (x - \lambda) q(x) $, where $ \text{deg }q = n - 1 $. But $ q $ itself by induction has only at most $ n - 1 $ zeros, so $ p $ has at most $ n $ zeros.

\end{proof}

\chapter{Eigenvalues and Eigenvectors}
\section{Invariant Subspaces}

\begin{definition}
    
    $ T: V \to V $ a linear map, i.e $ L(V,V) $ or $ L(V) $, is called an operator.
 
\end{definition}

\begin{definition}
    
    A subspace $ U \subseteq V $ is invariant under $ T $ if 
    \[
        Tu \in U \text{ for } u \in U
    \]

\end{definition}

\vspace{1em}
\begin{eg} \leavevmode
    
    $ T: \mathbb{R}^2 \to \mathbb{R}^2 $, $ T(x, y) = (x, 0) $.

    \vspace{1em}
    Then invariant subspaces are:
    \begin{itemize}
        \item $ \mathbb{R}^2 $
        \item \{0\}
        \item the y-axis and x-axis. The other lines through origin are \underbar{\textbf{NOT}} invariant under this $ T $.
    \end{itemize}


    \vspace{2em}
    This shows in general
    \begin{itemize}
        \item \{0\}
        \item $ V $
        \item $ \text{null }T $
        \item $ \text{range }T $
    \end{itemize}
    is invariant subspaces

    \vspace{2em}
    $ D: P(F) \to P(F) $, $ D(p) = p' $
    , then 
    \begin{itemize}
        \item $ P_m(F) $ is invariant under $ D $.
    \end{itemize}

\end{eg}

\vspace{2em}
Suppose $ T \in L(V) $, suppose $ U $ is a 1-dimensional invariant subspace. Pick any non-zero vector $ v \in U $, so $ U = \text{span }(v) $. Then, $ Tv \in U \implies Tv \in \text{span }(v)$, so $ Tv = \lambda v $ for some $ \lambda \in F $.

\vspace{1em}
\begin{definition} Eigenvalue \leavevmode
    
    $ T \in L(V), \lambda \in F $ is called an \underbar{eigenvalue} for $ T $ if there is a non-zero vector $ v \in V $ with
    \[
        Tv = \lambda v
    \]

\end{definition}

\vspace{1em}
\begin{eg} \leavevmode
    
    \begin{enumerate}
        \item 
            Let $ T \in L(\mathbb{R}^2), T(x, y) = (-y, x) $, find all eigenvalues of $ T $.

            \vspace{1em}
            \underbar{\textbf{Solution:}}
            There is no eigenvalue.

            \begin{explanation}
                
                $ T(x,y) = \lambda(x,y), (x,y) \neq (0,0) $

                \vspace{1em}
                $ \implies (-y,x) = (\lambda x, \lambda y) \implies
                \begin{cases}
                    -y = \lambda x \\
                    x = \lambda y
                \end{cases}
                 \implies -y = \lambda^2 y
                $

                There is no real $ \lambda $ that can satify this condition. So there is no eigenvalue.

            \end{explanation}

        \vspace{1em}    
        \item 
            Let $ T \in L(\mathbb{R}^2), T(x, y) = (x, 0) $, find all eigenvalues of $ T $.
            
            \vspace{1em}
                \underbar{\textbf{Solution:}}
                

                $ T(x, y) = \lambda(x, y), (x, y) \neq (0,0) $
                
                $ \implies (x,0) = (\lambda x, \lambda y) $

                $ \implies x = \lambda x, 0 = \lambda y $

                $ \implies x = 0 \text{ or } \lambda = 1 $, $ y = 0 \text{ or } \lambda = 0 $

                \begin{itemize}
                    \item If $ x = 0 $, then $ y \neq 0 $, so $ \lambda = 0 $: $ T(0,y) = (0)(0,y) $
                    \item if $ \lambda = 1 $, then $ y = 0 $: $ T(x, 0) = 1(x, 0) $
                \end{itemize}

        \vspace{1em}
        \item 
            $ T \in L(\mathbb{C}^2), T(z, w) = (-w, z) $ 

            $ \lambda \in \mathbb{C} $, $ T(z, w) = \lambda(z, w) $ with $ (z, w) \neq (0,0) $

            $ \implies (-w, z) = (\lambda z, \lambda w) $

            $ \implies 
            \begin{cases}
                -w = \lambda z \\
                z = \lambda w
            \end{cases} 
            \implies  -w = \lambda^2w \implies w(1 + \lambda^2) = 0 \implies w = 0 \text{ or } \lambda = \pm i
            $

            If $ w = 0 $, then $ z = \lambda w = 0 $, not allowed.

            If $ \lambda = \pm i $, plug in and test it is correct.

    \end{enumerate}

\end{eg}