% lec_03.tex
\lesson{3}{Wed Apr 17 2024 10:00}{Elliptic ElGamal \& MV-ElGamal}

\section{Review: ECDH Key Exchange}
Last time we saw Elliptic Curve Diffie–Hellman (ECDH):
\[
	\text{Public: }(p,E,P),\quad
	Q_A = n_A P,\; Q_B = n_B P,\quad
	n_AQ_B = n_BQ_A = n_An_BP
\]
which yields a shared secret (e.g.\ the \(x\)-coordinate of \(n_An_BP\)).

\section{Classical ElGamal over \(\F_p\)}
\begin{itemize}
	\item Public: \((p,g)\) in \(\F_p^*\).
	\item Alice’s private key \(a\), public key \(A = g^a\pmod p\).
	\item Bob picks random \(k\), encrypts message \(m\) as
	      \[
	      	(c_1,c_2)=(g^k,\;m\cdot A^k)\pmod p.
	      \]
	\item Alice recovers
	      \(\displaystyle
	      m = c_2\cdot c_1^{-a}\pmod p.
	      \)
\end{itemize}

\section{Elliptic ElGamal}
\begin{itemize}
	\item Public: \((p,E,P)\).
	\item Alice's private key \(n_A\), public key \(Q_A=n_AP\).
	\item Bob picks random \(k\), message \(M\in E(\F_p)\).  He sends
	      \[
	      	(c_1,c_2)=(kP,\;M + k\,Q_A).
	      \]
	\item Alice computes
	      \[
	      	c_2 - n_A\,c_1 = (M + kQ_A) - n_A(kP) = M.
	      \]
\end{itemize}

\subsection{Menezes-Vanstone (MV) Variant}
Allows numeric messages \(m_1,m_2\in\F_p\):
\begin{itemize}
	\item Public: \((p,E,P)\), Alice: private \(n_A\), public \(Q_A=n_AP\).
	\item Bob picks random \(k\), sets
	      \[
	      	C = kP,\quad S = kQ_A = (x_s,y_s).
	      \]
	\item Encrypt: 
	      \[
	      	c_1 = m_1\,x_s\pmod p,\quad c_2 = m_2\,y_s\pmod p,
	      \]
	      send \((C,c_1,c_2)\).
	\item Alice recovers
	      \[
	      	T = n_A\,C = (x_T,y_T),
	      	\quad
	      	m_1 = x_T^{-1}c_1,\; m_2 = y_T^{-1}c_2 \pmod p.
	      \]
\end{itemize}

\begin{eg}
	Let \(p=13\), \(E:y^2=x^3+x+1\), \(P=(5,1)\).  Alice picks \(n_A=7\):
	\[
		Q_A=7P=(1,9).
	\]
	Bob has \(m_1=3,m_2=8\), picks \(k=10\):
	\[
		C=10P=(11,2),\quad S=10Q_A=(4,2).
	\]
	He computes
	\[
		c_1=3\cdot4=12,\quad c_2=8\cdot2=3\pmod{13}
	\]
	and sends \(\bigl((11,2),12,3\bigr)\).  Alice gets
	\[
		T=7C=(4,0),\quad
		m_1 = 4^{-1}\cdot12=3,\quad
		m_2 = 0^{-1}\cdot3=8\pmod{13}.
	\]
\end{eg}

\section{Point Compression}
Instead of sending \((x_c,y_c)\), send \((x_c,\beta)\) where
\[
	\beta=\begin{cases}
	0,&0\le y_c<\tfrac p2,\\
	1,&\tfrac p2<y_c<p.
	\end{cases}
\]
Alice recovers \(y_c\) by solving \(y^2=x_c^3+ax_c+b\pmod p\)
and choosing the root consistent with \(\beta\).

\section{Solving \(y^2=a\pmod p\)}
\begin{lemma}[Square roots mod \(p\)]
	If \(p\equiv3\pmod4\) and \(\bigl(\tfrac ap\bigr)=1\), then
	\[
		y\;\equiv\;\pm\,a^{\tfrac{p+1}4}\pmod p
	\]
	are the two solutions to \(y^2\equiv a\pmod p\).
\end{lemma}
\begin{proof}
	\[
		\bigl(a^{\tfrac{p+1}4}\bigr)^2
		=a^{\tfrac{p+1}2}
		=a^{\tfrac{p-1}2}a
		\equiv (1)\,a
		=a\pmod p.
	\]
\end{proof}