% lec_04.tex
\lesson{4}{Mon Apr 22 2024 10:00}{Lenstra's Elliptic Curve Factorization}

\section{Elliptic Curves over \(\Z/N\Z\)}
\begin{definition}
	Let \(N>2\) be odd.  An \emph{elliptic curve} over \(\Z/N\Z\) is the set of solutions \((x,y)\) to
	\[
		y^2 \equiv x^3 + a x + b \pmod N,
	\]
	with \(\gcd(4a^3+27b^2,\,N)=1\), together with the point at infinity \(\mathcal{O}\).  We denote this group by \(E(\Z/N\Z)\).
\end{definition}

\begin{eg}
	On 
	\[
		y^2 = x^3 + 3x + 7,\quad N=187=11\cdot17,
	\]
	take \(P=(38,112)\).  To compute \(2P\) we use the tangent slope
	\[
		m=\frac{3x^2+3}{2y}\bmod 187
		=\frac{3\cdot38^2+3}{2\cdot112}
		=\frac{4335}{224}
		=\frac{34}{37}
		=34\cdot37^{-1}\equiv102\pmod{187}.
	\]
	The line is \(y-112=102(x-38)\), substitute back into the curve, and find
	\[
		2P=(43,126).
	\]
	Continuing,
	\[
		3P=(54,105),\quad4P=3P+P,\quad5P=3P+2P.
	\]
	But in the formula for \(5P\) the denominator \((x_2-x_1)\equiv-11\pmod{187}\) shares \(\gcd(11,187)=11>1\).  Hence \(5P\) “does not exist” in \(E(\Z/187\Z)\), and
	\(\gcd(21,-11)=11\)
	reveals the factor \(11\).
\end{eg}

\section{Algorithm (v1)}
Choose
\[
	L = k!\quad(k=1,2,3,\dots)
\]
and attempt to compute \(L\cdot P\) on \(E(\Z/N\Z)\).  Three outcomes:
\begin{itemize}
	\item You obtain a valid point—proceed to \((k+1)!\,P\).
	\item A denominator during addition has \(\gcd(d,N)>1\); that gcd is a nontrivial factor of \(N\).
	\item A denominator is zero (vertical line), yielding \(\mathcal O\); restart with a new curve or base point.
\end{itemize}

\section{Chinese Remainder Perspective}
Under
\[
	E(\Z/187\Z)\cong E(\Z/11\Z)\times E(\Z/17\Z),
\]
our base point 
\[
	P=(38,112)\mapsto (5,2)\bmod11,\;(4,10)\bmod17.
\]
We find \(5P\equiv\mathcal O\pmod{11}\) but \(5P\not\equiv\mathcal O\pmod{17}\), so the slope-denominator failure pinpoints the factor \(11\).

\section{Worked Example}
\begin{eg}
	Let \(N=6887\), curve \(y^2=x^3+14x+19\), and \(P=(1512,3116)\).  Compute
	\[
		1!\,P,\;2!\,P=2P,\;3!\,P=3\cdot2P,\;\dots,\;7!\,P.
	\]
	At \(7!\,P\) the slope-denominator is \(781\), and \(\gcd(781,6887)=71\).  Thus \(71\mid N\), and \(6887=71\cdot97\).
\end{eg}

\section{Choosing the Curve}
Instead of testing if \(x^3+ax+b\) is a square mod \(N\), pick
\((P,a)\) and solve for \(b\).  For example, \(N=35\), \(P=(2,6)\), \(a=3\):
\[
	6^2\equiv2^3+3\cdot2+b\pmod{35}
	\implies 36\equiv8+6+b\pmod{35}
	\implies b\equiv22\pmod{35},
\]
so use \(y^2=x^3+3x+22\) on \(\Z/35\Z\).