\lesson{25}{Mon Nov 3 2025 15:00}{}

\setcounter{chapter}{4}
\chapter{Eigenvalues and eigenvectors of a symmetric matrix}
\setcounter{section}{4}
\section{Householder's method}

\setcounter{lemma}{2}
\begin{lemma} \leavevmode
    
    $
    H^{(2k)} = 
    \begin{bmatrix}
      I_{k} &  \\
       & H_{n-k} \\
    \end{bmatrix} 
    \hspace{2em}
    \begin{cases}
        I_{k} \in \mathbb{R}^{k \times k} \text{ identity} \\
        H_{n-k} \in \mathbb{R}^{(n-k) \times (n-k)} \text{ Householder matrix} \\
    \end{cases}
    $
    is also a Householder matrix

\end{lemma}

\begin{proof}
    to be filled
\end{proof}

\vspace{2em}
\begin{lemma}
    
    Given $ x \in \mathbb{R}^n $ non-zero, there exists a Householder matrix such that 
    \[
        Hx = \alpha e_1, \hspace{1em} e_1 = \begin{bmatrix}
          1 \\
          0 \\
          \vdots \\
          0 \\
        \end{bmatrix} \in \mathbb{R}^n \text{ standard basis}
    \]

    \[
        H \begin{bmatrix}
          * \\
          * \\
          \vdots \\
          * \\
        \end{bmatrix}
        = \begin{bmatrix}
          \alpha \\
          0 \\
          \vdots \\
          0 \\
        \end{bmatrix}
    \]

\end{lemma}

\begin{proof}
    to be filled.
\end{proof}

\begin{eg}
    
    Let $ A \in R^{n \times n} _{\text{sym}} $.

    \[
        \begin{bmatrix}
          1 &  \\
           & H \\
        \end{bmatrix}
        \cdot
        \begin{bmatrix}
          * & * & * & * & * \\
          * & * & * & * & * \\
          * & * & * & * & * \\
          * & * & * & * & * \\
          * & * & * & * & * \\
        \end{bmatrix}
         = 
         \begin{bmatrix}
          * & * & * & * & * \\
          \alpha & * & * & * & * \\
          0 & * & * & * & * \\
          0 & * & * & * & * \\
          0 & * & * & * & * \\
        \end{bmatrix}
    \]
    then next step is 
    \[
        \begin{bmatrix}
          1 &  \\
           & H \\
        \end{bmatrix}
        \cdot
        \begin{bmatrix}
          * & * & * & * & * \\
          * & * & * & * & * \\
          * & * & * & * & * \\
          * & * & * & * & * \\
          * & * & * & * & * \\
        \end{bmatrix}
        \cdot
        \begin{bmatrix}
          1 &  \\
           & H \\
        \end{bmatrix}
        = 
        \begin{bmatrix}
          * & \alpha & 0 & 0 & 0 \\
          \alpha & * & * & * & * \\
          0 & * & * & * & * \\
          0 & * & * & * & * \\
          0 & * & * & * & * \\
        \end{bmatrix}
    \]
    and continue with $ H^{(2)} = \begin{bmatrix}
      I_2 & 0 \\
      0 & H_{(n-2)} \\
    \end{bmatrix} $

\end{eg}

\vspace{2em}
\setcounter{theorem}{6}
\begin{theorem} \leavevmode
    
    Any $ A \in \mathbb{R}^{n\ \times n}_\text{sym} $ is orthogonally similar to a tri-diagonal matrix $ T $, 
    \[
        A = QTQ^T \text{ where } Q \text{ orthogonal}
    \]
    $ T $ is tri-diagonal $\iff [T]_{ij} = 0  $ if $ |i - j| > 1 $

\end{theorem}

$ Q = H^{(n-2)} H^{(n-1)} \cdots H^{(1)} $ where $ H^{(i)} $ are householder matrices.

\begin{proof}
    
    \[
        A^{(k)} = \begin{bmatrix}
          B^{(k)} &  & C^{(k)T} \\
           &  & c^{(k)T} \\
          C^{(k)} & c^{(k)} & D^{(k)T} \\
        \end{bmatrix}
    \]
    where $ B^{(k)} \in \mathbb{R}^{k \times k}, C^{(k)} \in \mathbb{R}^{(n - k ) \times (k - 1)}, c^{(k)} \in \mathbb{R}^{(n - k) \times 1}, D^{(k)} \in \mathbb{R}^{(n - k) \times (n - k)} $.

    \[
        \text{Householder: } Q^{(k)} = \begin{bmatrix}
          I_{k} & 0 \\
          0 & H^{(k)} \\
        \end{bmatrix}
    \]
    where $ H^{(k)} = I_{n-k} - 2\frac{v^{(k)} \cdot v^{(k)T}}{v^{(k)T} \cdot v^{(k)}} $

    Then \[
        A^{(k+1)} = Q^{(k)} A^{(k)} Q{(k)^T}
    \]

    \underbar{\textbf{Goal}}: Pick $ v^{(k)} $ such that $ B^{(k)} $ is tri-diagonal.

    \vspace{2em}
    Proof by induction:

    \vspace{1em}
    At $ k = 1 $, set $ A^{(1)} = A $ (the given matrix).

    Assume $ B^{(k)} $ is tri-diagonal. \underbar{Want} $ B^{(k+1)} $ is tri-diagonal.

    \vspace{1em}
    Set $ v^{(k)} $ so that $ H^{(k)}c^{(k)} = \alpha^{(k)}e_1^{(k)} $, $ e_1^{(k)} $ as 1st standard basis in $ \mathbb{R}^{(n - k) \times 1} $.
    \[
        Q^{(k)} A^{(k)} Q{(k)^T}
        =
        \begin{bmatrix}
          I_{k} & 0 \\
          0 & H^{(k)} \\
        \end{bmatrix}
        \cdot
        \begin{bmatrix}
          B^{(k)} &  & C^{(k)T} \\
           &  & c^{(k)T} \\
          C^{(k)} & c^{(k)} & D^{(k)T} \\
        \end{bmatrix}
        \cdot 
        \begin{bmatrix}
          I_{k} & 0 \\
          0 & H^{(k)} \\
        \end{bmatrix}
        = 
        \text{ pic to be filled}
    \]

\end{proof}