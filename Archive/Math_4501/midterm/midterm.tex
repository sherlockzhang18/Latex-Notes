\documentclass[a4paper]{article}

\usepackage[margin=1in]{geometry} % reduce margin

% Basic packages
\usepackage[utf8]{inputenc}
\usepackage[T1]{fontenc}
\usepackage{textcomp}
\usepackage{url}
\usepackage{booktabs}
\usepackage{enumitem}
\usepackage[dvipsnames]{xcolor}
\usepackage{xifthen}

% Math packages
\usepackage{amsmath, amsfonts, mathtools, amsthm, amssymb}
\usepackage{mathrsfs}
\usepackage{cancel}
\usepackage{bm}
\usepackage{systeme}
\usepackage{stmaryrd} % for \lightning

% Math shortcuts
\newcommand\N{\ensuremath{\mathbb{N}}}
\newcommand\R{\ensuremath{\mathbb{R}}}
\newcommand\Z{\ensuremath{\mathbb{Z}}}
\renewcommand\O{\ensuremath{\emptyset}}
\newcommand\Q{\ensuremath{\mathbb{Q}}}
\newcommand\C{\ensuremath{\mathbb{C}}}
\newcommand\F{\ensuremath{\mathbb{F}}}
\DeclareMathOperator{\sgn}{sgn}
\DeclareMathOperator{\Ker}{ker}
\DeclareMathOperator{\im}{Im}

% Logic symbols
\let\svlim\lim\def\lim{\svlim\limits}
\let\implies\Rightarrow
\let\impliedby\Leftarrow
\let\iff\Leftrightarrow
\let\epsilon\varepsilon
\newcommand\contra{\scalebox{1.1}{$\lightning$}}

% Useful commands
\definecolor{correct}{HTML}{009900}
\newcommand\correct[2]{\ensuremath{\:}{\color{red}{#1}}\ensuremath{\to }{\color{correct}{#2}}\ensuremath{\:}}
\newcommand\green[1]{{\color{correct}{#1}}}

% Horizontal rule
\newcommand\hr{
    \noindent\rule[0.5ex]{\linewidth}{0.5pt}
}

% Simple theorem environments (without fancy boxes for homework)
\theoremstyle{definition}
\newtheorem{definition}{Definition}
\newtheorem{theorem}{Theorem}
\newtheorem{lemma}{Lemma}
\newtheorem{proposition}{Proposition}
\newtheorem{corollary}{Corollary}
\newtheorem*{remark}{Remark}
\newtheorem*{note}{Note}
\newtheorem*{example}{Example}

% Problem environment
\newcounter{problem}
\newenvironment{problem}[1][]
{
    \stepcounter{problem}
    \section*{Problem \theproblem\ifx\relax#1\relax\else: #1\fi}
}
{}

% Subproblem environment
\newcounter{subproblem}[problem]
\newenvironment{subproblem}[1][]
{
    \stepcounter{subproblem}
    \subsection*{(\alph{subproblem})\ifx\relax#1\relax\else\ #1\fi}
}
{}

% Solution environment
\newenvironment{solution}
{
    \noindent\textbf{Solution:}\\
}
{
    
}

% Headers
\usepackage{fancyhdr}
\pagestyle{fancy}
\fancyhf{}
\fancyhead[L]{Sherlock Zhang}
\fancyhead[C]{4501 - Midterm 1}
\fancyhead[R]{\today}
\fancyfoot[C]{\thepage}

% Title info
\title{Math 4501 - Midterm 1}
\author{Sherlock Zhang}
\date{\today}

\begin{document}

\maketitle

% =============================================================================
% HOMEWORK PROBLEMS START HERE
% =============================================================================

\begin{problem}
    Let $ f: \mathbb{R} \to \mathbb{R} $ be a function given by
    \[
        f(x) =
        \begin{cases}
            \frac{1}{2}x^2 &\text{if } x \geqslant 0 \\
            0 &\text{if } x < 0
        \end{cases}
    \]

    Show that $ f $ is a contraction in an interval containing $ x = 0 $, and specify a specific interval.
\end{problem}

\vspace{1em}
\begin{solution}

    We want to first prove that $ f'(x) $ is continuous on an interval containing 0, then use mean value theorem to prove that it is a contraction.

    \[
        f'(x) = 
        \begin{cases}
            x, &x \geqslant 0 \\
            0, &x < 0
        \end{cases}
    \]

    The function is continuous on $ x > 0 $ and $ x < 0 $. Checking continuity at $ x = 0 $:
    \[
        \lim_{x \to 0^{-}} f'(x) = 0 \hspace{2em} \lim_{x \to 0^{+}} f'(x) = 0
    \]

    So $ f' $ is continuous at 0.

    \vspace{2em}
    Let $ [-r, r] $ be a closed arbitrary interval containing 0. Then by Mean Value Theorem, for any $ x, y \in [-r, r] $,
    \[
        |f(x) - f(y)| \leqslant f'(\eta) |x - y| \leqslant \sup_{c \in [-r, r]}f'(c) |x - y|
    \]

    So if we choose $ r $ so that $ |f'(x) < 1|, x \in [-r, r] $, then $ f $ is a contraction.

    \vspace{2em}
    Choose $ r $ to be $ \frac{1}{2} $, then in the interval $ [-\frac{1}{2}, \frac{1}{2}] $, $ f'(x) \leqslant \frac{1}{2} $
    \[
        |f(x) - f(y)| \leqslant \sup_{c \in [-r, r]}f'(c) = \frac{1}{2} |x - y|
    \]
    which proves that $ f $ is a contraction on $ [-\frac{1}{2}, \frac{1}{2}] $.

\end{solution}

\newpage
\setcounter{problem}{2}
\begin{problem}
    
    We want to find $ \xi $ at which the following function vanishes
    \[
        f(x) = \pi - 20\arctan\left(\frac{1}{x}\right) - 8 \arctan \left(\frac{3}{79}\right)
    \]
    Suppose we know that the solution $ \xi $ likes in the interval $ [5, 10] $.

\end{problem}

\begin{subproblem}

    Consider the relaxation iteration:
    \[
        x_{k+1} = x_{k} - \lambda f(x_{k}).
    \]
    Find a relaxation parameter $ \lambda $ for which one can show that the iteration converges if $ x_0 $ is sufficiently close to $ \xi $.
    
\end{subproblem}

\vspace{2em}
\begin{solution}
    
    Let $ g(x) = x - \lambda f(x) $. We want to find $ \lambda $ so that it restrict $ |g'(x)|< 1 $.

    \[
        f'(x) = -20 \cdot \frac{1}{1 + \left(\frac{1}{x}\right)} \cdot (-\frac{1}{x^2}) = \frac{20}{1 + x^2}
    \]
    so $ f' $ is continuous and non-zero in the range $ [5,10] $.
    Then,
    \[
        g'(x) = 1 - \lambda f'(x) = 1 - \frac{20 \lambda}{1 + x^2}
    \]
    Thus, only need to choose $ \lambda $ such that $ |g'(x)| < 1 $, then by Theorem 1.5 the relaxation iteration converges.

    \begin{align*}
        |1-\lambda f'(x)| < 1 \\
        -1 < 1 - \lambda f'(x) < 1 \\
        0 < \lambda f'(x) < 2
    \end{align*}

    Since $ f'(x) $ is positive when $ x \in [5, 10] $, we got $ 0 < \lambda < \frac{2}{f'(x)} $. Since $ f'(x) $ is monotonically decreasing when $ x \in [5, 10] $, $ \lambda < \frac{2}{f'(5)} $ will make $ \lambda < \frac{2}{f'(x)} $ for all $ x \in {5, 10} $. So $ \lambda \in (0, \frac{2}{f'(5)}) = (0, 2.6) $.

    
    \vspace{2em}

    Choose $ \lambda = 2.5 $, then 
    \[
        \begin{cases}
            g'(5) = \frac{1}{26} \\
            g'(10) = \frac{76}{101}
        \end{cases}
    \]

    Then we know that $ |g'(x)| < 1 $ for $ x \in [5,10] $ if $ \lambda = 2.5 $. Therefore, 
    \[
        x_{k+1} = x_{k} - 2.5 f(x)
    \]
    converges if $ x_0 $ is sufficiently close to $ \xi $.

\end{solution}

\newpage
\begin{problem}
    In this problem, we will study the Frobenius norm.
\end{problem}

\begin{subproblem}

    Let us define for $ A, B \in \mathbb{R}^{m \times m} $
    \[
        A : B = \sum_{i=1}^{m} \sum_{j=1}^{n} a_{ij}b_{ij}, \hspace{2em} \left\lVert A \right\rVert _F = (A : A)^{\frac{1}{2}}
    \]
    Here $ a_{ij} $ denotes the $ (i, j) $-th entry of $ A $, and $ b_{ij} $ denotes the $ (i, j) $-th entry of $ B $. 

    \noindent Verify the identity for $ A, B \in \mathbb{R}^{m \times n} $,
    \[
        A : B = \frac{1}{4} (\left\lVert A + B \right\rVert _F^2 - \left\lVert A - B \right\rVert _F^2).
    \]

\end{subproblem}

\vspace{1em}
\begin{solution}
    
    \begin{align*}
        \frac{1}{4} (\left\lVert A + B \right\rVert _F^2 - \left\lVert A - B \right\rVert _F^2)
        &= \frac{1}{4} \left(\sum_{i=1}^{m} \sum_{j=1}^{n} (a+b)_{ij}^2 - \sum_{i=1}^{m} \sum_{j=1}^{n} (a-b)_{ij}^2 \right) \\
        &= \frac{1}{4} \sum_{i=1}^{m} \sum_{j=1}^{n} \left( (a_{ij}+b_{ij})^2 -  (a_{ij}-b_{ij})^2 \right) \\
        &= \frac{1}{4} \sum_{i=1}^{m} \sum_{j=1}^{n} \left( a_{ij}^2 + 2a_{ij}b_{ij} + b_{ij}^2 - (a_{ij}^2 - 2a_{ij}b_{ij} + b_{ij}^2) \right) \\
        &= \frac{1}{4} \sum_{i=1}^{m} \sum_{j=1}^{n} \left( 4a_{ij}b_{ij} \right) \\
        &= \sum_{i=1}^{m} \sum_{j=1}^{n} \left( a_{ij}b_{ij} \right) \\
        & = A : B
    \end{align*}

\end{solution}

\vspace{2em}
\begin{subproblem}
    
    Note that
    
    \begin{itemize}
        \item for any orthogonal matrix $ Q \in R^{m \times m} $, $ \left\lVert QA \right\rVert _F = \left\lVert A \right\rVert _F $
        \item $ \left\lVert A \right\rVert _F = \left\lVert A^T \right\rVert _F $.
    \end{itemize}
    Using these facts, show that $ \left\lVert A \right\rVert _2 \leqslant \left\lVert A \right\rVert _F $.

\end{subproblem}

\vspace{1em}
\begin{solution}
    
    Since $ \left\lVert A \right\rVert _2 = \sup_{v \in \mathbb{R}^n \backslash \{0\}} \frac{\left\lVert Av \right\rVert _2}{\left\lVert v \right\rVert _2} $, assume $ v_1 \in \mathbb{R}^n $ makes the $ \frac{\left\lVert Av_1 \right\rVert _2}{\left\lVert v_1 \right\rVert _2} $ reaches the supreme. By property of linear algebra, we can extend $ v_1 $ to $ v_1, \cdots , v_{n} $ where $ v_1, \cdots , v_{n} $ are linearly independent.

    \vspace{1em}
    Then, perform a Gram-Schmidt process on $ v_1, \cdots , v_{n} $ so that we got $ q_1, \cdots , q_{n} $ back, where $ q_1, \cdots , q_{n} $ is a orthonormal basis.

    let $ Q = [q_1 | q_2 | \cdots | q_{n}] $, then $ Q \in R^{n \times n} $ and $ Q $ is orthogonal. $ AQ \in R^{m \times n} $. By Lemma 3.6 we know that $ \left\lVert AQ_{n} \right\rVert _F = \left\lVert A \right\rVert _F $. Then
    \[
        \left\lVert A \right\rVert _F^2 = \left\lVert AQ \right\rVert _F^2
        = \sum_{i=1}^{m} \sum_{j=1}^{n} [(AQ)_{ij}]^2
        = \sum_{j=1}^{n} \sum_{i=1}^{m} [(AQ)_{ij}]^2
        = \sum_{j=1}^{n} \left\lVert Aq_{j} \right\rVert _2^2
        \geqslant \left\lVert Aq_{1} \right\rVert _2^2
    \]

    If $ \left\lVert Aq_{1} \right\rVert _2 = \left\lVert Av_{1} \right\rVert _2 $ then we are done. Since by the Gram-Schmidt process, $ q_1 = \frac{v_1}{\left\lVert v_1 \right\rVert} _2 $, and our choice of $ v_1 $, then 
    \[
        \left\lVert Av_1 \right\rVert _2 = \frac{\left\lVert Av_1 \right\rVert _2}{\left\lVert v_1 \right\rVert _2} = \frac{\left\lVert v_1 \right\rVert _2 \left\lVert Aq_1 \right\rVert _2}{\left\lVert v_1 \right\rVert _2} = \left\lVert Aq_1 \right\rVert _2
    \]

    Therefore, we have $ \left\lVert A \right\rVert _F^2 \geqslant \left\lVert Av_1 \right\rVert _2^2 $, which shows 
    $ \left\lVert A \right\rVert _2 \leqslant \left\lVert A \right\rVert _F $.

\end{solution}

\end{document}
