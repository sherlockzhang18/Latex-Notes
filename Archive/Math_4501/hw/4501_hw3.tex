\documentclass[a4paper]{article}

\usepackage[margin=1.3in]{geometry} % reduce margin

% Basic packages
\usepackage[utf8]{inputenc}
\usepackage[T1]{fontenc}
\usepackage{textcomp}
\usepackage{url}
\usepackage{booktabs}
\usepackage{enumitem}
\usepackage[dvipsnames]{xcolor}
\usepackage{xifthen}

% Math packages
\usepackage{amsmath, amsfonts, mathtools, amsthm, amssymb}
\usepackage{mathrsfs}
\usepackage{cancel}
\usepackage{bm}
\usepackage{systeme}
\usepackage{stmaryrd} % for \lightning

% Math shortcuts
\newcommand\N{\ensuremath{\mathbb{N}}}
\newcommand\R{\ensuremath{\mathbb{R}}}
\newcommand\Z{\ensuremath{\mathbb{Z}}}
\renewcommand\O{\ensuremath{\emptyset}}
\newcommand\Q{\ensuremath{\mathbb{Q}}}
\newcommand\C{\ensuremath{\mathbb{C}}}
\newcommand\F{\ensuremath{\mathbb{F}}}
\DeclareMathOperator{\sgn}{sgn}
\DeclareMathOperator{\Ker}{ker}
\DeclareMathOperator{\im}{Im}

% Logic symbols
\let\svlim\lim\def\lim{\svlim\limits}
\let\implies\Rightarrow
\let\impliedby\Leftarrow
\let\iff\Leftrightarrow
\let\epsilon\varepsilon
\newcommand\contra{\scalebox{1.1}{$\lightning$}}

% Useful commands
\definecolor{correct}{HTML}{009900}
\newcommand\correct[2]{\ensuremath{\:}{\color{red}{#1}}\ensuremath{\to }{\color{correct}{#2}}\ensuremath{\:}}
\newcommand\green[1]{{\color{correct}{#1}}}

% Horizontal rule
\newcommand\hr{
    \noindent\rule[0.5ex]{\linewidth}{0.5pt}
}

% Simple theorem environments (without fancy boxes for homework)
\theoremstyle{definition}
\newtheorem{definition}{Definition}
\newtheorem{theorem}{Theorem}
\newtheorem{lemma}{Lemma}
\newtheorem{proposition}{Proposition}
\newtheorem{corollary}{Corollary}
\newtheorem*{remark}{Remark}
\newtheorem*{note}{Note}
\newtheorem*{example}{Example}

% Problem environment
\newcounter{problem}
\newenvironment{problem}[1][]
{
    \stepcounter{problem}
    \section*{Problem \theproblem\ifx\relax#1\relax\else: #1\fi}
}
{}

% Subproblem environment
\newcounter{subproblem}[problem]
\newenvironment{subproblem}[1][]
{
    \stepcounter{subproblem}
    \subsection*{(\alph{subproblem})\ifx\relax#1\relax\else\ #1\fi}
}
{}

% Solution environment
\newenvironment{solution}
{
    \noindent\textbf{Solution:}\\
}
{
    
}

% Headers
\usepackage{fancyhdr}
\pagestyle{fancy}
\fancyhf{}
\fancyhead[L]{Sherlock Zhang}
\fancyhead[C]{4501 - Homework 3}
\fancyhead[R]{\today}
\fancyfoot[C]{\thepage}

% Title info
\title{Math 4501 - Homework 3}
\author{Sherlock Zhang}
\date{\today}

\begin{document}

\maketitle

% =============================================================================
% HOMEWORK PROBLEMS START HERE
% =============================================================================

\begin{problem}

    For $ u, v \in \mathbb{R}^n $ show the following inequality,
    \[
        \frac{1}{4} \frac{\left| \left\lVert u+v \right\rVert ^2_2 - \left\lVert u-v \right\rVert ^2_2 \right|}{\left\lVert u \right\rVert _2 \left\lVert v \right\rVert _2} \leqslant 1
    \]

\end{problem}

\vspace{2em}
\begin{solution}

    First, expand the left hand side:
    \[
        \left\lVert u+v \right\rVert _2^2 = \left( \sum_{i = 1}^{n} |u_{i} + v_{i}|^2 \right) = u^2 + 2uv + v^2
    \]
    Similarly, 
    \[
        \left\lVert u-v \right\rVert _2^2 = u^2 - 2uv + v^2
    \]

    So $ \left\lVert u+v \right\rVert _2^2 - \left\lVert u-v \right\rVert _2^2 = 4uv $

    \vspace{1em}
    Putting it back to the LHS we got
    \[
        \frac{1}{4} \frac{4uv}{\left\lVert u \right\rVert _2 \left\lVert v \right\rVert _2} = \frac{uv}{\left\lVert u \right\rVert _2 \left\lVert v \right\rVert _2}
    \]

    By Cauchy-Schwarz inequality, $ uv \leqslant {\left\lVert u \right\rVert _2 \left\lVert v \right\rVert _2} $, so $ \frac{uv}{\left\lVert u \right\rVert _2 \left\lVert v \right\rVert _2} \leqslant 1 $.

\end{solution}


\newpage
\begin{problem}
    
    For any $ A \in \mathbb{R}^{m \times n} $ with linearly independent columns, show that $ \left\lVert A \right\rVert _2 = \left\lVert A^T \right\rVert _2 $.

\end{problem}

\vspace{2em}
\begin{solution}
    
    As theorem 2.12 shows that 
    \[
        \left\lVert A \right\rVert _2 = max_{i} \lambda_{i}^{\frac{1}{2}}
    \]
    where $ \lambda_i $ is eigenvalue of the matrix $ A^T A $.

    \noindent So, we know that 
    \[
        \left\lVert A \right\rVert _2 = \lambda_{\text{max}}(A^TA) \hspace{3em} \left\lVert A^T \right\rVert _2 = \lambda_{\text{max}}(AA^T)
    \]

    Therefore now we want to prove $ \lambda_{\text{max}}(A^TA) = \lambda_{\text{max}}(AA^T) $.

    let $ \lambda $ be an eigenvalue of $ A^TA $ and $ w \neq 0 $ be the corresponding eigenvector. Then,
    \[
        A^TAw = \lambda w
    \]

    set $ z = Aw $, $ z \neq 0 $ as $ A $ is linearly independent and $ \text{null}A = \{0\} $.

    So
    \[
        AA^Tz = A(A^TA)w = \lambda aw = \lambda z
    \]

    This shows that eigenvalue of $ A^TA $ is also an eigenvalue of $ AA^T $.

    \vspace{1em}
    Similarly, we can show that eigenvalue of $ AA^T $ is also an eigenvalue of $ A^TA $.

    \vspace{2em}
    Thus, 
    \[
        \lambda_{\text{max}}(A^TA) = \lambda_{\text{max}}(AA^T)
    \]

\end{solution}

\newpage
\begin{problem}
    
    Let $ A, B \in \mathbb{R}^{n \times n} $ and denote by $ I $ the identity matrix of order $ n $.

\end{problem}

\begin{subproblem}

    Show that $ \left\lVert AB \right\rVert \leqslant \left\lVert A \right\rVert \left\lVert B \right\rVert$

\end{subproblem}

\vspace{1em}
\begin{solution}
    
    \[
        \frac{\left\lVert ABx \right\rVert}{\left\lVert x \right\rVert}
        = \frac{\left\lVert A(Bx) \right\rVert}{\left\lVert Bx \right\rVert} \cdot \frac{\left\lVert Bx \right\rVert}{\left\lVert x \right\rVert}
    \]

    As $ \left\lVert A \right\rVert = \sup_{x \in R^n \backslash \{0\}} \frac{\left\lVert Av \right\rVert}{\left\lVert v \right\rVert} $, 
    \[
        \frac{\left\lVert A(Bx) \right\rVert}{\left\lVert Bx \right\rVert} \cdot \frac{\left\lVert Bx \right\rVert}{\left\lVert x \right\rVert}
        \leqslant \left\lVert A \right\rVert \cdot \frac{\left\lVert Bx \right\rVert}{\left\lVert x \right\rVert}
        \leqslant \left\lVert A \right\rVert \left\lVert B \right\rVert
    \]

\end{solution}

\vspace{2em}
\begin{subproblem}
    
    Show that if the matrix $ I -B $ is singular, then there exists a nonzero vector $ x \in \mathbb{R}^n $ such that $ (I - B)x = 0 $, and that $ \left\lVert B \right\rVert \geqslant 1 $.

\end{subproblem}

\vspace{1em}
\begin{solution}
    
    As $ I - B $ is nonsingular and it is a square matrix, its null space is not $ \{0\} $. Then, there must be some non-zero vector $ x $ such that $ (I-B)x = 0 $.

    \vspace{1em}
    This shows that $ \exists x \neq 0 $ such that $ Bx = Ix = x $.

    As $ \left\lVert B \right\rVert = \sup_{x \in R^n \backslash \{0\}} \frac{\left\lVert Bx \right\rVert}{\left\lVert x \right\rVert} $, pick the $ x $ above, we have 
    \[
        \left\lVert B \right\rVert \geqslant 
        \frac{\left\lVert Bx \right\rVert}{\left\lVert x \right\rVert} 
        = \frac{\left\lVert x \right\rVert}{\left\lVert x \right\rVert} 
        = 1
    \]

\end{solution}

\vspace{2em}
\begin{subproblem}
    
    Suppsoe that $ \left\lVert A \right\rVert < 1 $. Show that $ I - A $ is nonsingular, and that 
    \[
        (I - A)^{-1} = I + A(I - A)^{-1},
    \]
    then show that
    \[
        \left\lVert (I - A)^{-1} \right\rVert \leqslant \frac{1}{1-\left\lVert A \right\rVert} 
    \]

\end{subproblem}

\vspace{2em}
\begin{solution}
    
    If $ \left\lVert A \right\rVert < 1 $, then there is no $ x \neq 0 $ such that $ \frac{\left\lVert Ax \right\rVert}{\left\lVert x \right\rVert} = 1 $, so $ Ax \neq x = Ix $, so $ I - A $ has null space $ \{0\} $. So $ A $ is invertible.

    \[
        I = (I-A)(I-A)^{-1} = (I-A)^{-1} - A(I-A)^{-1}
    \]
    This shows 
    \[
        (I - A)^{-1} = I + A(I - A)^{-1},
    \]

    \[
        \left\lVert (I - A)^{-1} \right\rVert = \left\lVert I + A(I - A)^{-1} \right\rVert \leqslant \left\lVert I \right\rVert + \left\lVert A(I - A)^{-1} \right\rVert
    \]

    \[
        (1 - \left\lVert A \right\rVert) \left\lVert (I-A)^{-1} \right\rVert \leqslant \left\lVert I \right\rVert = 1
    \]

    So this implies 
    \[
        \left\lVert (I - A)^{-1} \right\rVert \leqslant \frac{1}{1-\left\lVert A \right\rVert} 
    \]
    
\end{solution}

\newpage
\begin{problem}
    See Jupyter Notebook for implementation.
\end{problem}

\newpage
\begin{problem}
    
    Two different norms $ \left\lVert \cdot \right\rVert $ and $ \left\lVert \cdot \right\rVert _{*} $ on $ R^n $ are saide to be equivalent if there exists constants $ c, C > 0 $ such that
    \[
        c \left\lVert v \right\rVert \leqslant \left\lVert v \right\rVert _* \leqslant C \left\lVert v \right\rVert \text{ for all } v \in \mathbb{R}^n.
    \]
    Show that the 1-norm, 2-norm, and the $ \infty $-norm are all equivalent.

\end{problem}

\vspace{2em}
\begin{solution}
    
    Relating $ \left\lVert \cdot \right\rVert _1 $ and $ \left\lVert \cdot \right\rVert _\infty $:

    pick $ c = \frac{1}{n} $ and $ C = 1 $, then 
    \[
        c \cdot \left\lVert v \right\rVert _1 = \frac{1}{n} \sum_{i = 1}^{n} |v_{i}| \leqslant \frac{1}{n} \cdot n \cdot \max_{i}|v_{i}| = \left\lVert v \right\rVert _\infty
    \]

    Also, 
    \[
        C \cdot \left\lVert v \right\rVert _1 = \sum_{i = 1}^{n} |v_{i}| \geqslant \max_{i}|v_{i}| = \left\lVert v \right\rVert _\infty
    \]


    Thus, $ \left\lVert \cdot \right\rVert _1 $ and $ \left\lVert \cdot \right\rVert _\infty $ are equivalent.

    \vspace{2em}
    Relating $ \left\lVert \cdot \right\rVert _2 $ and $ \left\lVert \cdot \right\rVert _\infty $

    As
    \[
        \left\lVert v \right\rVert _\infty^2 = \max_{i}|v_{i}|^2 \leqslant \sum_{i = 1}^{n} |v_{i}|^2 = \left\lVert v \right\rVert _2^2
    \] 
    we can pick $ c = 1 $
    
    \vspace{1em}
    Also,
    \[
        \left\lVert v \right\rVert _2^2 = \sum_{i = 1}^{n} |v_{i}|^2 \leqslant n \cdot \max_{i}|v_{i}|^2 = n \left\lVert v \right\rVert _\infty^2
    \]
    so picking $ C = \sqrt{n} $ is valid.

    Since we have proved that $ \left\lVert \cdot \right\rVert _1 \equiv \left\lVert \cdot \right\rVert _\infty $ and $ \left\lVert \cdot \right\rVert _2 \equiv \left\lVert \cdot \right\rVert _\infty $, by transitivity of equivalence relation 
    \[
        \left\lVert \cdot \right\rVert _1 \equiv \left\lVert \cdot \right\rVert _2 \equiv \left\lVert \cdot \right\rVert _\infty
    \]

\end{solution}

\end{document}
