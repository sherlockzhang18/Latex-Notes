\lesson{23}{Mon Oct 27 2025 15:00}{}

\begin{definition} [Gerschgorin's disks] \leavevmode
    
    $ A \in \mathbb{R}^{n \times n} $, Gerschgorin's Disk 
    \[
        D_{i} = \{z \in \mathbb{C} : |a_{ii}| < R_{i}\}, R_{i} = \sum_{j = 1, j \neq i}^{n}  |a_{ij}|
    \]

\end{definition}

\vspace{1em}
\begin{theorem} [Gerschgorin's Theorem] \leavevmode
    
    Given $ A \in \mathbb{R}^{n \times n} $ 
    
    \begin{enumerate}
        \item Any eigenvalue $ \lambda $ lies inside $ \bigcup^{n}_{i = 1} D_{i} $.
        \item $ D = \{D_{i}\}_{i = 1}^{n} $ has partition $ D = D^{(1)} \bigcap D^{(2)} $, $ \# D^{(1)} = p, \# D^{(2)} = q $, $ \bigcup_{D_{i} \in D^{(1)}} D_{i}  \cap \bigcup_{D_{i} \in D^{(2)}} D_{i} = \emptyset $, then there are exactly $ p $ eigenvalues in the union of disks in $ D^{(1)} $ and $ q $ eigenvalues in $ D^{(2)} $.
    \end{enumerate}
    
\end{theorem}

\vspace{1em}
\begin{proof} \leavevmode
    
    \begin{enumerate}
        \item 
            $ Ax = \lambda x $ for $ x \neq 0 $. Since $ x \neq 0 $, at least 1 entry of $ x $ is non-zero. then let $ i_* = \arg\max_{i} |x_{i}| $, then $ |x_{i_*}| > 0 $. 

            \vspace{1em}
            Then the i-th row: $ \sum_{j = 1}^{n} a_{ij}x_{j} = \lambda x_{i} $, 
            
            $ a_{i_* i_*}x_{i_*} + \sum_{j = 1, j \neq i}^{n} a_{ij}x_{j} = \lambda x_{i_*} $

            Divide by $ x_{i_*} $: 
            \begin{align*}
                |a_{ii_*} - \lambda| 
                &= \left|-\sum_{j = 1, j \neq i_*}^{n} a_{i_*j}\frac{x_{j}}{x_{i_*}}\right| \\
                &\leqslant \left(\sum_{j=1, j \neq i_{*}}^{n} |a_{i_*j}| \right) \max_{j = 1, j \neq i_*}\left(\frac{|x_{j}|}{|x_{i_*}|}\right) \\
                &\leqslant R_{i_*}
            \end{align*}


        \item 
            $ B(\epsilon) \in \mathbb{R}^{n \times n} $
            \vspace{1em}
            
            $ 
            b_{ij}(\epsilon) 
            = 
            \begin{cases}
                a_{ij} \hspace{1em} \text{ if } i = j \\
                \epsilon a_{ij} \hspace{1em} \text{ if } i \neq j
            \end{cases} 
            $

            Denote the Gerschgorin's disks of $ B(\epsilon) $ as $ D_{i}(\epsilon) $ for $ i = 1, \cdots , n $.

            then,
            \[
                \begin{cases}
                    D_{i}(\epsilon_1) \subseteq D_{i}(\epsilon_2) \text{ if } \epsilon_1 \leqslant  \epsilon_2 \\
                    D_{i}(1): \text{ Gerschgorin's disk of } A \\
                    D_{i}(0) = \{a_{ii}\}
                \end{cases}
            \]

            \vspace{1em}
            i-th eigenvalue $ \lambda_1(\epsilon) $ of $ B_{i}(\epsilon) $ is a continuous function of $ \epsilon $. (Why?)

            \vspace{1em}
            
            
            Let $ D^{(1)} (\epsilon) = \bigcup_{i: D_{i} \in D^{(1)}} D_{i}(\epsilon) $ and $ D^{(2)} (\epsilon) = \bigcup_{i: D_{i} \in D^{(2)}} D_{i}(\epsilon)$

            If $ \lambda_{i}(\epsilon) \in D^{(j)}(\epsilon) $ for some $ \epsilon \in [0,1] $

            then $ \lambda_{i}(\epsilon) \in D^{(j)}(\epsilon) $ for all $ \epsilon \in [0,1] $. Here $ j = 1 \text{ or } 2 $.

            \vspace{1em}
            So, $ \lambda_{i}(0) \in D^{(j)} (0) \iff \lambda_{i}(1) \in D^{(j)} (1) $, which is the union of any georshgorin disks in $ D^{(j)} $.
        
    \end{enumerate}

\end{proof}