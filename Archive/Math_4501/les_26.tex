\lesson{26}{Wed Nov 5 2025 15:00}{}

\begin{theorem}
    
    $ A \in \mathbb{R}^{n \times n} $ is orthogonally similar to a upper-Hessenberg matrix
    \[
        A = QKQ^T
    \]
    where $ Q = Q^{(1)} Q^{(2)} \cdots Q^{(n-2)} $, $ Q^{(k)}: $ householder matrices.

\end{theorem}

\vspace{1em}
\begin{remark}
    
    Householder matrices can be used to compute the QR factorization. 

    \[
        Q \cdot 
        \begin{bmatrix}
          x & x & x & x \\
          x & x & x & x \\
          x & x & x & x \\
          x & x & x & x \\
          x & x & x & x \\
        \end{bmatrix}(A)
        =
        \begin{bmatrix}
          x & x & x & x \\
          0 & x & x & x \\
          0 & 0 & x & x \\
          0 & 0 & 0 & x \\
          0 & 0 & 0 & 0 \\
        \end{bmatrix}(R)
    \]

    \vspace{2em}
    Third way to compute using rotational matrix:
    \[
        \cdots R_{1,3}^{(\varphi)(2)}R_{1,2}^{(\varphi)(1)} = 
        \begin{bmatrix}
          x & x & x \\
          x & x & x \\
          x & x & x \\
          x & x & x \\
        \end{bmatrix}(A)
        =
        \begin{bmatrix}
          x & x & x \\
          0 & x & x \\
          0 & 0 & x \\
          0 & 0 & 0 \\
        \end{bmatrix}(R)
    \]

\end{remark}


\vspace{2em}
{\textbf{Why tri-diagonalize ?}}

\vspace{1em}
Tri-diagonalize before the Jacobi's method.