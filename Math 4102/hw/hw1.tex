\documentclass[a4paper]{article}

% \usepackage[margin=1in]{geometry} % reduce margin

% Basic packages
\usepackage[utf8]{inputenc}
\usepackage[T1]{fontenc}
\usepackage{textcomp}
\usepackage{url}
\usepackage{booktabs}
\usepackage{enumitem}
\usepackage[dvipsnames]{xcolor}
\usepackage{xifthen}

% Math packages
\usepackage{amsmath, amsfonts, mathtools, amsthm, amssymb}
\usepackage{mathrsfs}
\usepackage{cancel}
\usepackage{bm}
\usepackage{systeme}
\usepackage{stmaryrd} % for \lightning

% Math shortcuts
\newcommand\N{\ensuremath{\mathbb{N}}}
\newcommand\R{\ensuremath{\mathbb{R}}}
\newcommand\Z{\ensuremath{\mathbb{Z}}}
\renewcommand\O{\ensuremath{\emptyset}}
\newcommand\Q{\ensuremath{\mathbb{Q}}}
\newcommand\C{\ensuremath{\mathbb{C}}}
\newcommand\F{\ensuremath{\mathbb{F}}}
\DeclareMathOperator{\sgn}{sgn}
\DeclareMathOperator{\Ker}{ker}
\DeclareMathOperator{\im}{Im}

% Logic symbols
\let\svlim\lim\def\lim{\svlim\limits}
\let\implies\Rightarrow
\let\impliedby\Leftarrow
\let\iff\Leftrightarrow
\let\epsilon\varepsilon
\newcommand\contra{\scalebox{1.1}{$\lightning$}}

% Useful commands
\definecolor{correct}{HTML}{009900}
\newcommand\correct[2]{\ensuremath{\:}{\color{red}{#1}}\ensuremath{\to }{\color{correct}{#2}}\ensuremath{\:}}
\newcommand\green[1]{{\color{correct}{#1}}}

% Horizontal rule
\newcommand\hr{
    \noindent\rule[0.5ex]{\linewidth}{0.5pt}
}

% Simple theorem environments (without fancy boxes for homework)
\theoremstyle{definition}
\newtheorem{definition}{Definition}
\newtheorem{theorem}{Theorem}
\newtheorem{lemma}{Lemma}
\newtheorem{proposition}{Proposition}
\newtheorem{corollary}{Corollary}
\newtheorem*{remark}{Remark}
\newtheorem*{note}{Note}
\newtheorem*{example}{Example}

% Problem environment
\newcounter{problem}
\newenvironment{problem}[1][]
{
    \stepcounter{problem}
    \section*{Problem \theproblem\ifx\relax#1\relax\else: #1\fi}
}
{}

% Subproblem environment
\newcounter{subproblem}[problem]
\newenvironment{subproblem}[1][]
{
    \stepcounter{subproblem}
    \subsection*{(\alph{subproblem})\ifx\relax#1\relax\else\ #1\fi}
}
{}

% Solution environment
\newenvironment{solution}
{
    \noindent\textbf{Solution:}\\
}
{
    
}

% Headers
\usepackage{fancyhdr}
\pagestyle{fancy}
\fancyhf{}
\fancyhead[L]{Sherlock Zhang}
\fancyhead[C]{Math 4102 - Homework 1}
\fancyhead[R]{ID: 518915}
\fancyfoot[C]{\thepage}

% Title info
\title{Math 4102 - Homework 1}
\author{Sherlock Zhang}
\date{\today}

\begin{document}

\maketitle

% =============================================================================
% HOMEWORK PROBLEMS START HERE
% =============================================================================

\begin{problem}[Ex.5.6]

    Let $ f(x) = 1 $ for $ x = \frac{1}{n}, n = 1, 2, \cdots , $ and suppose $ f(x) = 0 $ otherwise. Show that $ f $ is Riemann integrable and that $ \int_{0}^{1}f = 0 $.

\end{problem}

\vspace{2em}
\begin{solution}

    Let $ f_1 = 0 $, then $ \int_{0}^{1} f_1 dx = 0 $.

    Need to choose another step function $ f_2 $ so that $ f_2 > f $ and $ \int_{0}^{1} f_2 < \epsilon $ for any $ \epsilon $ so that $ \int_0^1f_2 - \int_0^1f_1 < \epsilon $.

    First choose $ N $ large enough so that $ \frac{1}{N} < \frac{\epsilon}{2} $. Then set $ f_2(x) = 1 $ on $ (0, \frac{1}{N}] $ so that for portion of $ n > N $, $ f_2 > f $. 
    
    The rest points with non-zero values in $ f $ is finite: $ \{1, \frac{1}{2}, \cdots , \frac{1}{N_1}\} $. Let $ k = 1, 2, \cdots N-1 $. For each point, pick $ I_{k} $ such that $ \frac{1}{k} \in I_{k} \subset (0,1] $ and $ \sum_{k=1}^{N-1} |I_{k}| < \frac{\epsilon}{2} $, such as choose $ I_{k} = \frac{\epsilon}{2^k+1} $ .


    \vspace{1em}
    Therefore, we can define
    \[
        f_2 = 
        \begin{cases}
            1, &x \in (0, \frac{1}{N} \text{ or } x \in \cup_{k=1}^{N-1} I_{k}) \\
            0, \text{otherwise}
        \end{cases}
    \]

    \vspace{1em}
    Thus, there exists step functions $ f_1, f_2 $ such that 
    
    $ f_1 \leqslant f \leqslant f_2 $ and $ \int_{0}^{1}f_2(x)dx - \int_{0}^{1}f_1(x) dx < \epsilon $. 
    
    Use theorem 1.6 to conclude that $ f $ is Riemann Integrable.

    Known $ \int_{0}^{1} f $ is non-negative 
    
    by Corollary 1.7, $ \int_{0}^{1} = \inf\{\int_{0}^{1} f'(x)dx | f' \text{ is a step function and } f' \geqslant  f\} $.

    as $ f_2 < \epsilon $ for all $ \epsilon > 0 $, $ \inf \{f_2\} = 0 $. Therefore $ \int_{0}^{1} f_2 \leqslant 0 $. Thus $ \int_{0}^{1} f = 0 $.

\end{solution}

\newpage
\begin{problem}[Ex 5.20]

    Invent a function which is monotone on $ [0,1] $ but is not piecewise continuous.

\end{problem}

\vspace{2em}
\begin{solution}
    
    \[
        f(x) = \sum_{n=1}^{\infty} 2^{-n} I_{n}(x)  
    \]
    where $ I_{n}(x) = \begin{cases}
        1, & x \geqslant q_{n} \\
        0, & x < q_{n}
    \end{cases} $

    Here $ q_{n} $ is listing all rational numbers in $ [0,1] $ in a sequence.

    \vspace{1em}
    In this function it is non-decreasing, but there are infinitely many jump discontinuity at every rational numbers.

\end{solution}

\newpage
\begin{problem}[Ex. 5.21]
    
    Let $ \chi_{A} $ denote the characteristic function of set $ A $:
    \[
        \chi_{A}(x) = 
        \begin{cases}
            1, & x \in A \\
            0, & x \notin A \\
        \end{cases} 
    \]

\end{problem}

\begin{subproblem}
    Prove that for any sets $ A, B $, $ \chi_{A \cap B} = \chi_{A} \chi_{B} $.
\end{subproblem}

\vspace{1em}
\begin{solution}
    There are two cases to explore
    Case 1: $ x \in A \cap B $:

    Then $ x \in A $ and $ x \in B $, so $ \chi_{A} \chi_{B} = 1 = \chi_{A \cap B} $.

    \vspace{1em}
    Case 2: $ x \notin A \cap B $:
    Then $ x $ is not in at least one set of $ A $ or $ B $. either $ \chi(A) = 0 $ or $ \chi(B) = 0 $.
    Therefore $ \chi_{A}\chi_{B} = 0 = \chi_{A \cap B} $.

\end{solution}

\vspace{1em}
\begin{subproblem}
    
    for $ \chi_{A \cup B} $, it equals to 1 iff $ x \in A \text{ or } B $. So we can use adding to see if $ x $ is in either set. But if $ x $ is in both then it might add up to 2, so we take that off by subtracting $ \chi_A\chi_{B} $ as in (a) we showed that $ \chi_{A \cap B} = \chi_{A} \chi_{B} $.
    \[
        \chi_{A \cup B} = \chi_{A} + \chi_{B} - \chi_A\chi_{B}
    \]

    \vspace{1em}
    For $ \chi_{A \backslash B} $, it equals to 1 iff $ \chi_{A} = 1 $ and $ \chi_{B} = 0 $. Therefore, 
    \[
        \chi_{A \backslash B} = \chi_{A}(1-\chi_{B})
    \]

\end{subproblem}

\vspace{1em}
\begin{subproblem}
    
    Show that $ \chi_{A} + \chi_{B} = \chi_{A \cap B} + \chi_{A \cup B} $

\end{subproblem}

\vspace{1em}
\begin{solution}
    
    From (b) we constructed that $ \chi_{A \cup B} = \chi_{A} + \chi_{B} - \chi_A\chi_{B} $, and in (a) we proved that $ \chi_{A \cap B} = \chi_{A} \chi_{B} $. Therefore, adding them up we geometry
    \[
        \chi_{A \cap B} + \chi_{A \cup B} = \chi_{A} + \chi_{B} - \chi_A\chi_{B} - \chi_{A} \chi_{B} = \chi_{A} + \chi_{B}
    \]

\end{solution}

\newpage
\begin{problem}[Ex. 5.25]
    
    Let $ \epsilon > 0 $. As $ f_{n} \to f $ uniformly, pick $ N $ large such that for all $ x \in [a,b] $, 
    \[
        |f(x) - f_{N}(x)| < \frac{\epsilon}{3(b-a)}
    \]

    That is $ \left\lVert f - f_{N} \right\rVert _\infty < \frac{\epsilon}{3(b-a)} $.

    Since $ f_{N} $ is R.I., there exists $ \delta $ such that any $ P $ with $ \left\lVert P \right\rVert < \delta $ and any Rieman Sum $ R(f_{N}, P) $:
    \[
        |R(f_{N}, P) - \int_{a}^{b} f_{N}| < \frac{\epsilon}{3}
    \]

    Then we can take any $ P $ with $ \left\lVert P \right\rVert < \delta $
    \begin{align*}
        \left|R(f,p) - \int_{a}^{b}f_{N}\right| 
        & \leqslant |R(f,P) - R(f_{N}, P)| + \left|R(f_{N}, P) - \int_{a}^{b}f_{N}\right| \\
        & \leqslant \left\lVert f - f_{N} \right\rVert _\infty(b-a) + \frac{\epsilon}{3}\\
        & < \frac{2\epsilon}{3}
    \end{align*}

    Therefore, $ f $ is R.I.

    \vspace{2em}
    Since $f_n \to f$ uniformly on $[a,b]$, we have
\[
\|f_n - f\|_\infty = \sup_{x\in[a,b]} |f_n(x)-f(x)| \to 0.
\]

Let $\varepsilon > 0$. Then there exists $N$ such that for all $n \ge N$, $ \|f_n - f\|_\infty < \frac{\varepsilon}{b-a} $

Now for $n \ge N$,
\[
\left| \int_a^b f(x)\,dx - \int_a^b f_n(x)\,dx \right|
= \left| \int_a^b \big(f(x) - f_n(x)\big)\,dx \right|.
\]

as $|g(x)| \le \|g\|_\infty$ for all $x$,
\[
\left| \int_a^b (f - f_n) \right|
\le \int_a^b |f - f_n|
\le \int_a^b \|f - f_n\|_\infty
= (b-a)\|f - f_n\|_\infty
\]

Hence,
\[
    \left| \int_a^b f - \int_a^b f_n \right|
    < (b-a)\frac{\varepsilon}{b-a} = \varepsilon
\]

Therefore,
\[
    \lim_{n\to\infty} \int_a^b f_n(x)\,dx = \int_a^b f(x)\,dx
\]


\end{problem}

\newpage
\begin{problem}[Ex. 9.3]
    
    Prove that if $ \mu $ is a measure on $ S $ and $ \{x\} \in S $ for every $ x \in [a,b] $, and $ \mu(\{x\}) = \mu(\{y\}) $ for all $ x, y \in [a,b] $, then $ \mu(Q) = 0 $

\end{problem}

\vspace{2em}
\begin{solution}
    
    Since $ \mu(\{x\}) = \mu(\{y\}) $ for all $ x, y \in [a,b] $, let that common value be:
    \[
        c = \mu(\{x\})
    \]

    Pick $ n $ distinct points $ x_1, \cdots , x_{n}  \in [a,b]$, then since they are disjoint singletons, 
    \[
        \mu(\{x_1, \cdots , x_{n}\}) = \sum_{i=1}^{n} \mu(\{x_{i}\}) = nc
    \]

    since $ \{x_1, \cdots , x_{n}\} \subset [a,b] $, 
    \[
        nc \leqslant \mu(a,b)
    \]

    as $ \mu[a,b] \leqslant b-a $ is finite, for all $ n $ it is impossible  because $ nc \to \infty $ as $ n \to \infty $. Therefore, $ c = 0 $.

    \vspace{2em}
    The set $ Q \cap [a,b] $ is countable, so list it as $ \{q_1, q_2, \cdots \} $. 
    Then
    \[
        \mu(Q) = \sum_{i=1}^{\infty} \mu(\{q_{i}\})  = \sum_{i=1}^{\infty} 0 = 0
    \]

\end{solution}

\newpage
\begin{problem}[Ex. 9.13]

    Do there exist open subsets $ G_1, G_2 $ of $ E $ such aht $ G_1 \neq G_2 $ but $ \mu^*(G_1) = \mu^*(G_2) $?

\end{problem}

\vspace{2em}
\begin{solution}
    
    Yes, take
    \[
        G_1 = (0,\frac{1}{2}) \text{ and } G_2 = (0,\frac{1}{4}) \cup (\frac{3}{4}, 1)
    \]

    \vspace{2em}
    Both $ G_1 $ and $ G_2 $ are open sets and they are not equal, but they both have $ \mu^* = \frac{1}{2} $.
    
\end{solution}

\newpage
\begin{problem}[Ex 9.15]
    
    Prove that $ \mu^* $ is countably additive on the class of open subsets of $ E $.

\end{problem}

\vspace{2em}
\begin{solution}
    
    Let $G_1,G_2,\dots$ be pairwise disjoint open subsets of $E$, and set
    $G=\bigcup_{n=1}^\infty G_n$. Then $G$ is open.

    By Theorem~7.2, there exist  disjoint open intervals for each $ G_{n} $ as 
    $\{(a_{n,k},b_{n,k})\}_{k\ge1}$ such that
    \[
    G_n=\bigcup_{k=1}^\infty (a_{n,k},b_{n,k}).
    \]
    By Definition~7.3, $m^*(G_n)=\sum_{k=1}^\infty (b_{n,k}-a_{n,k})$.

    Since the sets $G_n$ are disjoint, any interval $(a_{n,k},b_{n,k})\subset G_n$
    is disjoint from any interval $(a_{m,\ell},b_{m,\ell})\subset G_m$ whenever
    $m\neq n$. 
    
    Therefore the group $\{(a_{n,k},b_{n,k}) : n\ge1,\ k\ge1\}$ is pairwise
    disjoint and
    \[
    G=\bigcup_{n=1}^\infty G_n=\bigcup_{n=1}^\infty\bigcup_{k=1}^\infty (a_{n,k},b_{n,k})
    =\bigcup_{n,k}(a_{n,k},b_{n,k}).
    \]

    By the uniqueness Theorem 7.2, this is the decomposition of $G$ into
    pairwise disjoint open intervals, so
    \[
    m^*(G)=\sum_{n,k}(b_{n,k}-a_{n,k}).
    \]
    Since all terms are nonnegative, we may group the sum by $n$ to obtain
    \[
    m^*(G)=\sum_{n=1}^\infty\sum_{k=1}^\infty(b_{n,k}-a_{n,k})
    =\sum_{n=1}^\infty m^*(G_n).
    \]
    Therefore $m^*$ is countably additive on the class of open subsets of $E$.

\end{solution}

\newpage
\begin{problem}[Ex 9.17]
    
    Show that if $ A \subset B \subset E $, then $ \mu^*(A) \leqslant \mu^*(B) $.

\end{problem}

\begin{solution}
    
    Assume $A\subset B\subset E$. Define the open sets
    $ G_A=\{G\subset E: G \text{ open and } A\subset G\}$ 
    
    and $ G_B=\{G\subset E: G \text{ open and } B\subset G\}$. Then $A\subset B$
    implies $ G_B\subset  G_A$, since $B\subset G$ gives $A\subset G$.

    By Definition~7.4 we have
    \[
    \mu^*(A)=\inf\{\mu^*(G): G\in G_A\},\qquad
    \mu^*(B)=\inf\{\mu^*(G): G\in G_B\}.
    \]
    Let $S_A=\{\mu^*(G): G\in G_A\}$ and $S_B=\{\mu^*(G): G\in G_B\}$.
    From $ G_B\subset G_A$ we get $S_B\subset S_A$. For sets of real
    numbers, $S_B\subset S_A$ implies $\inf S_A\le \inf S_B$ since $\inf S_B$ is a lower
    bound for $S_B$, hence also a lower bound for $S_A$ only after possibly decreasing,
    so the glb of $S_A$ cannot exceed glb $S_B$. Therefore
    \[
    \mu^*(A)=\inf S_A\le \inf S_B=\mu^*(B).
    \]
    
\end{solution}

\newpage
\begin{problem}[Ex 9.21]
    
    In Example 7.7, prove that if $ E_{\alpha} \cap E_{\beta} \neq \empty $, then $ E_{\alpha} = E_{\beta} $.

\end{problem}

\vspace{2em}
\begin{solution}
    
    Assume there is $ x \in  E_{\alpha} \cap E_{\beta}$, then we have
    \[
        x - \alpha \in Q \text{ and } x - \beta \in Q
    \]

    Therefore, 
    \[
        (x - \alpha) - (x - \beta) = \beta - \alpha \in Q
    \]

    Let $ y \in E_{\alpha} $, then $ y - \alpha \in Q $. Since $ \alpha - \beta \in Q $, we get
    \[
        y - \beta = y - \alpha + (\alpha - \beta) \in Q
    \]
    Therefore $ y \in E_{\beta} $, $ E_{\alpha} \subseteq E_{\beta} $.

    By symmetry claim, we get $ E_{\beta} \subseteq E_{\alpha} $.

    Therefore, $ E_{\alpha} = E_{\beta} $.

\end{solution}

\newpage
\begin{problem}[Ex. 9.27]
    
    Prove that if $ A, B \subset E $ and $ \mu^*(A) + \mu^*(B) = \mu_*(A) + \mu_*(B) $, then $ A $ and $ B $ are measurable. (Hint: Lemma 8.7)
    
\end{problem}

\vspace{2em}
\begin{solution}

    Lemma 8.7 shows that 
    \[
        \mu^*(A) \geqslant \mu_*(A) \text{ and } \mu^*(B) \leqslant \mu_*(B)
    \]
    so we get 
    \[
        \mu^*(A) + \mu^*(B) \geqslant  \mu_*(A) + \mu_*(B)
    \]

    Let $ a = \mu^*(A) - \mu_*(A) \geqslant 0 $ and $ b = \mu^*(B) - \mu_*(B) \geqslant 0 $

    It reaches equality when $ a + b = 0 $, but both are non-negative. Therefore, $ a = b = 0 $

    Thus
    \[
        \mu^*(A) = \mu_*(A) \text{ and } \mu^*(B) = \mu_*(B)
    \]

    Therefore, by definition 8.2 $ A $ and $ B $ are measurable.

\end{solution}

\end{document}
