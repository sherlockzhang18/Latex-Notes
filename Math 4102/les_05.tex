\lesson{5}{Fri Jan 23 2026 10:00}{}

\begin{eg}
    Define \underbar{a set function} to be $ f: S \to \mathbb{R} $ 
    $ f(s) = 
    \begin{cases}
        1, & \text{ if } \frac{1}{2} \in S \\
        0, &\text{otherwise} 
    \end{cases} $

    \vspace{1em}
    Check countable additivity: Suppose $ A = \dot\cup^\infty_{i = 1} A_{i} $, $ A_{i} \subseteq [0,1] $, then $ \mu(A) = 1 \iff \frac{1}{2} \in A \iff (\exists i) \frac{1}{2} \in A_{i} \iff \sum_{i=1}^{\infty} \mu(A_{i}) = 1 $.
\end{eg}

\begin{eg}
    Trivial measure $ \mu(S) = 0 $, all $ S \subseteq [0,1] $
\end{eg}

\begin{eg}
    Try $ \mu(A) = \int_{a}^{b} X_{A}(x) dx$ where $ X_{A}(x) = \begin{cases}
        1, & x \in A \\
        0, & \text{otherwise}
    \end{cases} $

    Works for some $ A $ not others:

    \begin{itemize}
        \item $ X_{Q} $ is not R.I. on $ [0,1] $, since for any step functions $ f_1\leqslant X_{Q} \leqslant  f_2 $ we have $ \int_{0}^{1}f_1 = 0 < 1 = \int_{0}^{1} f_2 $
        \item Try $ \mu(A) = \inf \{\int_{0}^{1} f(x)dx | f \text{ is step function } f_{n}, X_{A} \leqslant f\} $. (Called "Jordan Constant") set function, monotonic
        
        but not countably additive: Let $ Q \cap [0,1] = \{q_1, q_2, \cdots \} $ and observe that $ Q \cap [0,1] = \dot\cup^{\infty}_{i=1} \{q_{i}\} $. 

        But $ \mu(A) = 1 \neq \sum_{i=1}^{\infty} \mu(A_{i}) = 0   $
    \end{itemize}

    
\end{eg}


\begin{definition} [Lebesgue Outer measure $ \mu^* $] \leavevmode

    \begin{itemize}
        \item Should agree with length for intervals, so put $ \mu^*([a,b]) = b - a = \mu^*((a,b)) = \mu^*([a,b)) = \mu^*((a,b]) $.
        \item Generalize to all open subsets of $ \mathbb{R} $ using:
    \end{itemize}
    
\end{definition}

\begin{theorem}[theorem 7.2] \leavevmode

    If $ G \subseteq \mathbb{R} $ is open then there exists a \underbar{unique countable} collection $ \{G_{i}\} $ of disjoint intervals such that
    \[
        G = \dot\cup^{\infty}_{i = 1} G_{i}
    \]

    Note this embraces finite unions, since $ \emptyset \cup \emptyset = \emptyset $
\end{theorem}


\begin{proof}
    
    Given $ x \in G $, define its component open interval 
    \[
        I_{x} = (a_{x}, b_{x}) \text{ where } \begin{cases}
            a_{x} = \inf\{y: (y,x) \subseteq G\} \text{ (could be )-} \infty\\
            b_{x} = \sup\{y: (x,y) \subseteq G\} \text{ (could be )+} \infty\\
        \end{cases}
    \]

    eg: $ G = \cup^{\infty}_{n = 1} (2^{-n}, 2^{-(n-1)}) $

    \vspace{1em}
    Claim: For any $ x, y \in G $, either $ I_{x} = I_{y} $ or $ I_{x} \cap I_{y} = \emptyset $. 

    So we could write $ G = \dot\cup_{x\in G} I_{x} $ if we ignore duplicates.

    But $ Q = \{q_1, q_2, \cdots \} $ is countable, so let us choose one point in $ Q \cap I_{x} $, each $ x $, and call it $ x_{i} $. Then $ \dot\cup_{i = 1}^{\infty} I_{x_{i}} $ is countable and disjoint and contains $ G $, so put $ G_{i} = I_{x_{i}} $ to get the decomposition.

    \vspace{1em}
    For \underbar{uniqueness}, suppose $ G = \dot\cup G_{i} = \dot\cup H_{i} $, then for each $ x \in G (\exists i) x \in G_{i} $ and $ x\in H_{j} $, for some unique $ i $ and $ j $.

    But $ H_j $ is a component interval so $ G_{i} \subseteq H_{j} $. Likewise $ G_{i} $ is a component interval so $ H_{j} \subseteq G_{i} $. This means $ G_{i} = H_{j} $. So both collections of intervals are the same.

\end{proof}

\vspace{2em}
\underbar{\textbf{Applications:}}

\begin{definition} \leavevmode

    $ \mu^*(G) = \sum_{i=1}^{\infty} b_{i} - a_{i} $ where $ G = \dot\cup _{i = 1}^\infty (a_{i}, b_{i}) $ ($ G_{i} $ component open interval). 

    Finally, for arbitrary $ S \subseteq \mathbb{R} $, put $ \mu^*(S) = \inf\{\mu^*(G) | \text{ open } G \subseteq \mathbb{R} \text{ with } S \subseteq G \} $.

    \underbar{\textbf{Properties}}:
    \begin{itemize}
        \item $ \mu^* (A + x) = \mu^*(A) $ (interval under transition)
        \item $ (\forall \epsilon > 0)(\forall A \subseteq [0,1])(\exists G \subseteq [0,1]) \mu^*(A) \leqslant \mu^*(G) + \epsilon $
    \end{itemize}

\end{definition}

\begin{eg}

    $ \mu^*(Q) = 0 $ in fact $ \mu^*(S) = 0 $ for any countable subsets.

\end{eg}